\section{Higher Order Unification}

Checking for the reducability of two full lambda terms has long been known to be only semi-decidable.
The matter becomes even more complicated when checking for the equality of terms with variables bound
by both existential and universal quantifiers.  Research from the past thirty years has constrained
the problem to a decidable subset known as the pattern fragment.

\subsection{Unification Terms}

\begin{definition}
Unification Terms:

\[
U ::= U \wedge U 
 \orr \forall V : T . U
 \orr \exists V : T . U 
 \orr T \doteq T
 \orr \top
\]
\label{def:hou:syn}
\end{definition}

When $\doteq$ is taken to mean $\equiv_{\beta\eta\alpha*}$, the unification problem is to determine 
whether a statement $U$ is ``true'' in the standard sense, and provide a proof of the truth of the statement. 

Unification problems of the form 
$\forall x : T_1 . \exists y : T_2 . U $ can be solved by solving those of the form
$\exists y : \Pi x : T_1 . T_2 . \forall x : T_1 . [y\; x / y ]U $ 
in the process known as raising. 
Unification statements can always quantified over unused variables: $U \implies Q x : T . U$ (where $Q ::= \exists | \forall $).  

Thus, statements can always be converted to the form
\[
\exists y_1 \cdots y_n . \forall x_1 \cdots x_k . S_1 \doteq V_1 \wedge \cdots S_r \doteq V_r
\]




\subsection{Unification Term Meaning}

We can provide an provability relation of a unification formula
based on the obvious logic.

\begin{definition}
$\Gamma \Vdash F $ can be interpreted as $\Gamma$ implies $F$ 
is provable.

\[ \begin{array}{lcr}
\infer[\m{equiv}]{
\Gamma \Vdash M \doteq N
}{
\Gamma \vdash M : A
&
M \equiv_{\beta\eta\alpha*} N
&
\Gamma \vdash N : A
}
&
\infer[\m{true}]{
\Gamma \Vdash \top
}{}
&
\infer[\m{conj}]{
\Gamma \Vdash F \wedge G
}{
\Gamma \Vdash F
&
\Gamma \Vdash G
}
\end{array} \]

\[ \begin{array}{lr}
\infer[\m{exists}]{
\Gamma \Vdash \exists x : A . F
}{
\Gamma \Vdash [M/x] F
&
\Gamma \vdash M : A
}
&
\infer[\m{forall}]{
\Gamma \Vdash \forall x : A . F
}{
\Gamma, x : A \Vdash F
}
\end{array} \]

\label{def:hou:prf}
\end{definition}

While a truly superb logic programming language might 
be able to convert this very declarative 
specification into a runnable program, 
the essentially nondeterministic rule for existential
quantification in a unification formula prevents an 
obvious deterministic algorithm from being extracted.


\subsection{Higher Order Unification for CC}

\newcommand{\UnifiesTo}{\;\longrightarrow\;}

I now present an algorithm, similar to that presented in 
\citep{pfenning1991logic} for unification in the 
Calculus of Constructions.  Because we have already 
presented typed hereditary substitution with $\eta$-expansion, 
the presentation here will not add much 
but for types in the substitutions.  

$F \UnifiesTo F'$ shall mean that $F$ can be transformed to $F'$
without modifying the provability. 
An equation $F[G]$ will stand as notation for highlighting $G$
under the formulae context $F$.  
As an example, if we were to examine the formula 
$\forall x . \forall n . \exists y . ( y \doteq x \wedge \forall z . \exists r . [ x z \doteq r] )$
but were only interested in the last portion, we might instead write it as
$\forall x . F[\forall z . \exists r . [ x z \doteq r]]$
Again, $\phi$ shall be an injective partial permutation. 

Furthermore, rather than explicitly writing down the result of unification, 
we shall use $\exists x. F \UnifiesTo \exists x . [ L / x] F$ 
to stand for $\exists x. F \UnifiesTo \exists x . x \doteq L \wedge [ L / x] F$

The unification rules are symmetric, so any rule of the form 
$M \doteq N$ is equivalent to $N \doteq M$ practically.

Also, for the purpose of typed normalizing heredetary substitution, 
a formula prefix $F[e]$ of the form 
$Qx_1:A_1 . E_1\wedge \cdots Qx_n : A_n . e$ shall be considered as a context
$x_1 : A_1 ,\cdots ,x_n : A_n$ when written $\nu^{-1}(F)$.

\setcounter{tcase}{0}

\begin{tcase}
Lam-Any
\end{tcase}

\[
F[\lambda x : A . M \doteq N]
\UnifiesTo
F[\forall x : A . M \doteq \m{H}_{\nu^-1(F),x:A}(N , x)]
\]

Because application is normalizing, this can cover the case where both $N$ is also a $\lambda$ 
abstraction.

\begin{tcase}
Lam-Lam
\end{tcase}

\[
F[\lambda x : A . M \doteq \lambda x : A . N ]
\UnifiesTo
F[ \forall x : A . M \doteq N ]
\]

While this rule is not explicitly necessary as it is covered by the Lam-Any rule, 
when working
in a substitutive system with explicit names rather than DeBruijn indexes, 
this helps to reduce the number of substitutions from an original name. 

Lastly, these reductions make the 
assumption that no variable name is bound more than once.
This can seem restrictive during implementation, but it is possible to 
work past by alpha converting everywhere and annotating
new variables with their original names, and alpha converting
back to the original after unification. The other option is again
to use DeBruijn indexes.  DeBruijn indexes have their own drawbacks
here, as certain transformation such as ``Raising''
or the ``Gvar-Uvar'' rules involve insertion of multiple
variables into the context at an arbitrary point, 
which requiring the lifting of many variable names.  
It is possible to implement higher order unification
with DeBruijn indexes safetly and efficiently, 
but this is out of the scope of the thesis.

\begin{tcase}
Uvar-Uvar
\end{tcase}

\[
F[\forall y : A . G[y M_1 \cdots M_n \doteq y N_1 \cdots N_n  ]]
\UnifiesTo
F[\forall y : A . G[ M_1 \doteq \wedge N_1 \cdots \wedge M_1 \cdots N_n ]]
\]

\begin{tcase}
Identity
\end{tcase}

\[
F[M \doteq M] 
\UnifiesTo
F[\top]
\]

\begin{tcase}
Raising
\end{tcase}

\[
F[\forall y : A . \exists x : B . G]
\UnifiesTo
F[\exists x' : (\Pi y : A . B) . \forall y : A . [x' y / x : B]_{F, x : \Pi y : A . B , y : A} G]
\]

This rule is important, as correct or 
incorrect application of this rule can result in 
terminating or non terminating reductions sequences.

\begin{tcase}
Exists-And
\end{tcase}

\[
F[(\exists x : A . E_1) \wedge E_2]
\UnifiesTo
F[\exists x : A . E_1 \wedge E_2]
\]

\begin{tcase}
Forall-And
\end{tcase}

Unfortunately moving the universal quantifier to
capture a conjunction is not as simple, since
if done incorrectly, existential variables might be able
to be defined with respect to universal quantifiers that they
were not previously in the scope of.

\[
F[(\forall x : A . E_1) \wedge E_2]
\UnifiesTo
F[\forall x : A . E_1 \wedge E_2]
\]
provided no existential variables are declared in $E_2$.

While this restriction prevents most application of this rule, 
equations can still be flattened to the form
\[
Qx_1:A_1\cdots Qx_n : A_n . M_1 \doteq N_1 \wedge \cdots \wedge M_m \doteq N_n
\]

transforming $E_2$ first with the Raising rule untill 
an Exists-And transformation is possible, then repeating  
until $E_2$ no longer contains any existentially 
quantified variables.  This process is always terminating,
although potentially significantly slower.   


The following cases are based on unification 
of a formula of the form

\[
\Gamma[
\exists x : \Pi u_1 : A_1 \cdots \Pi u_n : A_n . A 
\\
F[\forall y_1 : A'_1 . G_1 
[ \cdots \forall y_p : A'_p . G_p 
[ x y_{\phi(1)} \cdots y_{\phi(n)} \doteq M ]
\cdots
]]]
\]

\begin{tcase}
Gvar-Uvar-Outside
\end{tcase}

$M$ has the form 
$y M_1 \cdots M_m$ some $y$ universally
quantified outside of $x$
and 
$y : \Pi v_1 : B_1 \cdots \Pi v_m : B_m . B$.  

Then we can \textit{imitate} $y$ with $x'$.
Let 
$L = \lambda u_1 : A_1 \cdots \lambda u_n : A_n .
 y (x_1 u_1 \cdots u_n) \cdots (x_m u_1 \cdots u_n)$.

Then we can transition to 

\[
\exists x_1 : \Pi u_1 : A_1 \cdots \Pi u_n : A_n . B_1 \cdots 
\exists x_m : \Pi u_1 : A_1 \cdots \Pi u_n : A_n . 
[x_{m-1}u_1\cdots u_n / v_{m-1} : B_{m-1}]
\cdots 
\]
\[
[x_1 u_1 \cdots u_n / v_1 : B_1]_{\Gamma, x_1:T_{x_1},\cdots, x_{m-1}:T_{x_n}} B_m .
[L / x : T_x ]_{\Gamma, x_1:T_{x_1},\cdots, x_m:T_{x_n}}F
\]

\begin{tcase}
Gvar-Uvar-Inside
\end{tcase}

If $M$ has the form 
$y_{\phi(i)} M_1 \cdots M_m$ for 
$1 \leq i \leq n$ then we can project $x$ to $y_{\phi(i)}$.

Here we can perform the same transition as in the Gvar-Uvar-Outside
case but let 
$L = \lambda u_1 : A_1 \cdots \lambda u_n : A_n .
 u_i (x_1 u_1 \cdots u_n) \cdots (x_m u_1 \cdots u_n)$.

\begin{tcase}
Gvar-Gvar-Same
\end{tcase}

$M$ has the form 
$x y_{\psi(1)} \cdots y_{\psi(n)}$.

In this case we pick the \textit{unique} permuation $\rho$ 
such that $\rho(k) = \psi(i)$ for all $i$ such that $\psi(i) = \phi(i)$.

Then letting
$L = \lambda u_1 : A_1 \cdots \lambda u_n : A_n .
 x' u_{\rho(1)} \cdots u_{\rho(n)} $, 
we can transition to
\[
\exists x' : \Pi u_1 : A_\rho(1) \cdots \Pi u_l : A_\rho(l) . A 
[L / x : T_x ]_{\Gamma, x' : T_x'} F
\] 


\begin{tcase}
Gvar-Gvar-Diff
\end{tcase}


$M$ has the form $z y_{\psi(1)} \cdots y_{\psi(m)}$
for some existentially quantified variable $z : \Pi v_1 B_1 \cdots \Pi v_m : B_m . B$ 
distinct from $x$ and partial permutation $\psi$. 

In this case, we can only transition if $z$ is existentially 
quantified consecutively outside of $x$.

In this case, we perform a dual imitation.

Let $\psi'$ and $\phi'$ be partial permutations 
such that for all $i$ and $j$ such that
$\psi(i) = \phi(j)$ then there is some $k$ such that
$\psi'(k) = i$ and $\psi'(k) = j$

Then let the $L,L'$ be as follows.

\[
L = \lambda u_1 : A_1 \cdots \lambda u_n : A_n . x' u_{\phi'(1)} \cdots u_{\phi'(l)}
\]

\[
L = \lambda v_1 : B_1 \cdots \lambda v_m : B_m . x' u_{\psi'(1)} \cdots u_{\psi'(l)}
\]

Then we can transition to

\[
\Gamma [ \exists x' : \Pi u_1 : A_{\phi'(1) } \cdots \Pi u_l : A_{\phi'(l)}
. [L'/z : T_z]_{\Gamma, x': T_{x'}}[L / x : T_x]_{\Gamma, x': T_{x'}} F
\]
 
While this case appears to be only applicable rarely,
it is the case that the Raising transition can be applied to allow this
rule to apply.  
Due to the restrictivity of this case, and the potential
for non termination with unrestricted application of the Raising rule,
it is the only case where the Raising rule is permitted.  


\subsection{Implementation}

Because typed substitution is necessary, we must now keep track of existential variable's
types.  This can significantly complicate the implementation of the unification algorithm
as the common technique of maintaining unbound existential variables with restrictions
can no longer be blindly used, as existential variables must be maintained in the 
formula.  If existential variables are maintained literally in the formula, 
the structure must provide the ability to add variables at both the top and bottom level.
This complication if implemented naively can lead to a significantly less efficient structure.
After experimentation, similar speeds have been observed when this structure is implemented
as a zipper \citep{huet1997functional}.  Unfortunately in this case, since variables are best 
implemented via DeBruijn indexes, variable reconstruction is no longer trivial. 
To reconstruct types, variable names might be included along side existential variable bindings.  
Again, this introduces significant complications.  In order to perform 
substitution for an existential variable, the context of existential variables would have to have
existential DeBruijn indexes continually swapped for their names.  
This is most readily implementable as having all existential variable instances 
also mention their names.

Another potential option is to maintain the type of the existential variable 
with each mention of the existential variable.  
While in this situation it is simplest to implement reconstruction, 
types might be enormous and it would be preferable to mention them only once.  

The last option is to perform unification with untyped substitution in certain cases.  
While there is no proof at the moment that unification on the pattern subset of 
the calculus of constructions with untyped substitution for only the existential substitutions
is total, it is not unbelievable.  Furthermore, omitting typed substitution does not alter
the correctness of the algorithm, only the potential totality.  

Ideally, knowledge that type checking terminated would be convincing enough
so it is not necessary to continue with the reconstruction.  However, reconstruction
is necessary for implementing the multi-pass proof search described previously.  
Furthermore, reconstruction is useful since the exposed typing rules are incoherent, 
meaning there could be multiple coercions to demonstrate a subtyping relation.  In these cases, it is desirable to see what was infered by type inference.
