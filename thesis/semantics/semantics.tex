In this chapter I lay out and justify an operational specification and semantics 
for Caledon.  I first discuss the history of the technique of unification.

Caledon's unification and proof search algorithm is based on the method designed for Twelf by Pfenning
\citep{pfenning1991logic}.  This algorithm does not terminate on all inputs and in this chapter 
I begin by charachterizing those inputs for which it is designed to termiante on.  In the next section, I describe
the techniques for evaluating terms to forms for unification.  I then describe the algorithm
used for higher order unification.  
The last sections pertain to proof search, which can be considered the traditional semantics of the language.
In the last section, I specify a technique for allowing the control of nondeterminism during proof search, 
allowing the programmer to choose between a complete breadth first search and an imperative depth first search.


