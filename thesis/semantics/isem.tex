\section{Semantics for $CICCI$}


\subsection{Substitution With Implicits}

The formulation of hereditary substitution in the presence of 
implicit arguments is not that unlike the presentation of
heredetary substitution without implicit arguments, 
but for additional checks required.
The main difficulty is the notion of an acceptable substitution. 
Because implicit bindings are not $\alpha$ convertable, 
certain substitutions are not permited.  
Because as many substitions should be permitted as possible, 
the situation becomes significantly more complex in the 
hereditary case, where substitutions might not carry types.  
The easiest way to define substitution in this case is with an ``illegal'' alpha substitution, 
which maps implicitly bound variables to fresh names, 
and produces a memory to map them back.

In this case, we can say that a substitution $[S/x] M$ is legal if 
$FV(S) \subseteq FV(\alpha_I^-1( [\alpha_I(S)/x] M) )$.

\begin{definition}
\textbf{(Implicit Typed Hereditary Substitution)}


\[
[S / x : A]^n_{\Gamma } (?\lambda y : B . N) := ?\lambda y:B . [S / x : A]^n_{\Gamma, y : B} N
\] 

\[
\eta^{-1}_{?\Pi x : A . B}(N) := ?\lambda x : A . N \; \{ x = \eta^{-1}_A(x) \}
\] since $N$ being typable by $?\Pi x $ means that $x$ can not appear free in $N$

\[
\m{H}_{\Gamma}(P \downarrow ?\Pi y : B_1 . B_2 , \{ v := N \} ) := P\; \{ v := N \} \downarrow [N/y : B_1]^n_{\Gamma}B_2
\]

\[
\m{H}_{\Gamma} ((?\lambda v : A_1 . N) \uparrow ?\Pi v : A_1 . A_2 , \{ v := P \}) 
:= [P/v]^n_{\Gamma \vdash v : A_1} N \uparrow A_2
\]

\[ 
\m{H}(?\lambda v : T . P \uparrow \_ , A) := ?\lambda v : T . \m{H}(P,A)
\]

\label{def:hered}
\end{definition}



\subsection{Unification With Implicits}

Now we can use the convenient fact that $\Gamma \vdash A  \leq B$ implies $\Gamma \vdash B \leq A$.