\section{Substitution}

The higher order unification algorithm described is only defined on the pattern form provided in the previous section.
As the pattern fragment is a restriction on $\beta\eta$ long-normal form, it is necessary to provide a normalizing
substitution that preserves $\beta\eta$ long-normal form.  The presentation here
revolves around heredetary substitution, a notion first described by Pfenning et al. \citep{pfenning1991logic} for
a calculus with only $\beta$ reduction.

Other presentations are possible, such as traditional substitution followed by Normalization by Evaluation 
for $\beta\eta$ conversions\citep{abel2010towards}. 
While this is a proven total method in a typed setting, it's mechanics
are complex and not particularly enlightening.  
Keller et al. \citep{keller2010normalization}
extended Heredetary substitution to $\eta$ expansion, but only in the simply typed case. 
The presentation here extends this version into $CC$.

\subsection{Untyped Substitution}

\begin{definition}
\textbf{(Hereditary Substitution)} 

\[ \begin{array}[3]{llr}
[ S / x ] x := S
&
[S / x] y := y
&
[S / x] P\;T := \m{H}([S/x] P, [S/x]T)
\end{array} \]

\[
[S / x] \lambda v : T . P := \lambda v' : [S/x]T . [S/x][v'/v]P
\;
\text{  where $v'$ is new.}
\]

\[ 
\m{H}(\lambda v : T . P , A) := [A/v] P
\]

\[ \begin{array}[2]{lr}
\m{H}(P A_1 , A_2) := P\;A_1\;A_2
&
\m{H}(V , A) := V\; A
\end{array} \]

\label{def:shered}
\end{definition}

It is important to note the alpha conversion in the $\lambda$ case, as alpha conversion will be lost on 
some terms when implicits are added.

A hereditary substitution is not necessarily terminating as is shown by
substitution replicating the $\omega$-combinator.

\[
[(\lambda x : T . x\; x) / x ] ( x \; (\lambda x : T. x\; x) )
\]

This is not defined, as it expands to the well known $\m{H}(\lambda x . x \; x , \lambda x . x \; x)$.

If the pattern and substitution are well typed terms in the Calculus of Constructions, by 
strong normalization, this version of hereditary substitution is total.


\begin{theorem} Substitution Theorem:

If $\Gamma,x : T \vdash A : T' : \m{prop}$  and $\Gamma \vdash S : T : \m{prop}$ then
$ \Gamma \vdash [S/x]_o A : [S/x]_o T' : \m{prop}$
\end{theorem}

By consistency, $[S/x]_o A $ can be normalized to strong head normal form.  Thus, the 
hereditary substitution $[S/x] A$ is defined.



\subsection{Typed Substitution}

Performing substitutions that maintain long $\beta\eta$-normal 
form is important to ensuring decidability of unification.  
Unfortunately this is not possible without some type information, 
as arbitrary $\eta$ expansion has no stop condition.
Keller \citep{keller2010normalization} gave a heredetary substitution algorithm
that results in canonical forms for simply typed lambda calculus. 
\ref{def:tyhered} solves this by performing a typed substitution under a context, 
generating type information.

\begin{theorem}
\textbf{(Reduction Decomposition)}
If $\Gamma \vdash A \Rightarrow_{\beta\eta*} B$ then there exists some $C$ such that
$A \Rightarrow_{\beta*} C$ and $\Gamma \vdash C \Rightarrow_{\eta*} B$

\label{thm:betaeta}
\end{theorem}

The property in \ref{thm:betaeta} stating that any reduction can be 
shown equivalent to first a series of $\beta$ reductions and then a series of
$\eta$ expansions forms the basis of the following algorithm.

For this section it is convenient to include $\Pi$ in the spine form: 
\[
N ::= P 
\orr \lambda V : N . N 
\orr \Pi V : N . N 
\]


\begin{definition}
\textbf{(Typed Hereditary Substitution)}

\[
[S / x : A]^n_{\Gamma } P := \m{E}([S / x : A]^p_{\Gamma } P)
\]

\[
[S / x : A]^n_{\Gamma } (\lambda y : B . N) := \lambda y:B . [S / x : A]^n_{\Gamma, y : B} N
\]

\[
\m{E}^M(M \uparrow P) := M
\]

\[
\m{E}(N \downarrow A) := \eta^{-1}_A (N)
\]

\[
\eta^{-1}_{\Pi x : A . B}(N) := \lambda z : A . N \; \eta^{-1}_A(z)
\] where $z$ is fresh

\[
\eta^{-1}_{P}(N) := N
\]

\[ 
[ N / x : A]^p_{\Gamma} x := N \uparrow A
\]

\[
[S / x : A]^p_{\Gamma} y := y \downarrow \Gamma(x)
\] 

\[
[S / x : A]^p_{\Gamma } P\;N := 
\m{H}_{\Gamma} ( [S/x : A]^p_{\Gamma} P , [S/x : A]^n_{\Gamma} N) 
\]

\[
\m{H}_{\Gamma} ((\lambda v : A_1 . N) \uparrow \Pi v : A_1 . A_2 , P) 
:= [P/v]^n_{\Gamma \vdash v : A_1} N \uparrow A_2
\]

\[
\m{H}_{\Gamma}(P \downarrow \Pi y : B_1 . B_2 , N) := P\; N \downarrow [N/y : B_1]^n_{\Gamma}B_2
\]

\label{def:tyhered}
\end{definition} 

Note, that while we could omit the distinction between $\uparrow$ and $\downarrow$ from
these rules, it would in practice result in excess computation.

Also, note that in the hereditary part of the recurrence, 
it is possible to omit cases for improperly formated terms.  $A \downarrow T$ has the invarient that
$A$ must be already in a cononical form within the context it is used since it existed previously in the
equation.  Similarly, $A \uparrow T$ has the invarient that $A$ must be locally in a canonical form.
This permits significant amounts of reduction of steps.

The usefull property of this hereditary substitution is spelled in \ref{thm:hed-sound}

\begin{theorem}
\textbf{(Soundness of Heredetary Substitution)}
If  $\Gamma \vdash S : A$ 
and $\Gamma \vdash x : A$ 
and $\Gamma \vdash S \; \m{pattern}$ 
and $\Gamma \vdash N \; \m{pattern}$ 
and $x$ is free for $S$ in $N$
and $[S / x : A]^n N$ is defined, 
then
and $\Gamma \vdash ([S / x : A]^n N) \; \m{pattern}$ 
\label{thm:hed-sound}
\end{theorem}

In general, it is provable that for any PTS that is strongly normalizing for $\beta$ reduction and 
$\eta$ expansion, this algorithm will terminate and substitution will be defined.  
This is a direct consequence of \ref{thm:betaeta}.
