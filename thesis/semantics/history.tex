\section{History}

Huet \citep{Huet75} gave the first semi-decision algorithm for unification of 
terms in the lambda calculus.  
Later, it was proven by Miller \citep{miller1986higher} 
that for terms in the pattern fragment of the lambda calculus, unification was decidable.  
Pfenning \citep{pfenning1988partial} \citep{pfenning1988higher} demonstrated unification for the typed lambda calculus and considered solving the dynamic pattern fragment where non pattern equations
are postponed.
Elliott\citep{elliott1989higher} gave a more efficient algorithm for unification in the context 
of dependent types, and later Pfenning \citep{pfenning1991unification}
did the same thing for unification in the Calculus of Constructions, although without a mixed prefix.  
The most succinct presentation comes from the 1991 paper describing the workings of Twelf 
\citep{pfenning1991logic}.  While the unification algorithm implemented in the interpreter for Caledon is 
an extension of that presented in Pfenning's 1991 paper on unification for 
the Calculus of Constructions\citep{pfenning1991unification}, 
I will briefly cover the main ideas from the presentation of the paper describing the workings of Twelf, 
and extend them later.


