\section{Proof Search}

In a traditional logic programming language, the order of declaration of quantified arguments is irrelevant, 
and the context can be considered an unordered set (even though for implementation reasons it is not). 
In a dependently typed logic programming language where types direct proof search, types must be maintained in the context
and the context thus must maintain order.   Since search dynamically poses unification problems, which may not be 
entirely solvable until later in the search, unification and proof search are naturally mutually recursive procedures.
As it is important to maintain the mixed quantifier prefix thought proof search, it is desirable to provide a version 
of the algorithm where unification and proof search are not distinct procedures. 
\citep{pfenning1991logic} gave a succinct formulation where inhabitance and immediate implication were represented
directly in the unification calculus.  

\subsection{Proof Sharing}

\begin{definition}
Unification Calculus with Search

\[
U ::= U \wedge U 
 \orr \forall V : T . U
 \orr \exists V : T . U 
 \orr U \doteq U
 \orr \top
  \orr T \in T 
  \orr T \in T >> T \in T
\]

\end{definition}

The following new transformations are added to represent proof search:

\[
G_\Pi : M \in \Pi x : A . B   \rightarrow \forall x : A . \exists y : B . y \doteq M x \wedge y \in B
\]

\[
G^1_\m{Atom} : \forall x : A . F[M\in C]  \rightarrow \forall x : A . F[x \in A >> M \in C]
\]

\[
G^2_\m{Atom} : F[M\in C]  \rightarrow \forall x : A . F[c_0 \in A >> M \in C]
\]  where $c_0 : A$ is a constant

\[
D_\Pi : N\in \Pi x : A . B >> M \in C \rightarrow \exists x : A ( N x \in B >> M \in C) \wedge x \in A
\]

\[
D_\m{Atom} : N\in a N_1 \cdots N_n >> M \in a M_1 \cdots M_n \rightarrow N_1 \doteq M_1 \wedge \cdots \wedge N_n \doteq M_n \wedge N \doteq M
\]

\subsection{Proof Sharing}

In a pure setting, significant improvements to the efficiency of the system can be made by 
extending the quantifiers of the unification calculus to include forced inhabitant existential quantification.

\[
U ::= U \wedge U 
 \orr \forall V : T . U
 \orr \exists V : T . U 
 \orr \exists_f V : T . U 
 \orr U \doteq U
 \orr \top
 \orr T \in T >> T \in T
\]

\[
G^1_\m{Atom} : \forall x : A . F[\exists_f V : T . \top]  \rightarrow \forall x : A . F[x \in A >> M \in C]
\]

\[
G^2_\m{Atom} : \exists_f x : A . F[\exists_f V : T . \top]  \rightarrow \forall x : A . F[x \in A >> M \in C]
\]

\[
D_\Pi : N\in \Pi x : A . B >> M \in C \rightarrow \exists_f x : A ( N x \in B >> M \in C)
\]

In this situation, it is permitted to use the results of future searches for the solution of the current search.
While this sharing is optimal from an operational standpoint, it can make reasoning about the behavior 
of impure logic programs very difficult.  Given that Caledon is an impure programming language, reasoning about program
behavior comes before optimizing proof search.  It is the subject of future research to determine proof sharing teqchniques
that do not interfere with I/O. 
