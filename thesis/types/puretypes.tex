\section{Pure Type Systems}

The type system for caledon is a pure type system \citep{mckinna1993pure} 
extended with explicit recursive types and implicit types.  In this section,
I discuss what a pure type system is and what its properties are.

Pure type systems are generalizations of the lambda cube
\citep{barendregt1991introduction} which allow for arbitrary 
relationships between terms and types.
With proper selection of constants, sorts, axioms, and relations,
pure type systems can embed the ``Calculus of Constructions'' \citep{coquand1986calculus}
and many other type systems one might want to construct.

As generalizations, these systems are important, as it has been proven by Jutting \citep{jutting1993typing} that 
type checking for normalizing pure type systems with finite axiom sets are decidable.  Thus, by showing how a system is a pure type system and is normalizing, 
you get decidability of type checking nearly for free.

It has also been shown that these systems have utility.  
Roorda \citep{roorda2001pure} gave an implementation of a functional programming language with 
pure type system and demonstrated its utility.

A pure type system is a set $S$ of sorts, 
$A\subseteq S \times S$ of axioms, and a relation 
$R \subseteq S \times S \times S$ along with the following grammar and inference rules:

\begin{definition}
\textbf{(PTS Syntax)}
\[ 
E ::=  V 
 \orr S 
 \orr E\;E 
 \orr \lambda V : E . E 
 \orr \Pi V : E . E 
\]

\label{pt:syntax}
\end{definition}

\begin{definition}

\textbf{(PTS Typing Rules)}

\[ \begin{array}{lr}
\infer[\m{WF-E}]
{
\cdot \vdash
}
{}
&
\infer[\m{WF-S}]
{
\Gamma, x : T \vdash
}
{
\Gamma \vdash T : s
&
x \notin DV(\Gamma)
}
\end{array} \]

\[
\infer[\m{axioms}]
{
\Gamma \vdash c : s
}
{
\Gamma \vdash
&
(c,s) \in A
}
\]

\[
\infer[\m{start}]
{
\Gamma,x:A \vdash x :A
}
{
\Gamma \vdash A:s
&
s \in S
}
\]

\[
\infer[\m{weakening}]
{
\Gamma,x:C \vdash A:B
}
{
\Gamma \vdash A:B
&
\Gamma \vdash C:s
&
s \in S
}
\]


\[
\infer[\m{product}]
{
\Gamma \vdash (\Pi x : A . B) : s_3
}
{
\Gamma \vdash A : s_1
&
\Gamma,x:A \vdash B : s_2
&
(s_1,s_2,s_3) \in R
}
\]

\[
\infer[\m{application}]
{
\Gamma \vdash F V : [V/x] B
}
{
\Gamma \vdash F : (\Pi x : A . B)
&
\Gamma \vdash V : A
&
\text{V is free for x in B}
}
\]

\[
\infer[\m{abstraction}]
{
\Gamma \vdash (\lambda x : A . F) : (\Pi x : A . B)
}
{
\Gamma , x : A\vdash F : B
&
\Gamma \vdash (\Pi x : A . B) : s
&
s \in S
}
\]

\[
\infer[\m{conversion}]
{
\Gamma \vdash A : B'
}
{
\Gamma \vdash A : B
&
\Gamma \vdash B \equiv_{\beta\eta\nu*} B'
&
\Gamma \vdash B' : s
&
s \in S
}
\]

\label{pt:typing}
\end{definition}


As Barendgregt\citep{barendregt1991introduction} points out, the common type theories can be recast as pure type systems
by choice of axioms.  
In the simplest example, the only axioms chosen are $(*,\Box)$ along with 
the single relationship $(*,*,*)$. This system describes the simply typed lambda calculus, 
where only terms can depend on terms.  We say that $A \rightarrow B \equiv \Pi x : A . B$ iff $ x \notin FV(B)$.

\begin{theorem} 
Subject Reduction: If $\Gamma \vdash A : T$ and $A \Rightarrow_\beta B$ then $\Gamma \vdash B : T$
\end{theorem}

Geuvers and Nederhof \citep{geuvers1991modular} proved subject reduction for any calculus on the $\lambda$ cube.
This property can be proved syntactically by induction on the structure of the typing derivation 
and there exist Twelf and Agda verified proofs of this property.  
Note that this is a useful property 
to maintain, even in the face of inconsistency of a system, because at the very least, the 
property allows for a consistent understanding of typing terms.

\begin{theorem}
Uniqueness of Types: If $\Gamma \vdash A : T$ and $\Gamma \vdash A : T'$ then $T \equiv_\beta T'$
\end{theorem}

The uniqueness of types with respect to $\beta$ reduction has also been shown for any system on the $\lambda$ cube.  
This last property is important to showing the decidability of type inference in 
the caledon language without implicits.



\begin{lemma}
Strengthening
\[
\infer-[\m{strength}]
{
\Gamma \vdash M : U
}
{
\Gamma , x : T \vdash M : U
&
x \notin FV(M) \cup FV(U)
}
\]
\end{lemma}

As it turns out, the strengthening lemma has important implications to the generation of bindings 
during proof search.
