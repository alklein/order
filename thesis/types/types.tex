In this section I introduce the specifics of the \textit{Caledon Implicit Calculus of Constructions} ($CICC$).
The internal Caledon type system is an extension of the well known Calculus of Constructions with the 
addition of implicit bindings, and explicit type constraints for implicit instantiation.  
While the inspiration for this comes from the theorem prover Agda, it appears as though no formal treatment
has been provided. This section provides a background on pure type systems, the history of the calculus of constructions
and introduces a formal defintion and treatment of $CICC$ with $\eta$ conversions for the purpose of type checking
and proof search.

The type system of Caledon is designed after two different formalisms for working with implicit arguments:  
the Bicolored Calculus of Constructions ($CC^{Bi}$) \citep{luther2001more}, 
the Implicit Calculus of Constructions ($ICC$) \citep{miquel2001implicit}.

It is comprised of two parts: The Caledon Implicit Calculus of Constructions ($CICC$), 
and the Implicit Caledon Implicit Calculus of Constructions ($CICC^-$). 
$CICC$ is a custom modification of $CC^{Bi}$ with Church style binders and explicit constraints 
where alpha conversion is not available for certain binder like constructs.
$CICC^-$ Is intended to be a combination of the first two, or a partial 
erasure system for $CICC$. 
