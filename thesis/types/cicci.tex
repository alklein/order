\section{Inference for CICC}

In order to make use of the implicit system of $CICC$, an inference
relation must be provided.  
This is accomplished by extending the typing rules and providing
a mapping from the extended type derivation and term to 
an original type derivation and term. 

We only have one syntactic difference in this calculus:  $E\; \{ V : A = E \}$ 
is now simply $ E \; \{ V  = E \}$.  
We might also include Curry style binders in this presentation, but they shed little 
extra light on the workings of type inference.

\begin{definition}
\textbf{($CICC^-$ Extended Typing Rules)}

%% inst/f %%
%%%%%%%%%%%%
\[
\infer[\m{inst/f}]
{
\Gamma \vdash_{ci^-} M : [N/x]U 
}
{
\Gamma \vdash_{ci^-} M : ?\Pi x :T . U
&
\Gamma \vdash_{ci^-} N : T
&
x \notin DV(\Gamma)
}
\]

%% abs2 %%
%%%%%%%%%%
\[
\infer[\m{abs}_2]
{
\Gamma\vdash_{ci^{-}} M : ?\Pi x : T . U
}
{
\Gamma, x : T \vdash_{ci^{-}} M : U
&
\Gamma \vdash_{ci^{-}} (?\Pi x : T . U) : K
&
x \notin FV(M) \cup DV(\Gamma)
}
\]

%% strength %%
%%%%%%%%%%%%%%
\[
\infer[\m{strength}]
{
\Gamma\vdash_{ci^{-}} M : U
}
{
\Gamma, x : T  \vdash_{ci^{-}} M : U
&
x \notin FV(M) \cup FV(U)  \cup DV(\Gamma)
}
\]

%% inst/b %%
%%%%%%%%%%%%
\[
\infer[\m{inst/b}]
{
\Gamma \vdash_{ci^-} M \{ x = N \} : [N/x]U 
}
{
\Gamma \vdash_{ci^-} M : ?\Pi x : T . U
&
\Gamma \vdash_{ci^-} N : T
& 
x \notin GN(M)
&
x \notin BN(U)
}
\]

\end{definition}

In $CICC$, as in $CC$, the strengthening rule is admissible,
while in $CICC^{-}$, it is not.  

While we no longer care about the semantics of this language since we will be
elaborating to the sublanguage $CICC$ before evaluating, we do not need to 
semantic related properties.  
However, it is still important to note that substitution holds.

\begin{theorem}
\textbf{(Substitution)}
\[
\infer-[\m{subst}]{ 
\Gamma \vdash_{ci^-} [N / x]M : [N/x]T_2
}{
\Gamma, x : T_1 \vdash_{ci^-} M : T_2
&
\Gamma \vdash_{ci^-} N : T_1
}
\]
\label{ci:sub}
\end{theorem}

Unfortunately, the projection function now requires more information than is available syntactically, 
and thus must be given on the typing derivation.

\begin{definition}
\textbf{ (Projection from $CICC^{-}$ to $CICC$) }

\newcommand{\CICCmproj}[1]{ \left\llbracket #1 \right\rrbracket_{ci^{-}}}

\[
\CICCmproj{
\infer[\m{wf/e}]
{
\cdot \vdash_{ci^-} 
}{}
}^{c}
:= \cdot
\]

\[
\CICCmproj{
\infer[\m{wf/s}]
{
\Gamma, x : T \vdash_{ci^-} 
}
{
\overset{\mathcal{D}}{ 
\Gamma \vdash_{ci^-} x : T 
}
&
\cdots
}
}^{c}
:= \CICCmproj{\Gamma \vdash_{ci^-}}^c, \CICCmproj{\mathcal{D}} 
\]

\[
\CICCmproj{
\infer[\m{start}]
{
\Gamma,x:A \vdash_{ci^-} x :A
}
{
\cdots
}
}
:= x
\]


\[
\CICCmproj{
\infer[\m{axioms}]
{
\Gamma,x:A \vdash_{ci^-} c : s
}
{
\cdots
}
}
:= c
\]

%% prod %%
%%%%%%%%%%
\[
\CICCmproj{
\infer[\m{prod}]{ \Gamma \vdash_{ci^-} (\Pi x : T . U) : s 
}{ 
\overset{\mathcal{D}_1}{ 
\Gamma \vdash T : s_1
}
&
\overset{\mathcal{D}_2}{ 
\Gamma,x:T \vdash U : s_2
}
&
\cdots
}
}
:=
\Pi x : \CICCmproj{ \mathcal{D}_1 }  . \CICCmproj{ \mathcal{D}_2 }
\]

%% prod* %%
%%%%%%%%$%%
\[
\CICCmproj{
\infer[\m{prod}*]{ \Gamma \vdash_{ci^-} (?\Pi x : T . U) : s 
}{ 
\overset{\mathcal{D}_1}{ 
\Gamma \vdash T : s_1
}
&
\overset{\mathcal{D}_2}{ 
\Gamma,x:T \vdash U : s_2
}
&
\cdots
}
}
:=
?\Pi x : \CICCmproj{ \mathcal{D}_1 }  . \CICCmproj{ \mathcal{D}_2 }
\]

%% gen %%
%%%%%%%%%
\[
\CICCmproj{
\infer[\m{gen}]
{
\Gamma \vdash_{ci^-} \lambda x : T . M : (\Pi x : T . U)
}
{
\overset{\mathcal{D}_1}{
\Gamma , x : T \vdash_{ci^-} M : U 
}
&
\infer[\m{prod}]{ \Gamma \vdash_{ci^-} (\Pi x : T . U) : s 
}{ 
\overset{\mathcal{D}_2}{ 
\Gamma \vdash T : s_1
}
&
\overset{\mathcal{D}_3}{ 
\Gamma,x:T \vdash U : s_2
}
&
\cdots
}
&
\cdots
}
}
:=
\lambda x : \CICCmproj{ \mathcal{D}_2 }  . \CICCmproj{ \mathcal{D}_1 }
\]

%% gen* %%
%%%%%%%%%%
\[
\CICCmproj{
\infer[\m{gen}*]
{
\Gamma \vdash_{ci^-} ?\lambda x : T . M : (?\Pi x : T . U)
}
{
\overset{\mathcal{D}_1}{
\Gamma , x : T \vdash_{ci^-} M : U 
}
&
\infer[\m{prod}*]{ \Gamma \vdash_{ci^-} (?\Pi x : T . U) : s 
}{ 
\overset{\mathcal{D}_2}{ 
\Gamma \vdash T : s_1
}
&
\overset{\mathcal{D}_3}{ 
\Gamma,x:T \vdash U : s_2
}
&
\cdots
}
&
\cdots
}
}
:=
?\lambda x : \CICCmproj{ \mathcal{D}_2 }  . \CICCmproj{ \mathcal{D}_1 }
\]

%% app %%
%%%%%%%%%
\[
\CICCmproj{ 
\infer[\m{app}]
{
\Gamma \vdash_{ci^-} M N : U [N/x]
}
{
\overset{\mathcal{D}_1}{ \Gamma \vdash_{ci^-} M : \Pi x : T . U }
&
\overset{\mathcal{D}_2}{ \Gamma \vdash_{ci^-} N : T }
}
}
:=
\CICCmproj{ \mathcal{D}_1 } \; \CICCmproj{\mathcal{D}_2}
\]

%% inst/b %%
%%%%%%%%%%%%
\[
\CICCmproj{ 
\infer[\m{inst/b}]
{
\Gamma \vdash_{ci^-} M \{ x = N \} : U [N/x]
}
{
\overset{\mathcal{D}_1}{ \Gamma \vdash_{ci^-} M : ?\Pi x :T . U }
&
\overset{\mathcal{D}_2}{ \Gamma \vdash_{ci^-} N : T }
& 
\cdots
}
}
:=
\CICCmproj{\mathcal{D}_1} \; \{ x : \CICCmproj{\Gamma \vdash T : \m{kind}} = \CICCmproj{\mathcal{D}_2} \}
\]

%% inst/b %%
%%%%%%%%%%%%
\[
\CICCmproj{ 
\infer[\m{inst/f}]
{
\Gamma \vdash_{ci^-} M : U [N/x]
}
{
\overset{\mathcal{D}_1}{ \Gamma \vdash_{ci^-} M : ?\Pi x : T . U }
&
\overset{\mathcal{D}_2}{ \Gamma \vdash_{ci^-} N : T }
&
\cdots
}
}
:=
\CICCmproj{\mathcal{D}_1} \; \{ x = \CICCmproj{\mathcal{D}_2} \}
\]

%% strength %%
%%%%%%%%%%%%%%
\[
\CICCmproj{ 
\infer[\m{strength}]
{
\Gamma \vdash_{ci^-} M : U
}
{
\overset{\mathcal{D}}{ \Gamma, x : T \vdash_{ci^-}M : U }
&
\cdots
}
}
:=
\CICCmproj{\mathcal{D}}
\]


%% strength %%
%%%%%%%%%%%%%%
\[
\CICCmproj{ 
\infer[\m{abs}_2]
{
\Gamma \vdash_{ci^-} M : ?\Pi x : T . U
}
{
\overset{\mathcal{D}}{ \Gamma \vdash_{ci^-}M : U }
&
\cdots
}
}
:=
?\lambda x : T . \CICCmproj{\mathcal{D}}
\]
\label{cicc-:proj}
\end{definition}


\begin{theorem}

\textbf{(Soundness of extraction)}  

\begin{alignat}{4}
\Gamma &\vdash_{ci^{-}}&  & \implies & \CICCproj{\Gamma \vdash_{ci^-}}^c & \vdash_{ci^-} &
\\
\Gamma &\vdash_{ci^{-}}& A : T & \implies & \CICCproj{\Gamma \vdash_{ci^-}}^c & \vdash_{ci^-} & \CICCproj{ \Gamma \vdash_{ci^-} A : T }
\end{alignat}

\label{cicc-:sound}
\end{theorem}



%%%%%%%%%%%%%%%%%%%%%%%%%%%%%%%%%%%%%%%%%%%%%%%%%%%%%%%%%%%%%
%%% Subtyping %%%%%%%%%%%%%%%%%%%%%%%%%%%%%%%%%%%%%%%%%%%%%%%
%%%%%%%%%%%%%%%%%%%%%%%%%%%%%%%%%%%%%%%%%%%%%%%%%%%%%%%%%%%%%
\subsection{Subtyping}

Similar to $ICC$, these rules result in a subtyping relation, which will be of
importance during type inference and elaboration.

\begin{definition}
Subtyping relation:
$\Gamma \vdash_{ci^-} T \leq T' \;\; \equiv \;\; \Gamma, x : T \vdash_{ci^-} x : T'$  where $x$ is new.
\end{definition}

\begin{lemma}
Subtyping is a preordering:
\[
\begin{array}{lr}
\infer-[\m{refl}]{ 
\Gamma \vdash_{ci^-} T \leq T
}{
\Gamma \vdash_{ci^-} T : s
}
&
\infer-[\m{trans}]{ 
\Gamma \vdash_{ci^-} T_1 \leq T_3
}{
\Gamma \vdash_{ci^-} T_1 \leq T_2
&
\Gamma \vdash_{ci^-} T_2 \leq T_3
}
\end{array}
\]

\[
\infer-[\m{sub}]{ 
\Gamma \vdash_{ci^-} M : T'
}{
\Gamma \vdash_{ci^-} M \leq T
&
\Gamma \vdash_{ci^-} T \leq T'
}
\]
\end{lemma}

This theorem is an application of the substitution lemma.

\begin{lemma}
Domains of products are contravariant and codomains are covarient:

\[
\begin{array}{lr}
\infer-[]{ 
\Gamma \vdash_{ci^-} \Pi x : T . U \leq \Pi x : T' . U'
}{
\Gamma \vdash_{ci^-} T' \leq T 
&
\Gamma,x : T' \vdash_{ci^-} U \leq U'
}
&
\infer-[]{ 
\Gamma \vdash_{ci^-} \forall x : T . U \leq \forall x : T' . U'
}{
\Gamma \vdash_{ci^-} T' \leq T 
&
\Gamma,x : T' \vdash_{ci^-} U \leq U'
}
\end{array}
\]
\end{lemma}

Unlike traditional subtyping relations where an explicit subtyping rule must be included in the type system,
this system's subtyping relation is much easier to manage during unification, as it is simply
a macro for a provability relation.  

This allows one to implement higher order unification almost exactly
as is usual without to much modification as would be the case in a lattice unification algorithm.  
Instead, the modification is made to the search procedure, and subtyping constraints 
are realized as search terms.  

However, with the addition of the strengthening rule, 
this kind of modification not entirely necessary, 
as it is provable that this subtyping relation is symetric \ref{ci:sym}, 
and thus an entirely symetric unification algorithm can be presented.


\begin{theorem}
\textbf{(Symmetry)}
$\Gamma \vdash_{ci^-} A \leq B $ implies 
$\Gamma \vdash_{ci^-} B \leq A $.
\label{ci:sym}
\end{theorem}


This theorem isn't exactly obvious upon first glance, 
so I will provide intuitive justification first.

In $CICC$, by uniqueness of types, 
$\Gamma \vdash x : A$ and 
$\Gamma \vdash x : B$ implies
$A \equiv_{\eta\beta\alpha*} B$.  
In $CICC^{-}$ however, there is no such uniqueness of 
types properties.  
Rather, the $\m{inst/f}$ and $\m{abs}_2$ 
rules permit you to repsectively,
add an initialize an implicit argument, 
abstract implicitely upon an unused argument. 

Thus if $\Gamma , x : ?\Pi z : T . A \vdash_{ci^-} x : A$
by implicit instantiation of the argument $z:T$,
we might also
derive
$\Gamma , x : A \vdash_{ci^-} x : ?\Pi z : T . A$
given that $z \notin FV(x)$ and that 
$\Gamma , x : ?\Pi z : T . A \vdash_{ci^-} x : A$ 
implies $ \Gamma , x : ?\Pi z : T . A \vdash_{ci^-}$ 
which implies $ \Gamma\vdash_{ci^-} x : (?\Pi z : T . A) : K$.
\begin{comment}
-------------------------- Id/x
G, z : T , x : A |- x : A        G |- ?[z:T]A : K   z notin FV(x)
----------------------------------------------------------------- abs_2
G , x : A |- x : ?[z : T ] A


G |- M : ?[x:T]U   G |- N : T
------------------------------ inst/f
 G |- M : [N/x] U



\end{comment}

\begin{comment}

CASE ABS2: 

G,x:A, z : T |- x : B      G |- ?[z:T]B : K      z notin FV(x)\cup FV(G,x:A) 
--------------------------------------------------------------------abs_2
            G, x : A |- x : ?[z:T] B
 
                                     G |- ?[z:T] B : K                 
                                   -------------------- weak         ----------
                                   G,z:T |- ?[z:T]B : K              G |- T : K
                            ------------------------------- id    -------------- id
then                         G,z:T, x : ?[z:T]B |- x : ?[z:T]B    G,z:T |- z : T
     I.H.                   -----------------------------------------------------  inst/f
G,z:T, x : B |- x : A              G, z : T , x : ?[z:T] B |- x : B
--------------------------------------------------------------------  subst
  G, z:T, x : ?[z:T] B |- x : A    since x notin FV(A)
  ----------------------------------------------------- strength
            G, x : ?[z:T] B |- x : A            

\end{comment}


\begin{comment}

CASE STRENGTH:

G, x : A, z : T |- x : B   z \notin FV(x)\cup FV(B) \cup DV(G)
------------------------------------------
    G, x : A |- x : B

then z is not in FV(A)

     I.H.
--------------------
G,z:T, x : B|- x : A 
-------------------- strength
G, x : B |- x : A

\end{comment}


\begin{proof}
\begin{comment}

G , x : A |- x : ?[z : T] B    G , x : A |- N : T    z \notin FV(\Gamma, x:A)
----------------------------------------------------------------------------- inst/f
G , x : A |- x : [N/z]B

since G , x : A |- x : ?[z : T]B   has x standing alone and minimal, the only two proof options 
are identity, conversion, and abs_2.  

then z \notin FV(N) \cup FV(T), since z \notin G

Assume we've gotten all the conversion and now have G , x : A |- x : ?[z' : T']B'  with ?[z' : T']B' == ?[z : T]B.

then the only option is identity and abs_2.

* Assume that it was identity.  

Then A == ?[z : T] B. 

so [q/x] A == ?[z : [q/x]T] [q/x]B

                     G , x : A |- N : T
                       ---------- str          
                       G |- N : T               G,z : T |- B : K_2
                     ------------------------------------------------- subst
G, x : A |- N : T                G|- [N/z] B : K                                  G,x:A |- T : K_1     G,x : A,z:T |- B : K_2
----------------- str     ----------------------------------------- Id           ---------------- st  ----------------------- st
    G |- N : T             G,q:A,x : [[q/x]N/z] B |- x : [[q/x]N / z] B         G |- T : K_1         G,z:T |- B : K_2
------------------------------------------------------------------  rev/subst  --------------------------------------- pi
G,q:A,z:T, x : [N/z] B |- x : B                                                 G|- ?[z:T]B : K_2               z notin FV(x) u FV(G,q:A,,x:[N/z]B)
-------------------------------------------------------------------------------------------------------------------------------------------------- abs_2
  G, x : [N/z]B |- x : ?[z : T]B == A

Thus, G, x :[N/z] B |- x : A


* Assume that it was abs_2.

 G, z:T , x : A |- x : B        G |- ?[z:T]B : K      z notin FV(x) \cup FV(\Gamma)
 ---------------------------------------------------------------------------------- abs_2
  G , x : A |- x : ?[z : T] B
                                  G,x:A |- N : T      
      I.H.                        --------------- strengthening since $x$ is new
 G,z : T, x : B |- x : A           G |- N : T
 --------------------------------------------------------- subst
         G, x : [N/z]B |-  x : [N/z]A                      since $z \notin FV(A)$


\end{comment}

We begin with the case 
\[
\infer[\m{inst/f}]
{
\Gamma,x : A \vdash_{ci^-} x : [N/z]B 
}
{
\Gamma, x : A \vdash_{ci^-} x : ?\Pi z :T . B
&
\Gamma x : A \vdash_{ci^-} N : T
}
\]



\end{proof}

