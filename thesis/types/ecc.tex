\section{Extended Calculus of Constructions}

The Extended Calculus of Constructions ($ECC$) \citep{luo1989ecc}, 
is the standard calculus of constructions extended with ``strong sums'', 
also known as dependent pairs.  I describe the calculus here because unlike $ICC^+$, 
it is not possible to translate $CICC$ to $ECC$, but it is also possible to translate back
from $ECC$ to $CICC$, thus providing $\beta$ strong normalization for $CICC$.  
Dealing with strong sums in higher order unification has also been covered by Elliot 
\citep{elliott1989higher}.

\begin{definition}
\textbf{(ECC syntax)}

\[
E ::= 
V 
\orr K
\orr E\;E 
\orr \lambda V : E . E 
\orr \Pi V : E . E 
\orr \Sigma V : E . E 
\orr \pi_i M
\orr \m{pair}(M_1, M_2)
\]
\label{ecc:syntax}
\end{definition}


\begin{definition}
\textbf{ (ECC conversion rules) } 

\[
(\lambda x : A . M ) N \equiv_{\beta} [N/ x] M
\]

\[
\pi_i( \m{pair}_{\Sigma x : A . B}(M_1, M_2) ) \equiv_{\sigma} M_i
\]

\[
\pi_i( \m{pair}_{\Sigma x : A . B}(M_1, M_2) ) \equiv_{\sigma} M_i
\]

\label{ecc:conversion}
\end{definition}

Take note that these rules do not include $\eta$ conversion.  While the effects of $\eta$ conversion
are believed to be understood, it appears as though they have not been fully investigated, and 
thus the properties of the system under the influence of $\eta$ conversion can not be definitivly 
stated.

The typing rules for $ECC$ are a direct extension of those from $CC$.  Unfortunately, the system
is no longer a pure type system, and thus can not be given by an extention of the sort rules.

\begin{definition}
\textbf{ (ECC typing rules) } All those from standard $CC_\beta$ and these:


\[
\infer[\Sigma\m{/form}]
{
\Gamma \vdash_{ECC} \Sigma x : A . B : K
}{
\Gamma \vdash_{ECC} A : \m{type}
&
\Gamma,x : A \vdash_{ECC} B : \m{type}
}
\]

\[
\infer[\m{pair}]
{
\Gamma \vdash_{ECC} \m{pair}_{\Sigma x : A . B} (M,N) : \Sigma x : A . B
}{
\Gamma \vdash_{ECC} M : A
&
\Gamma,x : A \vdash_{ECC} N : [M/x]B
&
\Gamma, x : A \vdash B : \m{prop}
}
\]


\[
\infer[\pi_1]
{
\Gamma \vdash_{ECC} \pi_1(M) : A
}{
\Gamma \vdash_{ECC} M : \Sigma x : A . B
}
\]

\[
\infer[\pi_2]
{
\Gamma \vdash_{ECC} \pi_2(M) : [\pi_1 M / x] B
}{
\Gamma \vdash_{ECC} M : \Sigma x : A . B
}
\]


\label{ecc:type}
\end{definition}

The strong sum $\Sigma$'s formation rule is slightly different from that of $\Pi$ in that 
$\Sigma x : A . B$ is not permitted to have type $\m{prop}$.  Hook showed that allowing strong existential
types to be impredicative, and thus have type $\m{prop}$ in the calculus of constructions, 
was equivalent to allowing the rule $\m{type} : \m{type}$, known to be inconsistent 
\citep{hook1986impredicative}.  However, this does not pose a problem for translation from $CICC$, 
since the only locations we ever encounter a strong sum is as the first argument of a dependent product, 
of the form $\Pi x : (\Sigma y : A . B) . C$, and thus can always have type $\m{type}$.


\begin{theorem}
\textbf{(Strong Normalization for $ECC_{\beta}$)}  $\forall M \in \m{Term}_{ECC_{\eta\beta}}. SN(M)$
\label{ecc:sn}
\end{theorem}

The proof is due to Lou \citep{luo1989ecc}.

\newtheorem{conjecture}[theorem]{Conjecture}

\begin{conjecture}
\textbf{(Strong Normalization for $ECC_{\beta\eta}$)}  $\forall M \in \m{Term}_{ECC_{\eta\beta}}. SN(M)$
\label{ecc:sne}
\end{conjecture}

While no proof of this conjecture currently exists, it has been used
as a conjecture in other proof systems \citep{coquand2003logical}.  
As strong normalization has been shown for the simpler system, $ECC_\beta$, it is not
far fetched to assume that the theorem extends to $\eta$ expansion as it does in the $CC$ case 
\citep{barthe1995extensions}.


