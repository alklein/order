\section{Bicolored Calculus of Constructions}

His system is based on the bicolored calculus of constructions, where two product operators exist to 
differentiate between arguments that need to be provided explicitely and explicitly.  
Miquel \citep{miquel2001implicit} provides the more general system, ICC to allow for implicit arguments.
Here, I will briefly explain the system and some of the relevant theoretical results that have been obtained.
As maintaining the flexibility of the system is important to future extentions of the Caledon language, 
I will present the implicit calculus in terms of Pure Type Systems.

\begin{figure}[h]
\[ 
E ::= V 
 \orr S 
 \orr E\;E 
 \orr E\;[E]
 \orr \lambda V . E 
 \orr \Pi V : E . E 
 \orr \forall V : E . E 
\]
\caption{Syntax of ICC}
\label{icc:syntax}
\end{figure}
