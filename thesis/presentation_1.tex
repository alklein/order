%%%%%%%%%%%%%%%%%%%%%%%%%%%%%%%%%%%%%%%%%
% Beamer Presentation
% LaTeX Template
% Version 1.0 (10/11/12)
%
% This template has been downloaded from:
% http://www.LaTeXTemplates.com
%
% License:
% CC BY-NC-SA 3.0 (http://creativecommons.org/licenses/by-nc-sa/3.0/)
%
%%%%%%%%%%%%%%%%%%%%%%%%%%%%%%%%%%%%%%%%%

%----------------------------------------------------------------------------------------
%	PACKAGES AND THEMES
%----------------------------------------------------------------------------------------

\documentclass[hyperref={pdfpagelabels=false}]{beamer}

\usepackage{verbatim} 
\usepackage[round]{natbib}



\renewcommand{\rmdefault}{ppl}
\renewcommand{\sfdefault}{phv}

\usepackage{proof-dashed}
\usepackage{dashrule}


\newcommand{\m}[1]{\mathsf{#1}}

% symbols of linear logic
\newcommand{\lolli}{\multimap}
\newcommand{\tensor}{\otimes}
\newcommand{\with}{\mathbin{\binampersand}}
\newcommand{\paar}{\mathbin{\bindnasrepma}}
\newcommand{\one}{\mathbf{1}}
\newcommand{\zero}{\mathbf{0}}
\newcommand{\bang}{{!}}
\newcommand{\whynot}{{?}}
\newcommand{\bilolli}{\mathrel{\raisebox{1pt}{\ensuremath{\scriptstyle\circ}}{\lolli}}}
% \oplus, \top, \bot

% symbols of pi-calculus
\newcommand{\FN}{\mathsf{fn}} % free names
% \mid for parallel composition
\newcommand{\recv}[2]{#1(#2)}
\newcommand{\send}[2]{\overline{#1}\langle #2\rangle}
% \newcommand{\zero}{\boldsymbol{0}} % already above
\newcommand{\fwd}{\leftrightarrow}
\newcommand{\case}{\mathsf{case}}
\newcommand{\inl}{\mathsf{inl}}
\newcommand{\inr}{\mathsf{inr}}

% symbols of linear lambda-calculus
\newcommand{\pair}[2]{\langle #1, #2\rangle}
\newcommand{\llet}{\m{let}\;}
\newcommand{\iin}{\mathrel{\m{in}}}
\newcommand{\ccase}{\m{case}\;}
\newcommand{\oof}{\mathrel{\m{of}}}
\newcommand{\unit}{\langle\,\rangle}
\newcommand{\abort}{\m{abort}}

% substructural operational semantics
\newcommand{\eval}{\m{eval}}
\newcommand{\retn}{\m{retn}}
\newcommand{\cont}{\m{cont}}
\newcommand{\hole}{\mbox{\tt\char`\_}}
\newcommand{\rep}[1]{\ulcorner #1\urcorner}

% ordered logic
\newcommand{\gnab}{\mbox{!`}}
\newcommand{\fuse}{\bullet}
\newcommand{\esuf}{\circ}
\newcommand{\rimp}{\twoheadrightarrow}
\newcommand{\limp}{\rightarrowtail}

\newcommand{\orr}{\; | \; }

\usepackage{amsmath}
\usepackage{stmaryrd}

\usepackage{subfigure}
\usepackage{color}
\usepackage{graphicx}
\usepackage{booktabs}


\usepackage{listings}
\lstset{
language=Haskell,               % choose the language of the code
basicstyle=\footnotesize,       % the size of the fonts that are used for the code
numbers=left,                   % where to put the line-numbers
numberstyle=\footnotesize,      % the size of the fonts that are used for the line-numbers
stepnumber=1,                   % the step between two line-numbers. If it is 1 each line will be numbered
numbersep=5pt,                  % how far the line-numbers are from the code
backgroundcolor=\color{white},  % choose the background color. You must add \usepackage{color}
showspaces=false,               % show spaces adding particular underscores
showstringspaces=false,         % underline spaces within strings
showtabs=false,                 % show tabs within strings adding particular underscores
frame=none,           % adds a frame around the code
tabsize=2,          % sets default tabsize to 2 spaces
captionpos=b,           % sets the caption-position to bottom
breaklines=true,        % sets automatic line breaking
breakatwhitespace=false,    % sets if automatic breaks should only happen at whitespace
escapeinside={\%*}{*)}          % if you want to add a comment within your code
}


\mode<presentation> {


\usetheme{Marburg}

\setbeamertemplate{footline}[page number] % To replace the footer line in all slides with a simple slide count uncomment this line
% \setbeamertemplate{navigation symbols}{} % To remove the navigation symbols from the bottom of all slides uncomment this line
}



%----------------------------------------------------------------------------------------
%	TITLE PAGE
%----------------------------------------------------------------------------------------

\title[Caledon]{Logic Programming and Type Inference with the Calculus of Constructions } % The short title appears at the bottom of every slide, the full title is only on the title page

\author{Matthew Mirman} % Your name
\institute[CMU SCS] % Your institution as it will appear on the bottom of every slide, may be shorthand to save space
{
School of Computer Science, Carnegie Mellon University \\
\medskip
\textit{mmirman@andrew.cmu.edu} \\
\medskip
\textbf{Advisor:} Frank Pfenning \textit{fp@cs.cmu.edu}
}
\date{\today} % Date, can be changed to a custom date

\begin{document}

\begin{frame}
\titlepage % Print the title page as the first slide
\end{frame}


%----------------------------------------------------------------------------------------
%	PRESENTATION SLIDES
%----------------------------------------------------------------------------------------

\AtBeginPart{
  \begin{frame}
      \partpage
      \tableofcontents
    \end{frame}
}

\AtBeginSection[]{
  \begin{frame}
    \partpage
    \tableofcontents[currentsection]
  \end{frame}
}


\part[Introduction]{Introduction} 
    \section[Background]{Background}

\subsection{History}
\begin{frame}
\frametitle{History}

\begin{itemize}
\item Prolog generalized pattern matching with backtracking logic programming.
\item $\lambda$Prolog gave prolog closures \citep{miller1988overview}.
\item Twelf \citep{pfenning1999system} gave a $\lambda$Prolog like language with with dependent types.
\end{itemize}
\end{frame}

%------------------------------------------------
\begin{frame}
\frametitle{Twelf}
\begin{itemize}
\item It lacks polymorphism and thus useful libraries
\item It can search for proofs as the method of programming 
\item It does does not use these found proofs as programs
\item Very good for theorem proving
\item No IO
\end{itemize}

\end{frame}

%------------------------------------------------
\subsection{Caledon}
\begin{frame}

\frametitle{Caledon}
\begin{itemize}
\item Caledon, turns twelf into a general purpose language
\item Has implicit multi-universe polymorphism
\item Has IO
\item Has nondeterminism control
\item Has instance arguments
\item Instance arguments support adhoc typeclasses
\end{itemize}
\end{frame}

%------------------------------------------------

\begin{frame}
\frametitle{Dependent types and Proof Search}
\begin{itemize}

\item Dependent types allows you to mention code in your types.
\item Type inference allows you to generate types from your code.
\item Logic programming allows you generate code from your code.
\item since code is used in types, use logic programming for type inference to generate types for your code.
\item since types are used as logic programming code, type inference should come up with logic code.

\end{itemize}
\end{frame}

%------------------------------------------------

\begin{frame}
\frametitle{Result}
\begin{itemize}

\item types get inferred types which get inferred types... 
\item type inference becomes the runtime mechanism.

\end{itemize}
\end{frame}

%------------------------------------------------

\begin{frame}
\frametitle{Mechanism}
\begin{itemize}

\item Proof holes get filled by unification, then proof search.
\item Instance arguments allow for the automatic insertion of holes.
\item polymorphism permits generalization to logic programs in general.

\end{itemize}
\end{frame}

%------------------------------------------------

\begin{frame}
\frametitle{Uses}
\begin{itemize}

\item Programs which have different types on different computers, or for different users.
\item Metaprogramming for flexible syntax.
\item Take derivatives of types to generate types for zipper datastructures.
\item Ensure linearity of a type.

\end{itemize}

\end{frame}

%------------------------------------------------


    \section[Logic Primer]{Logic Programming Primer}

%------------------------------------------------
\subsection[Basics]{Learning by Addition}
%------------------------------------------------

\begin{frame}[fragile]
\frametitle{Logic Programming Basics}

\begin{lstlisting}
defn add : nat -> nat -> nat -> prop
   | addZ = add zero A A
   | addS = add (succ A) B (succ C) 
             <- add A B C
\end{lstlisting}
\end{frame}

%------------------------------------------------

\begin{frame}[fragile]
\frametitle{Comparing it to Haskell}

\begin{lstlisting}
add :: nat -> nat -> nat
add Zero a = a
add (Succ a) b = Succ c
   where c = add a b
\end{lstlisting}
\end{frame}

%------------------------------------------------
\subsection[Conceptualizing]{Conceptualizing Logic Programs}
%------------------------------------------------

\begin{frame}
\frametitle{Conceptualizing}
\begin{itemize}
\item Search allows for nondeterministic programs.
\item Finding solutions for search games - tic-tac-toe
\item A procedural view can be used
\end{itemize}
\end{frame}

%------------------------------------------------

\begin{frame}[fragile]
\frametitle{Procedural View}

\begin{lstlisting}
defn actNormal : bool -> prop
  >| truth = actNormal true 
            <- print ``first truth''
            <- print ``then the world''
  >| lies = actNormal false
            <- print ``You've failed me''

query a = actNormal true
query b = actNormal false
\end{lstlisting}
\end{frame}

%------------------------------------------------

\begin{frame}[fragile]
\frametitle{Logical View}

Can decide later on whether to match.

\begin{lstlisting}
defn failLater : prop 
  >| c1 = failLater 
           <- print ``prints anyway'' 
           <- false == true
  >| c2 = failLater
           <- print ``then this''

query prg = failLater
\end{lstlisting}
\end{frame}

%------------------------------------------------

\begin{frame}[fragile]
\frametitle{Generalized Patterns}

Enables trivial equality case

\begin{lstlisting}
defn same_nat : nat -> nat -> prop
  >| is_same_nat = same_nat A A
\end{lstlisting}

Haskell would be: 

\begin{lstlisting}
same_nat Zero Zero = True
same_nat (Succ a) (Succ b) = same_nat a b
same_nat _ _ = False
\end{lstlisting}
\end{frame}

%------------------------------------------------

\begin{frame}[fragile]
\frametitle{Predicates From Functions}

We can turn our addition algorithm into an even test trivially.

\begin{lstlisting}
defn even : nat -> prop
  >| is_even = even B <- add A A B

defn odd : nat -> prop
  >| is_even = odd B <- even (succ B)
\end{lstlisting}
\end{frame}

%------------------------------------------------

\begin{frame}[fragile]
\frametitle{Backtracing cases}

We can match on the results of predicates and backtrack accordingly.

\begin{lstlisting}
defn conv : term -> term -> prop
  >| c_app_r = conv (app A B) R
               <- conv A (lam F)
               <- conv B B'
               <- conv (F B') R
  >| c_app_ir = conv (app A B) (app (app A1 A2) B')
               <- conv A (app A1 A2)
               <- conv B B'
  >| c_lam = conv (lam F) (lam F) 
\end{lstlisting}
\end{frame}

%------------------------------------------------
\subsection[HOLP]{Higher order logic programming}
%------------------------------------------------

\begin{frame}[fragile]
\frametitle{Function Match}

Here's some unnecessarily verbose code.

\begin{lstlisting}
func (Var a) = code1
func (Forall var val) = Exists var code
func (Exists var val) = Forall var code
func (And a b) = code2
\end{lstlisting}
\end{frame}

%------------------------------------------------

\begin{frame}[fragile]
\frametitle{Logic Version}

Here's the logic solution

\begin{lstlisting}
defn func : term -> term -> prop 
  | f1 = func (var A) R <- [code1]
  | f2 = func (F Var Val) (f Var R) <- [code]
  | f3 = func (and A B) R <- [code2]
\end{lstlisting}
\end{frame}

%------------------------------------------------

\begin{frame}[fragile]
\frametitle{Computation with Free Variables}

We can also use higher order logic programming to make computations even when 
there are free variables.

\begin{lstlisting}
defn conv : term -> term -> prop
   | c_lam = conv (lam F) (lam F') 
            <- [x : term] conv (F x) (F' x)
\end{lstlisting}
\end{frame}

%------------------------------------------------
\subsection{Type Classes}
%------------------------------------------------

\begin{frame}[fragile]
\frametitle{Type classes in Haskell}

Type classes in haskell allow you to append constraints
to paramaterized values.

\begin{lstlisting}
show :: Show a => a -> String
\end{lstlisting}
\end{frame}

%------------------------------------------------

\begin{frame}[fragile]
\frametitle{Type classes in Caledon}

In Caledon we do this with an implicit argument.
\begin{lstlisting}
defn show : showC A => A -> String
defn show : {unused : showC A } A -> string
\end{lstlisting}
\end{frame}

%------------------------------------------------

\begin{frame}[fragile]
\frametitle{Modern haskell}

Modern haskell allows for logic style type class computation.
\verb|res| will result in ``SSSZ'' despite both
the arguments and cases for add being undefined.

\begin{lstlisting}
data Z
data S a

instance Show Z where
   show _ = ``Zero''
instance Show a => Show (S a) where
   show _ = show (undefined :: a)

class Add a b c | a b -> c where
   add :: a -> b -> c
instance Add Z a a
instance Add a b c => Add (S a) b (S c)

res = show (add (undefined :: S (S Z)) 
                (undefined :: S Z)) 
\end{lstlisting}
\end{frame}

%------------------------------------------------

\begin{frame}[fragile]
\frametitle{Caledon Equivalent}

We can see that \verb|add| actually performs no computation.

\begin{lstlisting}
defn nat2 : nat -> prop
   | any_nat2 = {r : nat} nat2 r

defn add : {a b c : nat}{ r : add a b c } 
           nat2 a -> nat2 b -> nat2 c
  as ?\ a b c . ?\ adder . \ na . \ nb . 
      any_nat2 c

defn show_nat2 : nat2 A -> string -> prop
   | show-A = show_nat2 (any_nat2 : nat2 A) St 
           <- show_nat A St

query res = findOne $ show_nat2 $ 
      add (any_nat2 : nat2 (succ (succ zero)))
          (any_nat2 : nat2 (succ zero))
                                 
\end{lstlisting}
\end{frame}

     
\part[Types]{Types} 
     \section{Overview}

% -----------------------------------------
\begin{frame}
\frametitle{Overview}
\begin{itemize}
\item Based on the Calculus of Construction ($CC$)
\item Caledon Implicit Calculus of Constructions $CICC$
\item And Implicit Caledon Implicit Calculus of Constructions $CICC^-$
\item Transforms $CICC^- \Rightarrow CICC \Rightarrow CC \Rightarrow CICC$ 
\end{itemize}
\end{frame}
% -----------------------------------------
\begin{frame}
\frametitle{Derivation of Types}
\begin{itemize}
\item $CC$ is typed by inference rules and first principles.
\item $CICC$ is typed by transformation into $CC$.
\item $CICC^-$ is typed by inference rules.
\item $CICC^-$ is not necessarily consistent or even preserving (to be shown?)
\item $CC \subset CICC \subset CICC^-$
\end{itemize}
\end{frame}

     \section[$CC$]{The Calculus of Constructions}
% -----------------------------------------

\subsection[Overview]{Overview}

\begin{frame}
\frametitle{The Calculus of Constructions}
\begin{itemize}
\item Defined by Coquand \citep{coquand1986calculus}.
\item A pure type system
\item When extended with universes, a good theorem proving language.

\end{itemize}
\end{frame}

% -----------------------------------------

\begin{frame}
\frametitle{Pure Type Definition}

\begin{definition}
\textbf{(PTS for $CC$)}

\begin{align}
A &= \{ *, \Box \}
\\
S &= \{ (* : \Box) \}
\\
R &= \{ (*,*,*),(*,\Box,\Box),(\Box,\Box,\Box),(\Box,*,*)\}
\end{align}  
\end{definition}

\end{frame}

% -----------------------------------------
\subsection{Properties}

\begin{frame}
\frametitle{Properties}

\textbf{(PTS for $CC$)}

\begin{itemize}
\item Terms can depend on types
\item Terms can depend on terms
\item Types can depend on terms
\item Types can depend on types
\end{itemize}

\end{frame}


% -----------------------------------------
\begin{frame}

\frametitle{Notation}

\begin{definition}
If $\Gamma \vdash_{CC} P : T : K$ means $\Gamma \vdash_{CC} P : T$ and $\Gamma \vdash_{CC} T : K$
\end{definition}


\begin{definition}
$ \m{Term}_{CC}  = \{ M : \exists T,\Gamma . \Gamma \vdash_{cc} M : T \}$
\end{definition}

\end{frame}

% -----------------------------------------

\begin{frame}
\frametitle{Consistency}

Strong normalization in the Calculus of Constructions implies consistency.

\begin{theorem}
\textbf{(Strong Normalization)} $\forall M \in \m{Term}_{CC}. SN(M)$
\label{cc:cons}
\end{theorem}

The easiest proof is due to Geuvers \citep{Geuvers94ashort} 

\end{frame}

% -----------------------------------------

\begin{frame}
\frametitle{Impredicativity}

\begin{itemize}
\item Small types can be generalized over small types
\item Useful for general logic programming libraries.
\item Restrictive for metaprogramming (type is not a type).
\item Extended Calculus of Constructions due to \citet{luo1989ecc} 
\item Cumulative universe inference employed in Caledon \citep{callaghan2001implementation}.
\end{itemize}

\end{frame}


% -----------------------------------------
\subsection{Theorems in Caledon}
\begin{frame}
\frametitle{Theorems in Caledon}

\begin{itemize}
\item Caledon, programs might not terminate. 
\item Caledon programs output consistent theorems in the Calculus of Constructions.
\end{itemize}

\begin{definition}
In the Caledon language, $\m{prop} = *$ and $\m{type} = \Box$
\end{definition}

\begin{definition}
If $\Gamma \vdash_{CICC} P : T : \m{prop}$ in the caledon language, then $T$ is a theorem, and $P$ is a proof.
\end{definition}

\end{frame}

     \section[$CICC$]{Caledon Implicit Calculus of Constructions}

% -------------------------------------------------------------

\subsection{Overview}

\begin{frame}
\frametitle{Caledon Implicit Calculus of Constructions}
\begin{itemize}
\item $CICC$ is typed by transformation into $CC$.
\item Based on the ``Bicolored Calculus of Constructions'' ($CC^{Bi}$) \citep{luther2001more}.
\item includes a form of non alpha convertible binding.
\item interpretation of the unification problem form ($UPF$) generated by elaboration.
\end{itemize}
\end{frame}


% -------------------------------------------------------------


\subsection{Definition}

\begin{frame}
\frametitle{Syntax}
\begin{definition}
\textbf{(CICC Syntax)}
\[ 
E ::= 
V 
\orr S 
\orr E\;E 
\orr \lambda V : T. E 
\orr ?\lambda V : T. E 
\orr \Pi V : E . E 
\orr ?\Pi V : E . E 
\orr E \{ V : E = E \}
\]

\end{definition}

\begin{itemize}
\item The \textit{dependent} explicit and implicit products are written $\Pi v : E . E $ and $?\Pi v : A . E$. 
\item The \textit{non-dependent} explicit and implicit products are written $T \rightarrow T$ 
      and $T \Rightarrow T$ respectively.
\end{itemize}
\end{frame}


% -------------------------------------------------------------

\begin{frame}[fragile]
\frametitle{Definition Extraction}
We care about the transforming the constrained binders.
We interpret names as a kind of record modifier.
\begin{lstlisting}
defn nat : prop 
  as [a : prop] a -> (a -> a) -> a

defn nat_1 : nat -> prop
  as \ N : nat . [a : nat -> prop] a zero -> succty a -> a N

defn rec : nat -> prop -> prop
  as \ nm : nat . \ N : kind . nat_1 nm * N

defn get : [N : kind] [nm : nat] nat_1 nm * N -> N
  as \ N : kind . \nm : nat . \ c : (nat_1 nm, N) . snd c

defn put : [N : kind] [nm : nat] nat_1 nm ->  N -> nat_1 nm * N
  as \ N : kind . \ nm : nat . \nmnm : nat_1 nm . \ c : N . pair nmnm N
\end{lstlisting}
\end{frame}



% -------------------------------------------------------------
\newcommand{\CICCproj}[1]{ \left\llbracket #1 \right\rrbracket_{ci}}

\begin{frame}
\frametitle{Extraction}

\begin{definition}

\textbf{ (Projection from $CICC$ to $CC$) }

\[
\CICCproj{v} := v
\]\[
\CICCproj{s} := s
\]\[
\CICCproj{E_1 \; E_2} := \CICCproj{E_1} \; \CICCproj{E_2}
\]\[
\CICCproj{E_1 \; \{ x : T = E \}} := \CICCproj{E_1} \; (\m{put}\;\CICCproj{T}\;\bar{x}\; \bar{\bar{x}}; \CICCproj{E_2} )
\]\[
\CICCproj{\lambda v : T . E } := \lambda v : \CICCproj{T} . \CICCproj{E}
\]\[
\CICCproj{?\lambda v : T . E } := \lambda y : \m{rec}\;\bar{v}\; \CICCproj{T} . \CICCproj{ [ \m{get}\; \CICCproj{T}\; \bar{v}\; y  / v ] E}
\]\[
\CICCproj{\Pi v : T . E } := \Pi v : \CICCproj{T} . \CICCproj{E}
\]\[
\CICCproj{?\Pi v : T . E } := \Pi y : \m{rec}\;\bar{v}\;\CICCproj{T} . \CICCproj{ [ \m{get}\;\CICCproj{T}\; \bar{v}\; y  / v ] E}
\]

where $y$ is generally fresh
\end{definition}
\end{frame}


% -------------------------------------------------------------

\begin{frame}
\frametitle{Extraction Example}
\begin{example}
$N \{ x\; : \; T \; = \;A \} $
would become $N\;( \m{put} \; T \; \bar{x} \; \bar{\bar{x}} \; A)$.
\end{example}
\end{frame}


% -------------------------------------------------------------

\begin{frame}
\frametitle{Typing}


\begin{definition}
\textbf{(Typing for $CICC$)} We say $\Gamma \vdash_{ci} A : T$ iff $\CICCproj{\Gamma} \vdash_{cc} \CICCproj{A} : \CICCproj{T}$
\label{cicc:typing}
\end{definition}

\end{frame}

% -------------------------------------------------------------

\subsection{Results}

\begin{frame}
\frametitle{Initial Theorems}

We need to ensure that extraction preserves substitution.

\begin{theorem}
\textbf{(Projection Substitution)}  

$\CICCproj{[A/x] B} = [ \CICCproj{A} / x ] \CICCproj{B}$ provided $x$ is free for $A$ in $B$.

\end{theorem}

Reductions in $CICC$ also correspond to reductions in $CC$.

\begin{lemma}
\textbf{(Reduction Translation)}
Forall $M, N \in \m{Term}_{ci}$ if $M \rightarrow_{\beta\eta*} N$ then 
$\CICCproj{M} \rightarrow_{\beta\eta*} \CICCproj{N}$
\label{cicc:red}
\end{lemma}

\end{frame}

% -------------------------------------------------------------

\begin{frame}
\frametitle{Semantic Equivalence}

Semantic Equivalence ensures that reductions in $CC$ correspond to reductions in $CICC$.

\begin{theorem}
\textbf{(Semantic Equivalence)}

Forall $M \in \m{Term}_{ci}$ such that $\CICCproj{M} \rightarrow_{\beta\eta*} N'$ and 
$\Gamma \vdash M : T$ implies that there exists $M' \in \m{Term}_{cicc}$ such that 
$M \rightarrow_{\beta\eta*} M'$ and $\CICCproj{M'} \equiv N'$

\end{theorem}
\end{frame}

% -------------------------------------------------------------

\begin{frame}
\frametitle{Strong Normalization}

Strong normalization gives us consistency.

\begin{theorem}
\textbf{(Strong Normalization)} $\forall M \in \m{Term}_{ci}. SN(M)$
\label{ci:sn}
\end{theorem}


\end{frame}



\part[Semantics]{Runtime} 
    o\section[$CICC^-$]{Implicit Caledon Caledon Calculus of Constructions}

\subsection{Overview}

\begin{frame}
\frametitle{Implicit Caledon Implicit Calculus of Constructions}
\begin{itemize}
\item $CICC^-$ is the front facing language.
\item Can be considered vital to the ``Runtime''
\item $CICC^- \Rightarrow UPF \Rightarrow CICC$
\end{itemize}
\end{frame}

% -------------------------------------------------------------

\begin{frame}
\frametitle{$CICC^-$ Types}
\begin{itemize}
\item $CICC^-$ is typed with traditional inference rules.
\item Based on the the ``Implicit Calculus of Constructions'' ($ICC$) \citep{miquel2001implicit}.
\item Borrows notion of implicit instantiation of terms from $ICC$
\item Adds mechanisms for implicit binding of terms. 
\item The subtyping relation needs to be an equivalence class for traditional unification.
\item most methatheory for $CICC^-$ ignored as it is a transitional language.
\end{itemize}
\end{frame}

% -------------------------------------------------------------

\subsection{Definition}

\begin{frame}
\frametitle{Syntax}

\begin{itemize}
\item $E\; \{ V : A = E \}$ is now simply $ E \; \{ V  = E \}$
\item type filled in during elaboration to $CICC$
\end{itemize}

\end{frame}


\newcommand{\judgeCI}{ \vdash_{i^-}}

\begin{frame}
\frametitle{Types}

New and interesting rules.  The one shown showcases what is borrowed from $ICC$.

\begin{definition}
\textbf{($CICC^-$ Extended Typing Rules)}

%% inst/f %%
%%%%%%%%%%%%
\[
\infer[\m{inst/f}]
{
\Gamma \judgeCI M : [N/x]U 
}
{
\Gamma \judgeCI M : ?\Pi x :T . U
&
\Gamma \judgeCI N : T
&
x \notin DV(\Gamma)
}
\]

\end{definition}
\end{frame}

% -------------------------------------------------------------
\begin{frame}

\frametitle{Substitution}

\begin{theorem}
\textbf{(Substitution)}
\[
\infer-[\m{subst}]{ 
\Gamma, [N/x]\Gamma' \judgeCI [N / x]M : [N/x]T_2
}{
\Gamma, x : T_1, \Gamma' \judgeCI M : T_2
&
\Gamma \judgeCI N : T_1
}
\]
\label{ci:sub}
\end{theorem}

\end{frame}

% -------------------------------------------------------------
\newcommand{\CICCmproj}[1]{ \left\llbracket #1 \right\rrbracket_{ci^{-}}}

\begin{frame}
\frametitle{Projection to $CICC^-$}

Only one case is presented here.

\begin{definition}
\textbf{ (Projection from $CICC^{-}$ to $CICC$) }

%% inst/f %%
%%%%%%%%%%%%
\[
\CICCmproj{ 
\infer[\m{inst/f}]
{
\Gamma \judgeCI M : U [N/x]
}
{
\overset{\mathcal{D}_1}{ \Gamma \judgeCI M : ?\Pi x : T . U }
&
\overset{\mathcal{D}_2}{ \Gamma \judgeCI N : T }
&
\cdots
}
}
\]
\[
:=
\CICCmproj{\mathcal{D}_1} \; \{ x = \CICCmproj{\mathcal{D}_2} \}
\]

\end{definition}
\end{frame}


\begin{frame}

\frametitle{Soundness}

\begin{theorem}

\textbf{(Soundness of extraction)}  

\begin{alignat}{4}
\Gamma &\judgeCI &  & \implies & \CICCproj{\Gamma \judgeCI}^c & \judgeCI &
\\
\Gamma &\judgeCI & A : T & \implies & \CICCproj{\Gamma \judgeCI}^c & \judgeCI & \CICCproj{ \Gamma \judgeCI A : T }
\end{alignat}

\label{cicc-:sound}
\end{theorem}

\end{frame}

%%%%%%%%%%%%%%%%%%%%%%%%%%%%%%%%%%%%%%%%%%%%%%%%%%%%%%%%%%%%%
%%% Subtyping %%%%%%%%%%%%%%%%%%%%%%%%%%%%%%%%%%%%%%%%%%%%%%%
%%%%%%%%%%%%%%%%%%%%%%%%%%%%%%%%%%%%%%%%%%%%%%%%%%%%%%%%%%%%%
\subsection{Subtyping}
%--------------------------------------------------------------

\begin{frame}
\frametitle{Soundness}
\begin{definition}
Subtyping relation:
$\Gamma \judgeCI T \leq T' \;\; \equiv \;\; \Gamma, x : T \judgeCI x : T'$  where $x$ is new.
\end{definition}

\begin{lemma}
Subtyping is a preordering.
\end{lemma}

\begin{theorem}
\textbf{(Symmetry)}
$\Gamma \judgeCI A \leq B $ implies 
$\Gamma \judgeCI B' \leq A' $. where $A \equiv_{\beta} A'$ and $B \equiv_{\beta} B'$
\label{ci:sym}
\end{theorem}

\end{frame}

%--------------------------------------------------------------

    \section{Semantics}

% -------------------------------------------------------------
\subsection{Overview}
% -------------------------------------------------------------

\begin{frame}
\frametitle{History}
\begin{itemize}
\item Notation presented is based on that given by \citet{pfenning1991logic}.
\item Algorithm extended is based on that given by \citep{pfenning1991unification}.
\end{itemize}
\end{frame}


% -------------------------------------------------------------

\begin{frame}
\frametitle{Higher Order Unification With Search}
\begin{itemize}
\item Find a mapping from existential variables which makes two $CICC^-$ equivalent.
\item Can be decidable for the pattern fragment of terms.
\item We extend with a couple rules for cases like $inst/f$.
\item Unification problems are a target for elaboration.
\end{itemize}
\end{frame}



% -------------------------------------------------------------
\subsection[$UPF$]{Unification Problem Form}

\begin{frame}
\frametitle{Syntax}
\begin{definition}
\textbf{Unification Problem Form}
\[
U ::= U \wedge U 
 \orr \forall V : T . U
 \orr \exists V : T . U 
 \orr U \doteq U
\]\[
 \orr \top
  \orr T \in T 
  \orr T \in T >> T \in T
\]

\end{definition}

Can be provided meaning with inference rules.

\end{frame}



% -------------------------------------------------------------

\begin{frame}
\frametitle{Implementation}
\begin{itemize}
\item Finger tree zipper datastructure is used to describe the unification problem.
\item Solvable problems are searched in a nearest neighborhood. 
\item Subtrees are blocked until they are changed by substitution.
\end{itemize}
\end{frame}

    \section{Language}

% -------------------------------------------------------------
\subsection[Families]{Type Families}
% -------------------------------------------------------------
\begin{frame}
\frametitle{Type Families}
\begin{itemize}
\item Permitting entirely polymorphic axioms at the top level complicates proof search.
\item Axioms grouped by conclusion.
\item Optimizes proof search because it is now possible to limit the search.
\item A bit like ``freeze'' from Twelf.  
\item Mutually recursive definitions are automatically inferred.
\end{itemize}
\end{frame}
% -------------------------------------------------------------

\begin{frame}[fragile]
\frametitle{Type Families Example}

Caledon:

\begin{lstlisting}
defn add : nat -> nat -> nat -> prop
   | addZ = add zero A A
   | addS = add (succ A) B (succ C) 
             <- add A B C
\end{lstlisting}

Twelf:

\begin{lstlisting}
add : nat -> nat -> nat -> type.
addZ : add zero A A.
addS : add (succ A) B (succ C) <- add A B C.
%freeze add.
\end{lstlisting}
\end{frame}


% -------------------------------------------------------------
\subsection[Nondeterminism]{Nondeterminism Control}
% -------------------------------------------------------------
\begin{frame}
\frametitle{Nondeterminism Control}
\begin{itemize}
\item Would like to choose between depth first and breadth first search.
\item Can do this by adding syntax to axioms.
\end{itemize}
\end{frame}
% -------------------------------------------------------------
\begin{frame}[fragile]
\frametitle{Nondeterminism Control Example}

“ttt vqvqvqvq jjj”

\begin{lstlisting}

query main = runBoth false

defn runBoth : bool -> type
  >| run0 = runBoth A
            <- putStr ``ttt ``
            <- A =:= true
   | run1 = runBoth A 
            <- putStr ``vvvv''
            <- A =:= true
   | run2 = runBoth A
            <- putStr ``qqqq''
            <- A =:= true
  >| run3 = runBoth A
            <- putStr `` jjj''
            <- A =:= false

\end{lstlisting}
\end{frame}

  
\part[Conclusion]{Conclusion}
    \section{Uses}
% -------------------------------------------------------------
\subsection{Typeclasses}

\begin{frame}[fragile]
\frametitle{Serialize}

\begin{lstlisting}
defn serializeBool : bool -> string -> type
  >| serializeBool-t = 
       serializeBool true ``true''
  >| serializeBool-f = 
       serializeBool false ``false''

query readQ = 
   exists B : bool. serializeBool B ``true''

query printQ = 
   exists S : string . serializeBool false S
\end{lstlisting}

\end{frame}
% -------------------------------------------------------------

\begin{frame}[fragile]
\frametitle{Type Class for Serialize}

\begin{lstlisting}
defn serialize : {T}{serable : T -> string -> type} 
                 T -> string -> type
   | serializeImp = 
     [ Serer : T -> string -> type ]
     [ Serable : serializable T { serer = Serer }]
     serialize { serable = Serable } V S
     <- Serer V S

defn serializable : 
      [T]{serer : T -> string -> type} type
   | serialize-bool = 
      serializable bool { serer = serializeBool }
   | serialize-nat = 
      serializable nat { serer = serializeNat }

query readQueryBool = 
   exists B : bool . serialize B ``true''
query printQueryNat = 
   exists S . serialize (succ (succ zero)) S
\end{lstlisting}

\end{frame}


% -------------------------------------------------------------
\subsection[Linearity]{Computation During Inference}

\begin{frame}
\frametitle{Putting it all together}
\begin{itemize}
\item Can create a linearity predicate.
\item Uses implicit argument computation
\item acts on higher order abstract syntax
\item works on types by employing universes
\end{itemize}
\end{frame}


\begin{frame}[fragile]
\frametitle{Linearity}
\begin{lstlisting}
defn linear : 
     [T] [ P : T -> type ] 
     ( [ x : T ] P x ) -> type
  >| linear-var = 
      linear T (\ x . T) (\ x : T . x)
  >| restrict-lam = 
      linear T (\ x . [y : A] P x y) 
               (\ x . \ y : A . F y x)
      <- [ y : A ] 
         linear T (\ x . P x y) (\ x . F y x)

defn lolli : 
      [T] ( T -> type ) -> type
   | llam = 
      [T]{TyF}[F : [x : T] TyF x]
      linear T TyF F => lolli T TyF
\end{lstlisting}
\end{frame}


\begin{frame}
\frametitle{Uses of Linearity}
\begin{itemize}
\item Solves unique instance problem
\item Could be used to automate type class instancing
\item Application could get special syntax
\item Subject of future research
\end{itemize}
\end{frame}

    \section{Results}

\begin{frame}
\frametitle{Results}
\begin{itemize}
\item Designed a logic programming language based on $CC$
\item Added implicits in a formal sense.
\item Outlined implementation methods and translations.
\item Gave a method to constrain proof search to a small subset of axioms.
\item Implemented the work in Haskell.
\item Designed a standard library.
\end{itemize}
\end{frame}


    \section*{References}
    \begin{frame}
      \frametitle{References}
      \bibliographystyle{plainnat}
      \bibliography{MyBib}
    \end{frame}


\end{document} 
