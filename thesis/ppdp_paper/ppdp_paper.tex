%for a more compact document, add the option openany to avoid
%starting all chapters on odd numbered pages

\documentclass[12pt]{cmuthesis} %% something else!

% some useful packages
\usepackage{times}
\usepackage{fullpage}
\usepackage{graphicx}
\usepackage{amsmath}
\usepackage{verbatim} 

\usepackage{float}
\floatstyle{boxed}
\restylefloat{figure}
\usepackage{placeins} 

\renewcommand{\rmdefault}{ppl}
\renewcommand{\sfdefault}{phv}

\usepackage{dashrule}
\usepackage{proof-dashed}
\usepackage{verbatim}


\newtheorem{theorem}{Theorem}
\newtheorem{lemma}[theorem]{Lemma}
\newtheorem{definition}[theorem]{Definition}
\newtheorem{corollary}[theorem]{Corollary}

\newenvironment{proof}{\trivlist \item[\hskip \labelsep{\bf
Proof:}]}{\hfill$\Box$ \endtrivlist}
\newenvironment{attempt}{\trivlist \item[\hskip \labelsep{\bf
Proof attempt:}]}{\hfill$\Diamond$ \endtrivlist}

\newcommand{\ednote}[1]{\footnote{\it #1}\message{ednote!}}
\newenvironment{note}{\begin{quote}\message{note!}\it}{\end{quote}}
\newcommand{\highlight}[1]{\par\vspace{\abovedisplayskip}%
\framebox{\addtolength{\linewidth}{-1em}\begin{minipage}{\linewidth}#1\end{minipage}}%
\par\vspace{\belowdisplayskip}}



\usepackage{latexsym}
\usepackage{amssymb}            % for \multimap (-o)
\usepackage{stmaryrd}           % for \binampersand (&), \bindnasrepma (\paar)

\newcommand{\m}[1]{\mathsf{#1}}
\newcommand{\f}[1]{\framebox{#1}}
\newcommand{\DD}{\mathcal{D}}
\newcommand{\EE}{\mathcal{E}}
\newcommand{\FF}{\mathcal{F}}
\newcommand{\palign}[1]{\raisebox{-0.75em}{#1}}
\newcommand{\ih}[1]{\mbox{i.h.}(#1)}

% judgments of linear logic
\newcommand{\eph}{\mathit{eph}}
\newcommand{\pers}{\mathit{pers}}
\newcommand{\um}[1]{\underline{\m{#1}}}

\newcommand{\seq}{\vdash}
\newcommand{\cfseq}{\Rightarrow}
\newcommand{\cseq}{\rightarrow}
\newcommand{\semi}{\mathrel{;}}
\newcommand{\lequiv}{\mathrel{\dashv\vdash}}
\newcommand{\cut}{\m{cut}}
\newcommand{\cutbang}{\m{cut}{!}}
\newcommand{\id}{\m{id}}
\newcommand{\defeq}{\triangleq}
\newcommand{\vvdash}{\mathrel{\vdash\kern-0.8ex\vdash}}

% symbols of linear logic
\newcommand{\lolli}{\multimap}
\newcommand{\tensor}{\otimes}
\newcommand{\with}{\mathbin{\binampersand}}
\newcommand{\paar}{\mathbin{\bindnasrepma}}
\newcommand{\one}{\mathbf{1}}
\newcommand{\zero}{\mathbf{0}}
\newcommand{\bang}{{!}}
\newcommand{\whynot}{{?}}
\newcommand{\bilolli}{\mathrel{\raisebox{1pt}{\ensuremath{\scriptstyle\circ}}{\lolli}}}
% \oplus, \top, \bot

% symbols of pi-calculus
\newcommand{\FN}{\mathsf{fn}} % free names
% \mid for parallel composition
\newcommand{\recv}[2]{#1(#2)}
\newcommand{\send}[2]{\overline{#1}\langle #2\rangle}
% \newcommand{\zero}{\boldsymbol{0}} % already above
\newcommand{\fwd}{\leftrightarrow}
\newcommand{\case}{\mathsf{case}}
\newcommand{\inl}{\mathsf{inl}}
\newcommand{\inr}{\mathsf{inr}}

% symbols of linear lambda-calculus
\newcommand{\pair}[2]{\langle #1, #2\rangle}
\newcommand{\llet}{\m{let}\;}
\newcommand{\iin}{\mathrel{\m{in}}}
\newcommand{\ccase}{\m{case}\;}
\newcommand{\oof}{\mathrel{\m{of}}}
\newcommand{\unit}{\langle\,\rangle}
\newcommand{\abort}{\m{abort}}

% substructural operational semantics
\newcommand{\eval}{\m{eval}}
\newcommand{\retn}{\m{retn}}
\newcommand{\cont}{\m{cont}}
\newcommand{\hole}{\mbox{\tt\char`\_}}
\newcommand{\rep}[1]{\ulcorner #1\urcorner}

% ordered logic
\newcommand{\gnab}{\mbox{!`}}
\newcommand{\fuse}{\bullet}
\newcommand{\esuf}{\circ}
\newcommand{\rimp}{\twoheadrightarrow}
\newcommand{\limp}{\rightarrowtail}


\usepackage[numbers,sort]{natbib}

\usepackage[backref,pageanchor=true,plainpages=false, pdfpagelabels, bookmarks,bookmarksnumbered,
%pdfborder=0 0 0,  %removes outlines around hyper links in online display
]{hyperref}
\usepackage{subfigure}
  
\usepackage{../bussproofs.sty}

\usepackage{color}
\usepackage{listings}
\lstset{ %
language=Haskell,               % choose the language of the code
basicstyle=\footnotesize,       % the size of the fonts that are used for the code
numbers=left,                   % where to put the line-numbers
numberstyle=\footnotesize,      % the size of the fonts that are used for the line-numbers
stepnumber=1,                   % the step between two line-numbers. If it is 1 each line will be numbered
numbersep=5pt,                  % how far the line-numbers are from the code
backgroundcolor=\color{white},  % choose the background color. You must add \usepackage{color}
showspaces=false,               % show spaces adding particular underscores
showstringspaces=false,         % underline spaces within strings
showtabs=false,                 % show tabs within strings adding particular underscores
frame=none,           % adds a frame around the code
tabsize=2,          % sets default tabsize to 2 spaces
captionpos=b,           % sets the caption-position to bottom
breaklines=true,        % sets automatic line breaking
breakatwhitespace=false,    % sets if automatic breaks should only happen at whitespace
escapeinside={\%*}{*)}          % if you want to add a comment within your code
}

 
% Approximately 1" margins, more space on binding side
%\usepackage[letterpaper,twoside,vscale=.8,hscale=.75,nomarginpar]{geometry}
%for general printing (not binding)
\usepackage[letterpaper,twoside,vscale=.8,hscale=.75,nomarginpar,hmarginratio=1:1]{geometry}

\begin{document} 
\frontmatter

%initialize page style, so contents come out right (see bot) -mjz
\pagestyle{empty}

\title{ 
{\bf Type Inference as Interpretation, 
A Dependently Typed Logic Programming Language for Metaprogramming}}
\author{Matthew Mirman, Frank Pfenning}
\date{June 2013}
\Year{2013}
\trnumber{}

\support{}
\disclaimer{}

\keywords{Logic Programming, Pure Type System, 
Type Inference, Higher Order Unification, Caledon Language, Higher Order Abstract Syntax,
Metaprogramming, Universe Checking}

\maketitle

\pagestyle{plain} % for toc, was empty

%% Obviously, it's probably a good idea to break the various sections of your thesis
%% into different files and input them into this file...

\begin{abstract}

In this thesis I present a logic programming language, Caledon, with a pure type system
and a turing complete type inference and implicit argument system based on the semantics of the
object language.  Because the language has dependent types and type inference, terms can be generated 
by providing type constraints.  The addition of control structures such as implicits, costructor hiding,
shared holes, existential quantification, polymorphism, and nondeterminism make the language ideal for 
creating libraries for defining EDSLs.  Furthermore, the system can be made consistent and sound with 
the addition of totality, worlds, and universe checks. Proof irrelivance, universes, and uniqueness 
checking can be used to constrain the nondeterminism of type inference in a pure type system, and help 
programmers control coherence.



\end{abstract}

\mainmatter

%% Double space document for easy review:
\renewcommand{\baselinestretch}{1.66}\normalsize

% The other requirements Catherine has:
%
%  - avoid large margins.  She wants the thesis to use fewer pages, 
%    especially if it requires colour printing.
%
%  - The thesis should be formatted for double-sided printing.  This
%    means that all chapters, acknowledgements, table of contents, etc.
%    should start on odd numbered (right facing) pages.
%
%  - You need to use the department standard tech report title page.  I
%    have tried to ensure that the title page here conforms to this
%    standard.       
%
%  - Use a nice serif font, such as Times Roman.  Sans serif looks bad.
%
% Other than that, just make it look good...

\newtheorem{tcase}{Case}
      
``To be fair, even in last century's typed languages, 
types had a beneficial organisational effect on programmers. 
This century, it's just possible types will have a comparable effect on programs. 
Types are concepts and now mechanisms supporting program-discovery as well as error-discovery. 
I think that's more than just gravy.''
 - Conor McBride


\section{Caledon Implicit Calculus of Constructions}

Caledon's exposed type system is a varient of $ICC$,
which for the rest of the paper I will refer to as $CICC^{-}$.  
This sytem contains two products and two binders - one each for implicit and explicit arguments. 
Unlike $ICC$, there is no rule that allows for an unmarked value to obtain an implicit product type, which
makes type checking somewhat simpler.  
In addition, there is a new form of application to allow for the explicit
selection of an implicit argument to constrain. 
As implicit quantification is permitted on object level variables, semantics must be given with respect to 
the elaborated theory without implicit application.

\begin{figure}[H]
\[ 
E ::= 
V 
\orr S 
\orr E\;E 
\orr \lambda V : T. E 
\orr ?\lambda V : T. E 
\orr \Pi V : E . E 
\orr ?\Pi V : E . E 
\orr E \{ V : E = E \}
\]

\caption{Syntax of $CICC$}
\label{cicc:syntax}
\end{figure}

The \textit{non-dependent} explicit and implicit products are written $T \rightarrow T$ 
and $T \Rightarrow T$ respectively.

Note, that in $CICC$ system, $?\lambda x . A \neg\equiv_\alpha ?\lambda y . [x / y] A$ 
and $?\Pi x . A \neg\equiv_\alpha ?\Pi y . [x / y] A$  if $x \neq y$.  
This implies that the behavior of caledon's implicit argument is something akin to a structural dependent product.  

Before the typing rules of this system can be given, the notion of a constrained name of a term must be defined.

\begin{definition}
The constrained names on a term, written $CN(M)$ is a set defined as follows:

\[
CN(M \{ x = E \}) = \{ x \} \cup CN(M)
\]

\[ 
CN(\m{otherwise}) = \emptyset
\]

\end{definition}

\begin{definition}
The generalized names for a term, written $GN(M)$ is a set defined as follows:

\[ 
GN(?\Pi x : T . M) = \{ x \} \cup GN(M) \cup GN(T)
\]

\[ 
GN(\m{otherwise}) = \emptyset
\]

\end{definition}

\begin{definition}
The bound names for a term, written $BN(M)$ is a set defined as follows:

\[ 
BN(?\lambda x : T . M) = \{ x \} \cup BN(M) \cup BN(T)
\]

\[ 
BN(\m{otherwise}) = \emptyset
\]

\end{definition}

As defined above, bound names and bound variables can no longer be treated the same in the semantics.  
Specifically, $?\lambda x : A . B$ does not have the same semantics as $?\lambda y : A . [y / x] B$.  
This implies that alpha conversion is now severely limited.  

There are a few ways to deal with this.  
The most attractive possibility is to interpret names as a kind of record modifier.
This can be seen as saying $\{ x : T = N \} : \{ x : N \}$, 
and $?\lambda x : N . B$ is really just $\lambda y : \{ x : N \} . [ y.x / x ] N$ where $ .x : \{ x : N \} \rightarrow N$.

\begin{figure}[H]
\begin{lstlisting}
defn nat : prop 
  as [a : prop] a -> (a -> a) -> a

defn nat_1 : nat -> prop
  as \ N : nat . [a : nat -> prop] a zero -> succty a -> a N

defn rec : nat -> prop -> prop
  as \ nm : nat . \ N : kind . nat_1 nm * N

defn get : [N : kind] [nm : nat] nat_1 nm * N -> N
  as \ N : kind . \nm : nat . \ c : (string2 nm, N) . snd c

defn put : [N : kind] [nm : nat] nat_1 nm ->  N -> nat_1 nm * N
  as \ N : kind . \nm : nat . \nmnm : nat_1 nm . \ c : N . pair nmnm N
\end{lstlisting}
\caption{Definitions for extraction}
\label{code:ideal}
\end{figure}

We can further convert this into traditional dependent types by constructing
 type invariants as seen in \ref{code:ideal}.

Then $?\lambda x : A . B$ and $?\Pi x : A . B$ 
becomes $\lambda y : \m{rec}\;\bar{x}\;A . [get A \bar{x} y / x ] B$
and $?\Pi y : \m{rec}\;\bar{x}\;A . [get A \bar{x} y / x ] B$.

Similarly, $N \{ x\; : \; T \; = \;A \} $
would become $N\;( \m{put} \; T \; \bar{x} \; \bar{\bar{x}} \; A)$.

One might notice that $N$ in $\m{get}$ is of type $\m{kind}$.  
In simple $CC$, this is unfortunately not an actual type. 
Rather, it refers to the use of either $\m{type}$ or $\m{prop}$.  
Allowing $N : \m{type}$ is not permited in the standard $CC$ since it is the subject of quantification.
This is possible in $CC_\omega$ however.
In this case, $\m{kind}$ would always refer to the next universe after the highest
universe mentioned in the program.
In the Caledon language implementing simple $CC$ we are free to define 
$\m{kind}$ as $\m{type}_1$, since it will always be larger than any
type or kind mentioned in a Caledon program.  
Fortunately, Geuvers' proof \citep{geuvers1993logics} of strong normalization in the presence of 
$\eta$ conversion applies to the Calculus of Construction with one impredicative universe and two predicative
universes.

This intuitive conversion leads to the following typing rules for $CICC$.

\begin{definition}
\textbf{(Typing for $CICC$)}

\[ \begin{array}{lr}
\infer[\m{wf/e}]
{
\cdot \vdash_{ci} 
}{}
&
\infer[\m{wf/s}]
{
\Gamma, x : T \vdash_{ci} 
}
{
\Gamma \vdash_{ci} T : s
&
x \notin DV(\Gamma)
}
\end{array} \]

%% axioms %%
%%%%%%%%%%%%
\[
\infer[\m{axioms}]
{
\Gamma \vdash_{ci} c : s
}
{
\Gamma \vdash_{ci}
&
(c,s) \in A
}
\]

%% start %%
%%%%%%%%%%%
\[
\infer[\m{start}]
{
\Gamma,x:A \vdash_{ci} x :A
}
{
\Gamma \vdash_{ci} A:s
&
s \in S
}
\]

%% prod %%
%%%%%%%%%%
\[
\infer[\m{prod}]
{
\Gamma \vdash (\Pi x : A . B) : s_3
}
{
\Gamma \vdash A : s_1
&
\Gamma,x:A \vdash B : s_2
&
(s_1,s_2,s_3) \in R
}
\]

%% prod* %%
%%%%%%%%%%%
\[
\infer[\m{prod}*]
{
\Gamma \vdash (?\Pi x : A . B) : s_3
}
{
\Gamma \vdash A : s_1
&
\Gamma,x:A \vdash B : s_2
&
(s_1,s_2,s_3) \in R
}
\]

%% gen %%
%%%%%%%%%
\[
\infer[\m{gen}]
{
\Gamma \vdash_{ci} \lambda x : T . M : (\Pi x : T . U)
}
{
\Gamma , x : T \vdash_{ci} M : U
&
\Gamma \vdash_{ci} (\Pi x : T . U) : s
&
s \in S
&
x \notin FV(M) \cup BN(M) \cup GN(U)
}
\]

%% gen* %%
%%%%%%%%%%
\[
\infer[\m{gen}*]
{
\Gamma \vdash_{ci} ?\lambda x : T . M : (?\Pi x : T . U)
}
{
\Gamma , x : T \vdash_{ci} M : U
&
\Gamma \vdash_{ci} (?\Pi x : T . U) : s
&
s \in S
&
x \notin FV(M) \cup BN(M) \cup GN(U)
}
\]

%% app %%
%%%%%%%%%
\[
\infer[\m{app}]
{
\Gamma \vdash_{ci} M N : U [N/x]
}
{
\Gamma \vdash_{ci} M : \Pi x : T . U
&
\Gamma \vdash_{ci} N : T
}
\]

%% inst/b %%
%%%%%%%%%%%%
\[
\infer[\m{inst/b}]
{
\Gamma \vdash_{ci} M \{ x : T = N \} : U [N/x]
}
{
\Gamma \vdash_{ci} M : ?\Pi x : T . U
&
\Gamma \vdash_{ci} N : T
& 
x \notin GN(M)
&
x \notin BN(U)
}
\]

\label{cicc:typing}
\end{definition}


\subsection{Main Results}

Most of the theorems relating to $CICC$ can be obtained by a simple projection into $CC$.

\newcommand{\CICCproj}[1]{ \left\llbracket #1 \right\rrbracket_{ci}}

\begin{definition}
\textbf{ (Projection from $CICC$ to $CC$) }

\[
\CICCproj{v} := v
\]

\[
\CICCproj{s} := s
\]

\[
\CICCproj{E_1 \; E_2} := \CICCproj{E_1} \; \CICCproj{E_2}
\]

\[
\CICCproj{E_1 \; \{ x : T = E \}} := \CICCproj{E_1} \; (\m{put}\;\CICCproj{T}\;\bar{x}\; \bar{\bar{x}}; \CICCproj{E_2} )
\]

\[
\CICCproj{\lambda v : T . E } := \lambda v : \CICCproj{T} . \CICCproj{E}
\]

\[
\CICCproj{?\lambda v : T . E } := \lambda y : \m{rec}\;\bar{v}\; \CICCproj{T} . \CICCproj{ [ \m{get}\; \CICCproj{T}\; \bar{v}\; y  / v ] E}
\;\text{ where $y$ is fresh}
\] 

\[
\CICCproj{\Pi v : T . E } := \Pi v : \CICCproj{T} . \CICCproj{E}
\]

\[
\CICCproj{?\Pi v : T . E } := \Pi y : \m{rec}\;\bar{v}\;\CICCproj{T} . \CICCproj{ [ \m{get}\;\CICCproj{T}\; \bar{v}\; y  / v ] E}
\;\text{ where $y$ is fresh}
\]

\label{cicc:proj}
\end{definition}

It is significant that church numerals be used for the representation of the name in the record, 
as no extra axioms need to be included in the context of the translation for the translation to be valid.  
This necessity is seen in \ref{ci:sound}.

\begin{lemma}

If $\Gamma \vdash_{ci} A : T$ then $\Gamma \vdash_{ci}$

\label{ci:wfctxt}
\end{lemma}

\begin{lemma}

If $\Gamma \vdash_{ci} A : T$ then $\Gamma \vdash_{ci} T : s$ for some sort $s$

\label{ci:wtt}
\end{lemma}

\begin{theorem}

\textbf{(Soundness of extraction)}  

\begin{alignat}{4}
\Gamma &\vdash_{ci}&  & \implies & \CICCproj{\Gamma} & \vdash_{cc} &
\\
\Gamma &\vdash_{ci}& A : T & \implies & \CICCproj{\Gamma} & \vdash_{cc} & \CICCproj{A} : \CICCproj{ T }
\end{alignat}

\label{ci:sound}
\end{theorem}

It is easy to see how this proof follows from the 
definition \ref{cicc:proj}, and the two lemmas \ref{ci:wfctxt} and \ref{ci:wtt}.

\begin{definition}
$ \m{Term}_{ci}  = \{ M : \exists T,\Gamma . \Gamma \vdash_{ci} M : T \}$
\end{definition}

\begin{theorem}
\textbf{(Consistency)}  $\not \exists M \in \m{Term}_{ci}. \vdash M : \Pi x : \m{prop} . x$
\label{ci:cons}
\end{theorem}

This follows clearly from the soundness of the extraction, 
\ref{ci:sound}, the consistency of $CC$ \ref{cc:cons}, and the fact that we 
have obviously more possible reductions in $CC$ than in $CICC$.  

Before we can prove the strong normalization theorem, we need to show that reductions in the extracted calculus 
correspond to reductions in $CICC$. 

\begin{theorem}
\textbf{(Semantic Equivalence)} 
$\forall M \in \m{Term}_{cicc}$ such that $\CICCproj{M} \rightarrow_{\beta\eta*} N'$
implies 
$\exists M' \in \m{Term}_{cicc}$ such that $M \rightarrow_{\beta\eta\alpha} M'$ and $\CICCproj{M'} \equiv N'$
\label{ci:se}
\end{theorem}

While the proof here is somewhat technical, the result is obvious, as the extraction and calculus
have been written so as to ensure that the result is true.

\begin{theorem}
\textbf{(Subject Reduction)} If $\Gamma \vdash_{ci} M : T$ and $M \rightarrow_{\beta\eta\alpha*} M'$ then $\Gamma \vdash_{ci} M' : T$
\end{theorem}

In non dependent type theories, this theorem is rather trivial.  However, in $CC$, this theorem depends heavily on the Church-Rosser 
theorem.  Fortunately, $CICC$ can be considered a bicoloring of the syntax of $CC$, and the proof of subject reduction 
for $CC$ can be leveraged to $CICC$ using a forgetfull extraction.

\begin{theorem}
\textbf{(Strong Normalization)} $\forall M \in \m{Term}_{ci}. SN(M)$
\label{ci:sn}
\end{theorem}

That we can cleanly translate into the calculus of constructions without loss or gain of 
semantical translation implies strong normalization for $CICC$ with $\beta\eta$ equivalence, 
and $\eta$ expansion.  This is the most important theorem of the section, as it implies that typchecking a Caledon 
statement will allow that statement to be compiled to pattern form to be used in proof search.  
Without this theorem typechecking is a weak property and valid programs might not be useable.  
While type safety in the presence of the strong normalization theorem does not ensure bounded proof search, 
it does ensure bounded unification.
The strong normalization theorem is to logic programming languages as 
the progress theorem is to functional programming languages.

That $CICC$ is simply an extention of $CC$ and not a modification of $CC$ implies we have the completeness theorem, \ref{ci:comp}.

\begin{theorem}
\textbf{(Completeness)}  $\forall M,T \in \m{Term}_{cc}. \vdash_{cc} M : T \implies \vdash_{ci} M : T$
\label{ci:comp}
\end{theorem}

This theorem is trivial since the syntax of $CC$ is a subset of the syntax of $CICC$.

\section{Elaboration}

In order to make use of the implicit system of $CICC$, an inference
relation must be provided.  
This is accomplished by extending the typing rules and providing
a mapping from the extended type derivation and term to 
an original type derivation and term. 

We only have one syntactic difference in this calculus:  $E\; \{ V : A = E \}$ 
is now simply $ E \; \{ V  = E \}$.  
We might also include Curry style binders in this presentation, but they shed little 
extra light on the workings of type inference.

\newcommand{\judgeCI}[0]{ \vdash_{i^-}}

Let $\Gamma \vdash A : T \wedge B : T'$ stand for $\Gamma \vdash A : T$ and $\Gamma \vdash B : T'$.
 
\begin{definition}
\textbf{($CICC^-$ Extended Typing Rules)}

%% inst/f %%
%%%%%%%%%%%%
\[
\infer[\m{inst/f}]
{
\Gamma \judgeCI M : [N/x]U 
}
{
\Gamma \judgeCI M : ?\Pi x :T . U
&
\Gamma \judgeCI N : T
&
x \notin DV(\Gamma)
}
\]

%% abs2 %%
%%%%%%%%%%
\[
\infer[\m{abs/f}]
{
\Gamma\judgeCI M : ?\Pi x : T . U
}
{
\Gamma, x : T\judgeCI M : [N/x]U \wedge N : T
&
\Gamma \judgeCI (?\Pi x : T . U) : K
&
x \notin FV(M) \cup DV(\Gamma)
}
\]

%% strength %%
%%%%%%%%%%%%%%
\[
\infer[\m{strength}]
{
\Gamma\judgeCI M : U
}
{
\Gamma, x : T  \judgeCI M : U
&
x \notin FV(M) \cup FV(U)  \cup DV(\Gamma)
}
\]

%% inst/b %%
%%%%%%%%%%%%
\[
\infer[\m{inst/b}]
{
\Gamma \judgeCI M \{ x = N \} : [N/x]U 
}
{
\Gamma \judgeCI M : ?\Pi x : T . U
&
\Gamma \judgeCI N : T
& 
x \notin GN(M)
&
x \notin BN(U)
}
\]

\end{definition}

In $CICC$, as in $CC$, the strengthening rule is admissible,
while in $CICC^{-}$, it is not.  

Note that the rule $\m{abs/f}$ might appear to not make sense at first
glance since it abstracts to a known term, but it can be considered
equivalent to an existential pack without the pack proof term, since
$x \notin FV(M)$  

Further note that conversion is now restricted to $\beta$ conversion in order to 
allow for the Church Rosser theorem which is necessary to prove subject reduction.

While we no longer care about the semantics of this language since we will be
elaborating to the sublanguage $CICC$ before evaluating and type checking further, we do not need to 
semantic related properties.  

However, it is still important to note that substitution holds.

\begin{theorem}
\textbf{(Substitution)}
\[
\infer-[\m{subst}]{ 
\Gamma, [N/x]\Gamma' \judgeCI [N / x]M : [N/x]T_2
}{
\Gamma, x : T_1, \Gamma' \judgeCI M : T_2
&
\Gamma \judgeCI N : T_1
}
\]
\label{ci:sub}
\end{theorem}

\begin{theorem}
\textbf{(Subject Reduction)} If $\Gamma \judgeCI M : T$ and $M \rightarrow_{\beta*} M'$ then $\Gamma \judgeCI M' : T$

\label{ci:sr}
\end{theorem}

\ref{ci:sr} is at the moment beleived to be true, 
although no full formalization of them exists.  Provided reductions are restricted to $\beta$ conversion, the church rosser 
theorem is simply provable and the proof of subject reduction is similar to that in the traditional calculus of constructions.

Without the $\m{abs/f}$ rule, subject reduction becomes unnecessary for the metatheory since 
the single direction subtyping relation is sufficient.  However, unification becomes difficult to implement.  

Unfortunately, the projection function now requires more information than is available syntactically, 
and thus must be given on the typing derivation.

\begin{definition}
\textbf{ (Projection from $CICC^{-}$ to $CICC$) }

\newcommand{\CICCmproj}[1]{ \left\llbracket #1 \right\rrbracket_{ci^{-}}}

\[
\CICCmproj{
\infer[\m{wf/e}]
{
\cdot \judgeCI 
}{}
}^{c}
:= \cdot
\]

\[
\CICCmproj{
\infer[\m{wf/s}]
{
\Gamma, x : T \judgeCI 
}
{
\overset{\mathcal{D}}{ 
\Gamma \judgeCI x : T 
}
&
\cdots
}
}^{c}
:= \CICCmproj{\Gamma \judgeCI}^c, \CICCmproj{\mathcal{D}} 
\]

\[
\CICCmproj{
\infer[\m{start}]
{
\Gamma,x:A \judgeCI x :A
}
{
\cdots
}
}
:= x
\]


\[
\CICCmproj{
\infer[\m{axioms}]
{
\Gamma,x:A \judgeCI c : s
}
{
\cdots
}
}
:= c
\]

%% prod %%
%%%%%%%%%%
\[
\CICCmproj{
\infer[\m{prod}]{ \Gamma \judgeCI (\Pi x : T . U) : s 
}{ 
\overset{\mathcal{D}_1}{ 
\Gamma \vdash T : s_1
}
&
\overset{\mathcal{D}_2}{ 
\Gamma,x:T \vdash U : s_2
}
&
\cdots
}
}
:=
\Pi x : \CICCmproj{ \mathcal{D}_1 }  . \CICCmproj{ \mathcal{D}_2 }
\]

%% prod* %%
%%%%%%%%$%%
\[
\CICCmproj{
\infer[\m{prod}*]{ \Gamma \judgeCI (?\Pi x : T . U) : s 
}{ 
\overset{\mathcal{D}_1}{ 
\Gamma \vdash T : s_1
}
&
\overset{\mathcal{D}_2}{ 
\Gamma,x:T \vdash U : s_2
}
&
\cdots
}
}
:=
?\Pi x : \CICCmproj{ \mathcal{D}_1 }  . \CICCmproj{ \mathcal{D}_2 }
\]

%% gen %%
%%%%%%%%%
\[
\CICCmproj{
\infer[\m{gen}]
{
\Gamma \judgeCI \lambda x : T . M : (\Pi x : T . U)
}
{
\overset{\mathcal{D}_1}{
\Gamma , x : T \judgeCI M : U 
}
&
\infer[\m{prod}]{ \Gamma \judgeCI (\Pi x : T . U) : s 
}{ 
\overset{\mathcal{D}_2}{ 
\Gamma \vdash T : s_1
}
&
\overset{\mathcal{D}_3}{ 
\Gamma,x:T \vdash U : s_2
}
&
\cdots
}
&
\cdots
}
}
:=
\lambda x : \CICCmproj{ \mathcal{D}_2 }  . \CICCmproj{ \mathcal{D}_1 }
\]

%% gen* %%
%%%%%%%%%%
\[
\CICCmproj{
\infer[\m{gen}*]
{
\Gamma \judgeCI ?\lambda x : T . M : (?\Pi x : T . U)
}
{
\overset{\mathcal{D}_1}{
\Gamma , x : T \judgeCI M : U 
}
&
\infer[\m{prod}*]{ \Gamma \judgeCI (?\Pi x : T . U) : s 
}{ 
\overset{\mathcal{D}_2}{ 
\Gamma \vdash T : s_1
}
&
\overset{\mathcal{D}_3}{ 
\Gamma,x:T \vdash U : s_2
}
&
\cdots
}
&
\cdots
}
}
:=
?\lambda x : \CICCmproj{ \mathcal{D}_2 }  . \CICCmproj{ \mathcal{D}_1 }
\]

%% app %%
%%%%%%%%%
\[
\CICCmproj{ 
\infer[\m{app}]
{
\Gamma \judgeCI M N : U [N/x]
}
{
\overset{\mathcal{D}_1}{ \Gamma \judgeCI M : \Pi x : T . U }
&
\overset{\mathcal{D}_2}{ \Gamma \judgeCI N : T }
}
}
:=
\CICCmproj{ \mathcal{D}_1 } \; \CICCmproj{\mathcal{D}_2}
\]

%% inst/b %%
%%%%%%%%%%%%
\[
\CICCmproj{ 
\infer[\m{inst/b}]
{
\Gamma \judgeCI M \{ x = N \} : U [N/x]
}
{
\overset{\mathcal{D}_1}{ \Gamma \judgeCI M : ?\Pi x :T . U }
&
\overset{\mathcal{D}_2}{ \Gamma \judgeCI N : T }
& 
\cdots
}
}
:=
\CICCmproj{\mathcal{D}_1} \; \{ x : \CICCmproj{\Gamma \vdash T : \m{kind}} = \CICCmproj{\mathcal{D}_2} \}
\]

%% inst/b %%
%%%%%%%%%%%%
\[
\CICCmproj{ 
\infer[\m{inst/f}]
{
\Gamma \judgeCI M : U [N/x]
}
{
\overset{\mathcal{D}_1}{ \Gamma \judgeCI M : ?\Pi x : T . U }
&
\overset{\mathcal{D}_2}{ \Gamma \judgeCI N : T }
&
\cdots
}
}
:=
\CICCmproj{\mathcal{D}_1} \; \{ x = \CICCmproj{\mathcal{D}_2} \}
\]

%% strength %%
%%%%%%%%%%%%%%
\[
\CICCmproj{ 
\infer[\m{strength}]
{
\Gamma \judgeCI M : U
}
{
\overset{\mathcal{D}}{ \Gamma, x : T \judgeCI M : U }
&
\cdots
}
}
:=
\CICCmproj{\mathcal{D}}
\]


%% abs/f %%
%%%%%%%%%%%
\[
\CICCmproj{ 
\infer[\m{abs/f}]
{
\Gamma \judgeCI M : ?\Pi x : T . U
}
{
\overset{\mathcal{D}}{ \Gamma \judgeCI M : [N/x]U }
&
\cdots
}
}
:=
?\lambda x : T . \CICCmproj{\mathcal{D}}
\]
\label{cicc-:proj}
\end{definition}


\begin{theorem}

\textbf{(Soundness of extraction)}  

\begin{alignat}{4}
\Gamma &\judgeCI &  & \implies & \CICCproj{\Gamma \judgeCI}^c & \judgeCI &
\\
\Gamma &\judgeCI & A : T & \implies & \CICCproj{\Gamma \judgeCI}^c & \judgeCI & \CICCproj{ \Gamma \judgeCI A : T }
\end{alignat}

\label{cicc-:sound}
\end{theorem}

Since $CICC$ permits $\eta$ equivalence and $CICC^-$ does not, the extraction in the 
reverse direction is no longer sound.  For our purposes this is not objectionable 
since $CICC$ is known to be consistent and there is no reason to convert back into $CICC^-$, as it is 
used entirely as a pre-elaboration language.  Once terms are typechecked and type infered in $CICC^-$
they are typechecked in $CICC$ and normalized in $CICC$.  While the reverse extraction is
in general not sound, normal terms with normal types are clearly typable in $CICC^-$.

%%%%%%%%%%%%%%%%%%%%%%%%%%%%%%%%%%%%%%%%%%%%%%%%%%%%%%%%%%%%%
%%% Subtyping %%%%%%%%%%%%%%%%%%%%%%%%%%%%%%%%%%%%%%%%%%%%%%%
%%%%%%%%%%%%%%%%%%%%%%%%%%%%%%%%%%%%%%%%%%%%%%%%%%%%%%%%%%%%%
\subsection{Subtyping}

Similar to $ICC$, these rules result in a subtyping relation, which will be of
importance during type inference and elaboration.

\begin{definition}
Subtyping relation:
$\Gamma \judgeCI T \leq T' \;\; \equiv \;\; \Gamma, x : T \judgeCI x : T'$  where $x$ is new.
\end{definition}

\begin{lemma}
Subtyping is a preordering:
\[
\begin{array}{lr}
\infer-[\m{refl}]{ 
\Gamma \judgeCI T \leq T
}{
\Gamma \judgeCI T : s
}
&
\infer-[\m{trans}]{ 
\Gamma \judgeCI T_1 \leq T_3
}{
\Gamma \judgeCI T_1 \leq T_2
&
\Gamma \judgeCI T_2 \leq T_3
}
\end{array}
\]

\[
\infer-[\m{sub}]{ 
\Gamma \judgeCI M : T'
}{
\Gamma \judgeCI M \leq T
&
\Gamma \judgeCI T \leq T'
}
\]
\end{lemma}

This theorem is an application of the substitution lemma.

\begin{lemma}
Domains of products are contravariant and codomains are covarient:

\[
\begin{array}{lr}
\infer-[]{ 
\Gamma \judgeCI \Pi x : T . U \leq \Pi x : T' . U'
}{
\Gamma \judgeCI T' \leq T 
&
\Gamma,x : T' \judgeCI U \leq U'
}
&
\infer-[]{ 
\Gamma \judgeCI \forall x : T . U \leq \forall x : T' . U'
}{
\Gamma \judgeCI T' \leq T 
&
\Gamma,x : T' \judgeCI U \leq U'
}
\end{array}
\]
\end{lemma}

Unlike traditional subtyping relations where an explicit subtyping rule must be included in the type system,
this system's subtyping relation is much easier to manage during unification, as it is simply
a macro for a provability relation.  

This allows one to implement higher order unification almost exactly
as is usual without to much modification as would be the case in a lattice unification algorithm.  
Instead, the modification is made to the search procedure, and subtyping constraints 
are realized as search terms.  

However, with the addition of the strengthening rule, 
this kind of modification not entirely necessary, 
as it is provable that this subtyping relation is symetric \ref{ci:sym}, 
and thus an entirely symetric unification algorithm can be presented.

Theorem \ref{ci:sym} isn't exactly obvious upon first glance, 
so I will provide intuitive justification first.

In $CICC$, by uniqueness of types, 
$\Gamma \vdash x : A$ and 
$\Gamma \vdash x : B$ implies
$A \equiv_{\beta\eta*} B$.  
In $CICC^{-}$ however, there is no such uniqueness of 
types properties.  
Rather, the $\m{inst/f}$ and $\m{abs/f}$ 
rules permit you to repsectively,
add an initialize an implicit argument, 
abstract implicitely upon an unused argument. 

Thus if $\Gamma , x : ?\Pi z : T . A \judgeCI x : A$
by implicit instantiation of the argument $z:T$,
we might also
derive
$\Gamma , x : A \judgeCI x : ?\Pi z : T . A$
given that $z \notin FV(x)$ and that 
$\Gamma , x : ?\Pi z : T . A \judgeCI x : A$ 
implies $ \Gamma , x : ?\Pi z : T . A \judgeCI$ 
which implies $ \Gamma\judgeCI x : (?\Pi z : T . A) : K$.

\begin{theorem}
\textbf{(Symmetry)}
$\Gamma \judgeCI A \leq B $ implies 
$\Gamma \judgeCI B' \leq A' $. where $A \equiv_{\beta} A'$ and $B \equiv_{\beta} B'$
\label{ci:sym}
\end{theorem}

\begin{proof}

This is proved by induction on the structure of the proof
of $\Gamma, x : A \judgeCI x : B$.  Here I only consider 
the cases relevant to the new fragment.

\setcounter{tcase}{0}

%%%%%%% STRENGTH %%%%%%%%%
\begin{tcase}
We begin with the non admissible strengthening rule.
\end{tcase}

\begin{prooftree}
\AxiomC{$\Gamma, x : A, z : T \judgeCI x : B$}
\AxiomC{$z \notin FV(x)\cup FV(B) \cup DV(G)$}
\RightLabel{strength}
\BinaryInfC{$\Gamma, x : A \judgeCI x : B $}
\end{prooftree}

from this we can derive via the induction hypothesis, $\Gamma, x : B', z : T \judgeCI x : A'$ 
and then reapply strengthening.

\begin{prooftree}
\AxiomC{$\Gamma, x : B', z : T \judgeCI x : A'$}
\AxiomC{$z \notin FV(x)\cup FV(A') \cup DV(G)$}
\RightLabel{strength}
\BinaryInfC{$\Gamma, x : B' \judgeCI x : A' $}
\end{prooftree}

%%%%%%% INST %%%%%%%%%
\begin{tcase}
In this case we cover implicit instantiation.
\end{tcase}

\begin{prooftree}
\AxiomC{$\Gamma, x : A \judgeCI x : ?\Pi z :T . B$}
\AxiomC{$\Gamma, x : A \judgeCI N : T$}
\AxiomC{$z \notin DV(\Gamma, x:A)$}
\RightLabel{inst/f}
\TrinaryInfC{$\Gamma,x : A \judgeCI x : [N/z]B $}
\end{prooftree}

Suppose $z$ is not in $FV(B)$ then $[N/z]B \equiv B$.   
From this the following proof is possible:

The first steps are the following few derivations:

\begin{prooftree}
  \AxiomC{$\Gamma,\judgeCI B' : K $} 
  \AxiomC{$z \notin FV(B')\cup FV(K) $} 
\RightLabel{streng}
\BinaryInfC{$\Gamma,z:T',  \judgeCI B' : K $} 
\end{prooftree}

\begin{prooftree}
  \AxiomC{$\Gamma \judgeCI T': K' $} 
  \AxiomC{$\Gamma,z:T' \judgeCI B' : K $} 

\RightLabel{form/f}
\BinaryInfC{$\Gamma \judgeCI ?\Pi z : T' . B' : K$}
\end{prooftree}

\begin{prooftree}
 \AxiomC{$B \equiv_{\beta}B'$}
 \AxiomC{$z \notin B'$}
\BinaryInfC{$z \notin FV(B')$}
\end{prooftree}

From these, we can derive the following result about $B'$.

\begin{prooftree}
    \AxiomC{$\Gamma,z:T' \judgeCI B' : K $} 

   \RightLabel{start}
   \UnaryInfC{$\Gamma,z:T', x : B' \judgeCI x : B'$}

   \AxiomC{$\Gamma \judgeCI ?\Pi z : T' . B' : K$}

   \AxiomC{$z \notin FV(x)\cup DV(\Gamma)$}
\RightLabel{abs/f}
\TrinaryInfC{$\Gamma, x : B' \judgeCI x : ?\Pi z : T' . B'$}
\end{prooftree}

Finally, we can derive the desired result:

\begin{prooftree}
   \AxiomC{$\Gamma, x : B' \judgeCI x : ?\Pi z : T' . B'$}

      \AxiomC{$IH(\Gamma , x : A \judgeCI x : ?\Pi z : T . B)$}
    \UnaryInfC{$\Gamma, x : ?\Pi z : T' . B' \judgeCI x : A'$}

  \RightLabel{subst}
  \BinaryInfC{$\Gamma, x : B' \judgeCI x : A'$}

  \AxiomC{$z \notin FV(B')$}
\BinaryInfC{$\Gamma, x : [N/z]B' \judgeCI x : A'$}
\end{prooftree}

Note that the only unexplained axioms here are 
$\Gamma \judgeCI B': K$ and $\Gamma \judgeCI T' : K$ in this proof.
Because $\Gamma, x : A' \judgeCI x : ?\Pi z:T' . B'$ is true, we know that 
$\Gamma, x : A' \judgeCI ?\Pi z : T' . B' : K$ and thus that 
$\Gamma,x : A' \judgeCI ?\Pi T' : K' $ and $\Gamma, x : A', z : T' \judgeCI B' : K$.  
By strengthening we can infer $\Gamma \judgeCI B': K$ and $\Gamma \judgeCI T' : K$.

%%%%%%%%%%%%%%%%%%%%%%%%%%%

On the other hand, if $z$ is in $FV(B)$ we achieve different proofs.  
Now we can infer that $x \notin FV(N)$, but we can not show that $[N/z]B' \equiv B'$.

By the induction hypothesis, we can infer $\Gamma , x : ?\Pi z : T' . B'\judgeCI x : A'$.

First, we know that $\Gamma,x : A \judgeCI [N/z]B : K$ by well formedness of the judgement 
$\Gamma,x : A \judgeCI x : [N/z]B' $ and the conversion rule.

\begin{prooftree}
\AxiomC{$\Gamma, x : A \judgeCI [N/z]B' : K$}
\AxiomC{$x \notin FV(B')$}
\RightLabel{strength}
\BinaryInfC{$\Gamma \judgeCI [N/z]B' : K$}
\RightLabel{start}
\UnaryInfC{$\Gamma, x : [N/z]B'\judgeCI x : [N/z]B'$}
\AxiomC{$z \notin FV(N)$}
\RightLabel{strength}
\BinaryInfC{$\Gamma, x : [N/z]B', z : T\judgeCI x : [N/z]B'$}
\end{prooftree}

thus, we can use the $\m{abs/f}$ rule to construct a form we can use in substitution.

\begin{prooftree}
\AxiomC{$\Gamma, x : [N/z]B', z : T \judgeCI (x : [N/z]B') \wedge N : T'$}
\AxiomC{$\Gamma, x : [N/z]B' \judgeCI ?\Pi z : T' . B' : K $}
\AxiomC{$z \notin FV(M) \cup DV(\Gamma)$}
\RightLabel{abs/f}
\TrinaryInfC{$\Gamma, x : [N/z]B' \judgeCI x : ?\Pi z : T' . B'$}
\end{prooftree}

Finally, we get the following derivation.

\begin{prooftree}
\AxiomC{$\Gamma, x : [N/z]B' \judgeCI x : ?\Pi z : T' . B'$}
\AxiomC{$\Gamma , x : ?\Pi z : T' . B'\judgeCI x : A'$}
\RightLabel{subst}
\BinaryInfC{$\Gamma , x : [N/z]B' \judgeCI x : A'$}
\end{prooftree}

%%%%%%% ABS %%%%%%%%%
\begin{tcase}
In this case we examine the $\m{abs/f}$ rule.
\end{tcase}

\begin{prooftree}
\AxiomC{$\Gamma, x : A, z : T \judgeCI x : [N/z]B \wedge N : T$}
\AxiomC{$\Gamma, x : A \judgeCI ?\Pi z : T' . B' : K $}
\AxiomC{$z \notin FV(M) \cup DV(\Gamma, x)$}
\RightLabel{abs/f}
\TrinaryInfC{$\Gamma, x : A \judgeCI x : ?\Pi z : T . B$}
\end{prooftree}

From this we can infer that $z \notin FV(A)$.  This is usefull since we can derive:
\[
\Gamma, z : T, x : A \judgeCI x : [N/z]B \wedge N : T
\]

We can then apply the induction hypothesis to get the following:

\[
\Gamma, z : T, x : [N'/z]B' \judgeCI x : A'
\]

From this we can infer $\Gamma, x : [N'/z]B' \judgeCI x : A'$
by strengthening since $z \notin FV(x)\cup FV(A')$.

Furthermore, we can infer that $\Gamma \judgeCI N' : T'$ since $N' \equiv_{\beta} N$ 
so $\Gamma \judgeCI N' : T$ by subject reduction and $T' \equiv_{\beta} T$ so $\Gamma \judgeCI N' : T'$
by conversion.

We can also derive $\Gamma, x : ?\Pi z : T' . B' \judgeCI x : ?\Pi z : T' . B'$ by the start rule. 

We get the following proof:

\begin{prooftree}
\AxiomC{$\Gamma, x : ?\Pi z : T' . B' \judgeCI x : ?\Pi z : T' . B'$}
\AxiomC{$\Gamma \judgeCI N' : T'$}
\AxiomC{$ z \notin DV(\Gamma)$}
\RightLabel{inf/f}
\TrinaryInfC{$\Gamma, x : ?\Pi z : T' . B' \judgeCI x : [N'/x] B'$}
\end{prooftree}

Finally, with the knowledge that $z\notin FV(A')$, we can derive the following:

\begin{prooftree}
\AxiomC{$\Gamma, x : ?\Pi z : T' . B' \judgeCI x : [N'/x] B'$}
\AxiomC{$\Gamma, x : [N'/x] B' \judgeCI x : A'$}
\RightLabel{subst}
\BinaryInfC{$\Gamma, x : ?\Pi z : T' . B' \judgeCI x : A'$}
\end{prooftree}

\end{proof}
 %% inference
\section{Semantics for $CICCI$}


\subsection{Substitution With Implicits}

The formulation of hereditary substitution in the presence of 
implicit arguments is not that unlike the presentation of
heredetary substitution without implicit arguments, 
but for additional checks required.
The main difficulty is the notion of an acceptable substitution. 
Because implicit bindings are not $\alpha$ convertable, 
certain substitutions are not permited.  
Because as many substitions should be permitted as possible, 
the situation becomes significantly more complex in the 
hereditary case, where substitutions might not carry types.  
The easiest way to define substitution in this case is with an ``illegal'' alpha substitution, 
which maps implicitly bound variables to fresh names, 
and produces a memory to map them back.

In this case, we can say that a substitution $[S/x] M$ is legal if 
$FV(S) \subseteq FV(\alpha_I^-1( [\alpha_I(S)/x] M) )$.

\begin{definition}
\textbf{(Implicit Typed Hereditary Substitution)}


\[
[S / x : A]^n_{\Gamma } (?\lambda y : B . N) := ?\lambda y:B . [S / x : A]^n_{\Gamma, y : B} N
\] 

\[
\eta^{-1}_{?\Pi x : A . B}(N) := ?\lambda x : A . N \; \{ x = \eta^{-1}_A(x) \}
\] since $N$ being typable by $?\Pi x $ means that $x$ can not appear free in $N$

\[
\m{H}_{\Gamma}(P \downarrow ?\Pi y : B_1 . B_2 , \{ v := N \} ) := P\; \{ v := N \} \downarrow [N/y : B_1]^n_{\Gamma}B_2
\]

\[
\m{H}_{\Gamma} ((?\lambda v : A_1 . N) \uparrow ?\Pi v : A_1 . A_2 , \{ v := P \}) 
:= [P/v]^n_{\Gamma \vdash v : A_1} N \uparrow A_2
\]

\[ 
\m{H}(?\lambda v : T . P \uparrow \_ , A) := ?\lambda v : T . \m{H}(P,A)
\]

\label{def:hered}
\end{definition}



\subsection{Unification With Implicits}

Now we can use the convenient fact that $\Gamma \vdash A  \leq B$ implies $\Gamma \vdash B \leq A$.

\section{Dynamics}

Unification lies at the heart of the semantics of the Caledon language.  
In this section we present the syntax of the unification problems, as well as a modified version of the algorithm 
presented in \citep{pfenning1991logic} and \citep{pfenning1991unification} suited for implicit argument search.

Checking for the reducability of two full lambda terms has long been known to be only semidecidable.  
The matter becomes even more complicated when checking for the equality of terms with variables bound
by both existential and universal quantifiers.  Research from the past thirty years has constrained
the problem to a decidable subset known as the pattern fragment.

\subsection{Unification Terms}

\begin{definition}
Unification Terms:

\[
U ::= U \wedge U 
 \orr \forall V : T . U
 \orr \exists V : T . U 
 \orr T \doteq T
 \orr T \in T
 \orr \top
\]
\label{def:hou:syn}
\end{definition}

When $\doteq$ is taken to mean $\equiv_{\beta\eta\alpha*}$, the unification problem is to determine 
whether a statement $U$ is ``true'' in the common sense, and provide a proof of the truth of the statement. 

While we do not discuss the semantics of $T \in T$ or $T \in T >> T \in T$ here, 
one can refer to \citep{pfenning1991logic} for a complete presentation.

Unification problems of the form 
$\forall x : T_1 . \exists y : T_2 . U $ can be converted to the form
$\exists y : \Pi x : T_1 . T_2 . \forall x : T_1 . [y\; x / y ]U $ 
in the process known as raising. Unification
statements can always quantified over unused variables: $U \implies Q x : T . U$.  

Thus, statements can always be converted to the form
\[
\exists y_1 \cdots y_n . \forall x_1 \cdots x_k . S_1 \doteq V_1 \wedge \cdots S_r \doteq V_r
\]

\subsection{Typed Implicit Hereditary Substitution}

Before we can properly specify the semantics of this language, 
we must define typed hereditary substitution for this calculus.  This is important, as higher order unification for the calculus of constructions 
requires that terms maintain $\eta$-long normal form after substitutions have been performed.
Here, we do describe typed hereditary substituion in full, as a more complete presentation can be found in \citep{keller2010normalization}.

The formulation of hereditary substitution in the presence of 
implicit arguments is not that unlike the presentation of
hereditary substitution without implicit arguments \citep{miller1991uniform}, 
but for the additional checks required.
The main difficulty is the notion of an acceptable substitution. 
Because implicit bindings are not $\alpha$ convertible, 
certain substitutions are not permitted.  
Because as many substitions should be permitted as possible, 
the situation becomes significantly more complex in the 
hereditary case, where substitutions might not carry types.  
The easiest way to define substitution in this case is with an ``illegal'' alpha substitution, 
which maps implicitly bound variables to fresh names, 
and produces a memory to map them back. 


In this case, we can say that a substitution $[S/x] M$ is legal if 
$FV(S) \subseteq FV(\alpha_I^-1( [\alpha_I(S)/x] M) )$.

\begin{definition}
\textbf{(Implicit Typed Hereditary Substitution)}

\[
[S / x : A]^n_{\Gamma } (?\lambda y : B . N) := ?\lambda y:B . [S / x : A]^n_{\Gamma, y : B} N
\] 

\[
\eta^{-1}_{?\Pi x : A . B}(N) := ?\lambda x : A . N \; \{ x = \eta^{-1}_A(x) \}
\] since $N$ being typable by $?\Pi x $ means that $x$ can not appear free in $N$

\[
\m{H}_{\Gamma}(P \downarrow ?\Pi y : B_1 . B_2 , \{ v := N \} ) := P\; \{ v := N \} \downarrow [N/y : B_1]^n_{\Gamma}B_2
\]

\[
\m{H}_{\Gamma} ((?\lambda v : A_1 . N) \uparrow ?\Pi v : A_1 . A_2 , \{ v := P \}) 
:= [P/v]^n_{\Gamma \vdash v : A_1} N \uparrow A_2
\]

\[ 
\m{H}(?\lambda v : T . P \uparrow \_ , A) := ?\lambda v : T . \m{H}(P,A)
\]

\label{def:hered}
\end{definition}


\subsection{Unification Term Meaning}

We can provide an provability relation of a unification formula
based on the obvious logic.

\begin{definition}
$\Gamma \Vdash F $ can be interpreted as $\Gamma$ implies $F$ 
is provable.

\[ \begin{array}{lr}
\infer[\m{equiv}]{
\Gamma \Vdash M \doteq N
}{
\Gamma \vdash M : A
&
M \equiv_{\beta\eta\alpha*} N
&
\Gamma \vdash N : A
}
&
\infer[\m{true}]{
\Gamma \Vdash \top
}{}
\end{array} \]

\[
\infer[\m{conj}]{
\Gamma \Vdash F \wedge G
}{
\Gamma \Vdash F
&
\Gamma \Vdash G
}
\]

\[ \begin{array}{lr}
\infer[\m{exists}]{
\Gamma \Vdash \exists x : A . F
}{
\Gamma \Vdash [M/x] F
&
\Gamma \vdash M : A
}
&
\infer[\m{forall}]{
\Gamma \Vdash \forall x : A . F
}{
\Gamma, x : A \Vdash F
}
\end{array} \]

\label{def:hou:prf}
\end{definition}

While a truly superb logic programming language might 
be able to convert this very declarative 
specification into a runnable program, 
the essentially nondeterministic rule for existential
quantification in a unification formula prevents an 
obvious deterministic algorithm from being extracted.


\subsection{Higher Order Unification for CC}

\newcommand{\UnifiesTo}{\;\longrightarrow\;}

We now present an algorithm, similar to that presented in 
\citep{pfenning1991logic} for unification in the 
Calculus of Constructions.  Because we have already 
presented typed hereditary substitution with $\eta$-expansion, 
the presentation here will not add much 
but for types in the substitutions.  

$F \UnifiesTo F'$ shall mean that $F$ can be transformed to $F'$
without modifying the provability. 
An equation $F[G]$ will stand as notation for highlighting $G$
under the formulae context $F$.  
As an example, if we were to examine the formula 
$\forall x . \forall n . \exists y . ( y \doteq x \wedge \forall z . \exists r . [ x z \doteq r] )$
but were only interested in the last portion, we might instead write it as
$\forall x . F[\forall z . \exists r . [ x z \doteq r]]$
Again, $\phi$ shall be an injective partial permutation. 

Furthermore, rather than explicitly writing down the result of unification, 
we shall use $\exists x. F \UnifiesTo \exists x . [ L / x] F$ 
to stand for $\exists x. F \UnifiesTo \exists x . x \doteq L \wedge [ L / x] F$

The unification rules are symetric, so any rule of the form 
$M \doteq N \sim N \doteq M$ practically.

Also, for the purpose of typed normalizing heredetary substitution, 
a formula prefix $F[e]$ of the form 
$Qx_1:A_1 . E_1\wedge \cdots Qx_n : A_n . e$ shall be considered as a context
$x_1 : A_1 ,\cdots ,x_n : A_n$ when written $\nu^-1(F)$.

\setcounter{tcase}{0}

\begin{tcase}
Lam-Any
\end{tcase}

\[
F[\lambda x : A . M \doteq N]
\UnifiesTo
F[\forall x : A . M \doteq \m{H}_{\nu^-1(F),x:A}(N , x)]
\]

Because application is normalizing, this can cover the case where both $N$ is also a $\lambda$ 
abstraction.

\begin{tcase}
Lam-Lam
\end{tcase}

\[
\lambda x : A . M \doteq \lambda x : A . N
\UnifiesTo
\forall x : A . M \doteq N
\]

While this rule is not explicitly necessary as it is covered by the Lam-Any rule, 
when working
in a substitutive system with explicit names rather than DeBruijn indexes, 
this helps to reduce the number of substitutions from an original name. 

Lastly, these reductions make the 
assumption that no variable name is bound more than once.
This can seem restrictive, but it is possible to 
work past by alpha converting everywhere and annotating
new variables with their original names, and alpha converting
back to the original after unification. The other option is again
to use DeBruijn indexes.  DeBruijn indexes have their own drawbacks
here, as certain transformation such as ``Raising''
or the ``Gvar-Uvar'' rules involve insertion of multiple
variables into the context at an arbitrary point, 
which requiring the lifting of many variable names.  
It is possible to implement higher order unification
with DeBruijn indexes safetly and efficiently, 
but this is out of the scope of the thesis.

\begin{tcase}
Ilam-Ilam-Same
\end{tcase}

This case behaves just as the lam-lam case does, but only for implicit abstractions with the same names.

\begin{tcase}
Ilam-Other
\end{tcase}

if the constraint is of the form $F[(?\lambda x : A . M) A_1 \cdots A_n \doteq R]$, where $x$ is not constrained
in a prefix of $R$, we transition to 
\[
F[\exists x' : A .  x' \in A \wedge H(\cdots , H([x' / x : A]_F M, A_1), \cdots A_n) 
\]
\[
\doteq H(R, \{ x : A = x' \}) ]
\]

\begin{tcase}
Iforall-Iforall-Same
\end{tcase}

Because universal quantification is also subject to these subtyping rules, we do a similar thing to what 
we do in the case of implicit lambda abstraction: if the names match on both side, we unify.  

\begin{tcase}
Iforall-Other
\end{tcase}

This case is a bit different however, since in the constraint $F[?\Pi x : A . M \doteq R]$, $R$ is no longer an implicit
abstraction - rather it should be a type.  Here, we simply transition to 

\[
F[\exists x' : A .  x' \in A \wedge H(\cdots , H([x' / x : A]_F M, A_1), \cdots A_n) \doteq R ]
\]

\begin{tcase}
Uvar-Uvar
\end{tcase}

In the traditional case without implicit constraints, we have the following transition:

\[
F[\forall y : A . G[y M \doteq y N  ]]
\UnifiesTo
F[\forall y : A . G[ M_1 \doteq \wedge N_1 \cdots]]
\]

However, when implicit constraints are permitted, the same universally quantified variables might take different numbers of arguments.  
In this case, we must remove argument the unnecessary implicit constraints.  

We might then be presented with the following constraint: 
\[
F[\forall y : A . G[y M_1 \cdots M_m \doteq y N_1 \cdots N_n  ]]
\]
We define the following matching function $\uplus$


\[
\infer{
M_i M \uplus N_j N
\Rightarrow 
M_i \doteq N_i \wedge (M \uplus N)
}{
M_i \neq \{ a : A = B \}
&
N_i \neq \{ a : A = B \}
}
\]

\[
\{ a : A = M_i \} M \uplus \{ a : A' = N_j \} N
\Rightarrow
M_i \doteq N_i \wedge (M \uplus N)
\]


\[
\infer{
\{ a : A = M_i \} M \uplus N
\Rightarrow 
M \uplus N
}{
a \notin CN(N)
}
\]

Using this dropping match, we can define the transition as follows.

\[
\infer{ 
F[\forall y : A . G[y M_1 \cdots M_m \doteq y N_1 \cdots N_n  ]]
\UnifiesTo
F[\forall y : A . G[ Q  ]]
}{
M_1 \cdots M_m \uplus y N_1 \cdots N_n \Rightarrow Q
}
\]

The rest of the transitions are a bit mundane in comparison, mostly just ensuring they use the correct
quantifier when necessary.  Thus, we exclude the presentation of the Gvar-Gvar, and Gvar-Uvar-Inside and Gvar-Uvar-Outside cases.

\begin{tcase}
Forall-And
\end{tcase}

While in a good implementation, this case is not necessary, we take note of it here as it is a potential source of 
bugs when implementing such a language.  Unfortunately moving the universal quantifier to
capture a conjunction is not as simple, since
if done incorrectly, existential variables might be able
to be defined with respect to universal quantifiers that they
were not previously in the scope of.

\[
F[(\forall x : A . E_1) \wedge E_2]
\UnifiesTo
F[\forall x : A . E_1 \wedge E_2]
\]
provided no existential variables are declared in $E_2$.

While this restriction prevents most application of this rule, 
equations can still be flattened to the form
\[
Qx_1:A_1\cdots Qx_n : A_n . M_1 \doteq N_1 \wedge M_m \doteq N_n
\]

transforming $E_2$ first with the Raising rule untill 
an Exists-And transformation is possible, then repeating  
until $E_2$ no longer contains any existentially 
quantified variables.  This process is always terminating,
although potentially significantly slower.   

\subsection{Implementation}

Because typed substitution is necessary, we must now keep track of existential variable's
types.  This can significantly complicate the implementation of the unification algorithm
as the common technique of maintaining unbound existential variables with restrictions
can no longer be blindly used, as existential variables must be maintained in the 
formula.  The most advisable option is to maintain the type of the existential variable 
with each mention of the existential variable.  

After experimentation, good performance has been observed when this structure is implemented
as a zipper \citep{huet1997functional}. We have found exceptional performence when implementing this structure
as a zipper using a finger tree indexed lookups. Unfortunately, since variables are best 
implemented via DeBruijn indexes, general variable reconstruction is no longer trivial.  It is fortunate that general variable reconstruction 
is not necessary in the final implementation since an existential variable representing the body is always used at the top level.

Another option is to perform unification with untyped substitution.
While there is no proof at the moment that unification on the pattern subset of 
the calculus of constructions with untyped substitution for only the existential substitutions
is total, it is not unbelievable. Furthermore, omitting typed substitution does not alter
the correctness of the algorithm, only the potential totality.  

Ideally, knowledge that type checking terminated would be convincing enough
so it is not necessary to continue with the reconstruction.  However, reconstruction
is necessary for implementing the multi-pass proof search described previously.  
Furthermore, reconstruction is usefull since the exposed typing rules do not admit 
coherence.  In these cases, it is desirable to see what was infered by type inference.


\subsection{Program transformations}

\section{Results}

In this thesis I designed a logic programming language with a type system based on the
Calculus of Constructions which integrated the notion of an implicitly quantified type in
a manner useful for automating proof search. I demonstrated a series of reductions from
this language to the Calculus of Constructions where the output of the language could
be interpreted as meaningful theorems. I provided an abstract machine for the language
based on higher order unification with proof search, and demonstrated an elaboration
method to this machine. The semantics of the language based on this compilation and
evaluation joined the notions of type inference and traditional evaluation in a way that
does not appear to have been examined in great detail in the past.

I provided a method to constrain proof search of a predicate to a small subset of the
axioms in the environment using families. I demonstrated a way to explicitly control
whether a predicate was searched in a breadth first or depth first manner, allowing
constructs similar to fork and join to be defined.

I gave usage examples of usage of the Caledon language and demonstrated functionality
equivalent to type classes and ways to extend the applicability of this feature using library defined linearity checking.  

Finally, I provided an implementation of this language in Haskell and provided a
standard library. Since previous dependently typed logic languages have not included
polymorphism, standard libraries were not reasonable or possible to include. However,
with the addition of polymorphism, generic lists, type logic, printing, monad and
functor libraries become possible and useful.

\section{Future Work}

Although this thesis presented a language, little work was done to provide a framework
for proving theorems about this language. In general, compilation for a language
where the programs are theorems for a consistent logic allows for significant optimization
capability. In Twelf, totality, modes, and worlds allowed predicates to be converted
to programs. In general, running of Caledon programs in the current implementation
is excruciatingly slow, as types need to be recorded and searched during runtime. Algorithms
that take advantage of totality checking \citep{altenkirch2010termination}, 
uniqueness checking \citep{anderson2004verifying}, 
worlds checking\citep{anderson2004verifying}, 
mode checking\citep{anderson2004verifying}, 
and universe checking \citep{harper1991type}, 
could be implemented and applied as they were for Twelf and Agda.  It would be useful to have a type system for a logic programming
language which could ensure closed predicates were theorems. More work needs to be
done to automate typeclass instancing as was demonstrated in the section on Linearity.
While implemented, universe checking during unification has yet to be proven entirely
correct.

The possible applications of the language have only been shallowly addressed and
it is clear that much more interesting programs are possible. While derivatives of one
holed types are possible in the language, automatically providing traversals for these
zipper types is an unexplored topic. While I have demonstrated a concise method of
creating concurrency, libraries for controlling concurrency using the IO primitives have yet to be designed.

    
\appendix
\include{appendix} 

\backmatter

%\renewcommand{\baselinestretch}{1.0}\normalsize

% By default \bibsection is \chapter*, but we really want this to show
% up in the table of contents and pdf bookmarks.
\renewcommand{\bibsection}{\chapter{\bibname}}
%\newcommand{\bibpreamble}{This text goes between the ``Bibliography''
%  header and the actual list of references}
\bibliographystyle{plainnat}
\nocite{*}
\bibliography{../MyBib.bib} %your bib file

\end{document}
