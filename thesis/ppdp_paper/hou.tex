\section{Higher Order Unification}

Unification lies at the heart of the semantics of the Caledon language.  
In this section we present the syntax of the unification problems, as well as a modified version of the algorithm 
presented in \citep{pfenning1991logic} and \citep{pfenning1991unification} suited for implicit argument search.

Checking for the reducability of two full lambda terms has long been known to be only semidecidable.  
The matter becomes even more complicated when checking for the equality of terms with variables bound
by both existential and universal quantifiers.  Research from the past thirty years has constrained
the problem to a decidable subset known as the pattern fragment.

\subsection{Unification Terms}

\begin{definition}
Unification Terms:

\[
U ::= U \wedge U 
 \orr \forall V : T . U
 \orr \exists V : T . U 
 \orr T \doteq T
 \orr T \in T
 \orr T \in T >> T \in T
 \orr \top
\]
\label{def:hou:syn}
\end{definition}

When $\doteq$ is taken to mean $\equiv_{\beta\eta\alpha*}$, the unification problem is to determine 
whether a statement $U$ is ``true'' in the common sense, and provide a proof of the truth of the statement. 

While we do not discuss the semantics of $T \in T$ or $T \in T >> T \in T$ here, 
one can refer to \citep{pfenning1991logic} for a complete presentation.

Unification problems of the form 
$\forall x : T_1 . \exists y : T_2 . U $ can be converted to the form
$\exists y : \Pi x : T_1 . T_2 . \forall x : T_1 . [y\; x / y ]U $ 
in the process known as raising. Unification
statements can always quantified over unused variables: $U \implies Q x : T . U$.  

Thus, statements can always be converted to the form
\[
\exists y_1 \cdots y_n . \forall x_1 \cdots x_k . S_1 \doteq V_1 \wedge \cdots S_r \doteq V_r
\]

\subsection{Typed Implicit Hereditary Substitution}

Before we can properly specify the semantics of this language, 
we must define typed hereditary substitution for this calculus.  This is important, as higher order unification for the calculus of constructions 
requires that terms maintain $\eta$-long normal form after substitutions have been performed.
Here, we do describe typed hereditary substituion in full, as a more complete presentation can be found in \citep{keller2010normalization}.

The formulation of hereditary substitution in the presence of 
implicit arguments is not that unlike the presentation of
hereditary substitution without implicit arguments \citep{miller1991uniform}, 
but for the additional checks required.
The main difficulty is the notion of an acceptable substitution. 
Because implicit bindings are not $\alpha$ convertible, 
certain substitutions are not permitted.  
Because as many substitions should be permitted as possible, 
the situation becomes significantly more complex in the 
hereditary case, where substitutions might not carry types.  
The easiest way to define substitution in this case is with an ``illegal'' alpha substitution, 
which maps implicitly bound variables to fresh names, 
and produces a memory to map them back. 


In this case, we can say that a substitution $[S/x] M$ is legal if 
$FV(S) \subseteq FV(\alpha_I^-1( [\alpha_I(S)/x] M) )$.

\begin{definition}
\textbf{(Implicit Typed Hereditary Substitution)}

\[
[S / x : A]^n_{\Gamma } (?\lambda y : B . N) := ?\lambda y:B . [S / x : A]^n_{\Gamma, y : B} N
\] 

\[
\eta^{-1}_{?\Pi x : A . B}(N) := ?\lambda x : A . N \; \{ x = \eta^{-1}_A(x) \}
\] since $N$ being typable by $?\Pi x $ means that $x$ can not appear free in $N$

\[
\m{H}_{\Gamma}(P \downarrow ?\Pi y : B_1 . B_2 , \{ v := N \} ) := P\; \{ v := N \} \downarrow [N/y : B_1]^n_{\Gamma}B_2
\]

\[
\m{H}_{\Gamma} ((?\lambda v : A_1 . N) \uparrow ?\Pi v : A_1 . A_2 , \{ v := P \}) 
:= [P/v]^n_{\Gamma \vdash v : A_1} N \uparrow A_2
\]

\[ 
\m{H}(?\lambda v : T . P \uparrow \_ , A) := ?\lambda v : T . \m{H}(P,A)
\]

\label{def:hered}
\end{definition}


\subsection{Unification Term Meaning}

We can provide an provability relation of a unification formula
based on the obvious logic.

\begin{definition}
$\Gamma \Vdash F $ can be interpreted as $\Gamma$ implies $F$ 
is provable.

\[ \begin{array}{lcr}
\infer[\m{equiv}]{
\Gamma \Vdash M \doteq N
}{
\Gamma \vdash M : A
&
M \equiv_{\beta\eta\alpha*} N
&
\Gamma \vdash N : A
}
&
\infer[\m{true}]{
\Gamma \Vdash \top
}{}
&
\infer[\m{conj}]{
\Gamma \Vdash F \wedge G
}{
\Gamma \Vdash F
&
\Gamma \Vdash G
}
\end{array} \]

\[ \begin{array}{lr}
\infer[\m{exists}]{
\Gamma \Vdash \exists x : A . F
}{
\Gamma \Vdash [M/x] F
&
\Gamma \vdash M : A
}
&
\infer[\m{forall}]{
\Gamma \Vdash \forall x : A . F
}{
\Gamma, x : A \Vdash F
}
\end{array} \]

\label{def:hou:prf}
\end{definition}

While a truly superb logic programming language might 
be able to convert this very declarative 
specification into a runnable program, 
the essentially nondeterministic rule for existential
quantification in a unification formula prevents an 
obvious deterministic algorithm from being extracted.


\subsection{Higher Order Unification for CC}

\newcommand{\UnifiesTo}{\;\longrightarrow\;}

We now present an algorithm, similar to that presented in 
\citep{pfenning1991logic} for unification in the 
Calculus of Constructions.  Because we have already 
presented typed hereditary substitution with $\eta$-expansion, 
the presentation here will not add much 
but for types in the substitutions.  

$F \UnifiesTo F'$ shall mean that $F$ can be transformed to $F'$
without modifying the provability. 
An equation $F[G]$ will stand as notation for highlighting $G$
under the formulae context $F$.  
As an example, if we were to examine the formula 
$\forall x . \forall n . \exists y . ( y \doteq x \wedge \forall z . \exists r . [ x z \doteq r] )$
but were only interested in the last portion, we might instead write it as
$\forall x . F[\forall z . \exists r . [ x z \doteq r]]$
Again, $\phi$ shall be an injective partial permutation. 

Furthermore, rather than explicitly writing down the result of unification, 
we shall use $\exists x. F \UnifiesTo \exists x . [ L / x] F$ 
to stand for $\exists x. F \UnifiesTo \exists x . x \doteq L \wedge [ L / x] F$

The unification rules are symetric, so any rule of the form 
$M \doteq N \sim N \doteq M$ practically.

Also, for the purpose of typed normalizing heredetary substitution, 
a formula prefix $F[e]$ of the form 
$Qx_1:A_1 . E_1\wedge \cdots Qx_n : A_n . e$ shall be considered as a context
$x_1 : A_1 ,\cdots ,x_n : A_n$ when written $\nu^-1(F)$.

\setcounter{tcase}{0}

\begin{tcase}
Lam-Any
\end{tcase}

\[
F[\lambda x : A . M \doteq N]
\UnifiesTo
F[\forall x : A . M \doteq \m{H}_{\nu^-1(F),x:A}(N , x)]
\]

Because application is normalizing, this can cover the case where both $N$ is also a $\lambda$ 
abstraction.

\begin{tcase}
Lam-Lam
\end{tcase}

\[
\lambda x : A . M \doteq \lambda x : A . N
\UnifiesTo
\forall x : A . M \doteq N
\]

While this rule is not explicitly necessary as it is covered by the Lam-Any rule, 
when working
in a substitutive system with explicit names rather than DeBruijn indexes, 
this helps to reduce the number of substitutions from an original name. 

Lastly, these reductions make the 
assumption that no variable name is bound more than once.
This can seem restrictive, but it is possible to 
work past by alpha converting everywhere and annotating
new variables with their original names, and alpha converting
back to the original after unification. The other option is again
to use DeBruijn indexes.  DeBruijn indexes have their own drawbacks
here, as certain transformation such as ``Raising''
or the ``Gvar-Uvar'' rules involve insertion of multiple
variables into the context at an arbitrary point, 
which requiring the lifting of many variable names.  
It is possible to implement higher order unification
with DeBruijn indexes safetly and efficiently, 
but this is out of the scope of the thesis.

\begin{tcase}
Ilam-Ilam-Same
\end{tcase}

This case behaves just as the lam-lam case does, but only for implicit abstractions with the same names.

\begin{tcase}
Ilam-Other
\end{tcase}

if the constraint is of the form $F[(?\lambda x : A . M) A_1 \cdots A_n \doteq R]$, where $x$ is not constrained
in a prefix of $R$, we transition to 
$F[\exists x' : A .  x' \in A \wedge H(\cdots , H([x' / x : A]_F M, A_1), \cdots A_n) \doteq H(R, \{ x : A = x' \}) ]$

\begin{tcase}
Iforall-Iforall-Same
\end{tcase}

Because universal quantification is also subject to these subtyping rules, we do a similar thing to what 
we do in the case of implicit lambda abstraction: if the names match on both side, we unify.  

\begin{tcase}
Iforall-Other
\end{tcase}

This case is a bit different however, since in the constraint $F[?\Pi x : A . M \doteq R]$, $R$ is no longer an implicit
abstraction - rather it should be a type.  Here, we simply transition to 
$F[\exists x' : A .  x' \in A \wedge H(\cdots , H([x' / x : A]_F M, A_1), \cdots A_n) \doteq R ]$

\begin{tcase}
Uvar-Uvar
\end{tcase}

In the traditional case without implicit constraints, we have the following transition:

\[
F[\forall y : A . G[y M_1 \cdots M_n \doteq y N_1 \cdots N_n  ]]
\UnifiesTo
F[\forall y : A . G[ M_1 \doteq \wedge N_1 \cdots \wedge M_1 \cdots N_n ]]
\]

However, when implicit constraints are permitted, the same universally quantified variables might take different numbers of arguments.  
In this case, we must remove argument the unnecessary implicit constraints.  

We might then be presented with the following constraint: 
\[
F[\forall y : A . G[y M_1 \cdots M_m \doteq y N_1 \cdots N_n  ]]
\]
We define the following matching function $\uplus$


\[
\infer*{
M_i M_{i+1} \cdots M_m \uplus N_j N_{j+1} \cdots N_n 
\Rightarrow 
M_i \doteq N_i \wedge (M_{i+1} \cdots M_m \uplus N_{j+1} \cdots N_n)
}{
M_i \neq \{ a : A = B \}
&
N_i \neq \{ a : A = B \}
}
\]

\[
\{ a : A = M_i \} M_{i+1} \cdots M_m \uplus \{ a : A' = N_j \} N_{j+1} \cdots N_n 
\Rightarrow
M_i \doteq N_i \wedge (M_{i+1} \cdots M_m \uplus N_{j+1} \cdots N_n)
\]


\[
\infer[]{
\{ a : A = M_i \} M_{i+1} \cdots M_m \uplus N_j \cdots N_n 
\Rightarrow 
M_{i+1} \cdots M_m \uplus N_{j+1} \cdots N_n
}{
a \notin \m{ConstraintsPrefix}(N_j \cdots N_n)
}
\]

Using this dropping match, we can define the transition as follows.

\[
\infer[]{ 
F[\forall y : A . G[y M_1 \cdots M_m \doteq y N_1 \cdots N_n  ]]
\UnifiesTo
F[\forall y : A . G[ Q  ]]
}{
M_1 \cdots M_m \uplus y N_1 \cdots N_n \Rightarrow Q
}
\]

The rest of the transitions are a bit mundane in comparison, mostly just ensuring they use the correct
quantifier when necessary.  Thus, we exclude the presentation of the Gvar-Gvar, and Gvar-Uvar-Inside and Gvar-Uvar-Outside cases.

\begin{tcase}
Forall-And
\end{tcase}

While in a good implementation, this case is not necessary, we take note of it here as it is a potential source of 
bugs when implementing such a language.  Unfortunately moving the universal quantifier to
capture a conjunction is not as simple, since
if done incorrectly, existential variables might be able
to be defined with respect to universal quantifiers that they
were not previously in the scope of.

\[
F[(\forall x : A . E_1) \wedge E_2]
\UnifiesTo
F[\forall x : A . E_1 \wedge E_2]
\]
provided no existential variables are declared in $E_2$.

While this restriction prevents most application of this rule, 
equations can still be flattened to the form
\[
Qx_1:A_1\cdots Qx_n : A_n . M_1 \doteq N_1 \wedge M_m \doteq N_n
\]

transforming $E_2$ first with the Raising rule untill 
an Exists-And transformation is possible, then repeating  
until $E_2$ no longer contains any existentially 
quantified variables.  This process is always terminating,
although potentially significantly slower.   


The following cases are based on unification 
of a formula of the form

\[
\Gamma[
\exists x : \Pi u_1 : A_1 \cdots \Pi u_n : A_n . A 
\\
F[\forall y_1 : A'_1 . G_1 
[ \cdots \forall y_p : A'_p . G_p 
[ x y_{\phi(1)} \cdots y_{\phi(n)} \doteq M ]
\cdots
]]]
\]


\subsection{Implementation}

Because typed substitution is necessary, we must now keep track of existential variable's
types.  This can significantly complicate the implementation of the unification algorithm
as the common technique of maintaining unbound existential variables with restrictions
can no longer be blindly used, as existential variables must be maintained in the 
formula.  The most advisable option is to maintain the type of the existential variable 
with each mention of the existential variable.  

After experimentation, good performance has been observed when this structure is implemented
as a zipper \citep{huet1997functional}. We have found exceptional performence when implementing this structure
as a zipper using a finger tree indexed lookups. Unfortunately, since variables are best 
implemented via DeBruijn indexes, general variable reconstruction is no longer trivial.  It is fortunate that general variable reconstruction 
is not necessary in the final implementation since an existential variable representing the body is always used at the top level.

Another option is to perform unification with untyped substitution.
While there is no proof at the moment that unification on the pattern subset of 
the calculus of constructions with untyped substitution for only the existential substitutions
is total, it is not unbelievable. Furthermore, omitting typed substitution does not alter
the correctness of the algorithm, only the potential totality.  

Ideally, knowledge that type checking terminated would be convincing enough
so it is not necessary to continue with the reconstruction.  However, reconstruction
is necessary for implementing the multi-pass proof search described previously.  
Furthermore, reconstruction is usefull since the exposed typing rules do not admit 
coherence.  In these cases, it is desirable to see what was infered by type inference.

