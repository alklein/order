\subsection{Logic Programming}

The Caledon language is, at the surface, a backtracking higher order logic programming language in the style of Twelf.  
Programs in the language constitute a set of axioms that look much like types in a language like Haskell.  The ``value'' of these axioms are fact 
their types. As seen in \ref{code:add}, the type of the axiom ``add'' is $\m{nat} \rightarrow \m{nat} \rightarrow \m{nat} \rightarrow \m{prop}$. 

\begin{figure}[H]
\begin{lstlisting}
defn nat  : prop 
  >| zero = nat
  >| succ = nat -> nat

defn add  : nat -> nat -> nat -> prop
  >| addZ = add zero A A
  >| addS = add (succ A) B (succ C) 
             <- add A B C
\end{lstlisting}
\caption{Addition in Caledon}
\label{code:add}
\end{figure}

One can read the logic programming definition as one might read the functional equivalent with pattern
matching, knowing that an intelligent compiler would be able to convert the first into the second.  

Higher order logic programming languages such as $\lambda$Prolog remove the tedium of redefining features such as $\alpha$ substitution.
is in fact possible to express as shown in \ref{code:lprolog}.

\begin{figure}[H]
\begin{lstlisting}
defn term : prop
  | lam = (term -> term) -> term
  | app = term -> term -> term
\end{lstlisting}
\caption{Higher order abstract syntax}
\label{code:lprolog}
\end{figure}

Fortunately, higher order functions need not be restricted to patterns.  Lisp style macros provide even more ways
to generalize code.  

A great example is the function application operator from Haskell.  
We can define this in Caledon as shown in \ref{code:macros}

\begin{figure}[H]
\begin{lstlisting}
fixity right 0 @
defn @ : {at bt:prop} (at -> bt) -> at -> bt
  as ?\ at bt . \ f . \ a . f a

\end{lstlisting}
\caption{Macros for expressive syntax}
\label{code:macros}
\end{figure}

In many cases, allowing these macros allows for significant simplification of syntax.
