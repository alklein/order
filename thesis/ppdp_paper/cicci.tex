\subsection{Inference Terms}

In order to make use of the implicit system of $CICC$, an inference
relation must be provided.  
This is accomplished by extending the typing rules and providing
a mapping from the extended type derivation and term to 
an original type derivation and term. 

The only syntactic difference in this calculus is that $E\; \{ V : A = E \}$ 
is now simply $ E \; \{ V  = E \}$.  
We might also include Curry style binders in this presentation, but they shed little 
extra light on the workings of type inference.  
The important difference comes from the ability to 
use an implicit abstraction without explicitly initializing it, for example: $(?\lambda x : \m{print}\;\m{zero}. \m{succ}) \m{zero}$

\newcommand{\judgeCI}[0]{ \vdash_{i^-}}

Let $\Gamma \vdash A : T \wedge B : T'$ stand for $\Gamma \vdash A : T$ and $\Gamma \vdash B : T'$.
 
\begin{definition}
\textbf{($CICC^-$ Extended Typing Rules)}

%% inst/f %%
%%%%%%%%%%%%
\[
\infer[\m{inst/f}]
{
\Gamma \judgeCI M : [N/x]U 
}
{
\Gamma \judgeCI M : ?\Pi x :T . U
&
\Gamma \judgeCI N : T
&
x \notin DV(\Gamma)
}
\]


%% strength %%
%%%%%%%%%%%%%%
\[
\infer[\m{strength}]
{
\Gamma\judgeCI M : U
}
{
\Gamma, x : T  \judgeCI M : U
&
x \notin FV(M) \cup FV(U)  \cup DV(\Gamma)
}
\]


%% abs2 %%
%%%%%%%%%%


\[
\infer
{
\Gamma\judgeCI M : ?\Pi x : T . U
}
{
\Gamma, x : T\judgeCI M : [N/x]U \wedge N : T
&
\Gamma \judgeCI (?\Pi x : T . U) : K
&
x \notin QV(M,\Gamma)
}
\]

where $QV(M,\Gamma) = FV(M) \cup DV(\Gamma)$

($\m{abs/f}$ rule)


%% inst/b %%
%%%%%%%%%%%%
\[
\infer 
{
\Gamma \judgeCI M \{ x = N \} : [N/x]U 
}
{
\Gamma \judgeCI M : ?\Pi x : T . U
&
\Gamma \judgeCI N : T
& 
x \notin GN(M) \cup BN(U)
}
\]

($\m{inst/b}$ rule)

\end{definition}

In $CICC$, as in $CC$, the strengthening rule is admissible,
while in $CICC^{-}$, it is not.  

Note that the rule $\m{abs/f}$ might appear to not make sense at first
glance since it abstracts to a known term, but it can be considered
equivalent to an existential pack without the pack proof term, since
$x \notin FV(M)$  

Further note that conversion is now restricted to $\beta$ conversion in order to 
allow for the Church Rosser theorem which is necessary to prove subject reduction.

While we no longer care about the semantics of this language since we will be
elaborating to the sublanguage $CICC$ before evaluating and type checking further, we do not need to 
semantic related properties.  

However, it is still important to note that substitution holds.

\begin{theorem}
\textbf{(Substitution)}
\[
\infer-[\m{subst}]{ 
\Gamma, [N/x]\Gamma' \judgeCI [N / x]M : [N/x]T_2
}{
\Gamma, x : T_1, \Gamma' \judgeCI M : T_2
&
\Gamma \judgeCI N : T_1
}
\]
\label{ci:sub}
\end{theorem}

\begin{theorem}
\textbf{(Subject Reduction)} If $\Gamma \judgeCI M : T$ and $M \rightarrow_{\beta*} M'$ then $\Gamma \judgeCI M' : T$

\label{ci:sr}
\end{theorem}

\ref{ci:sr} is at the moment beleived to be true, 
although no full formalization of them exists.  Provided reductions are restricted to $\beta$ conversion, the church rosser 
theorem is simply provable and the proof of subject reduction is similar to that in the traditional calculus of constructions.

Without the $\m{abs/f}$ rule, subject reduction becomes unnecessary for the metatheory since 
the single direction subtyping relation is sufficient.  However, unification becomes difficult to implement.  

Unfortunately, the projection function now requires more information than is available syntactically, 
and thus must be given on the typing derivation.

\begin{definition}
\textbf{ (Projection from $CICC^{-}$ to $CICC$) }

\newcommand{\CICCmproj}[1]{ \left\llbracket #1 \right\rrbracket_{ci^{-}}}

\[
\CICCmproj{
\infer[\m{wf/e}]
{
\cdot \judgeCI 
}{}
}^{c} 
:= \cdot
\]

\[
\CICCmproj{
\infer[\m{wf/s}]
{
\Gamma, x : T \judgeCI 
}
{
\overset{\mathcal{D}}{ 
\Gamma \judgeCI x : T 
}
&
\cdots
}
}^{c}
:= \CICCmproj{\Gamma \judgeCI}^c, \CICCmproj{\mathcal{D}} 
\]

\[
\CICCmproj{
\infer[\m{start}]
{
\Gamma,x:A \judgeCI x :A
}
{
\cdots
}
}
:= x
\]


\[
\CICCmproj{
\infer[\m{axioms}]
{
\Gamma,x:A \judgeCI c : s
}
{
\cdots
}
}
:= c
\]

%% prod %%
%%%%%%%%%%
\[
\CICCmproj{
\infer[\m{prod}]{ \Gamma \judgeCI (\Pi x : T . U) : s 
}{ 
\overset{\mathcal{D}_1}{ 
\Gamma \vdash T : s_1
}
&
\overset{\mathcal{D}_2}{ 
\Gamma,x:T \vdash U : s_2
}
&
\cdots
}
}
:=
\Pi x : \CICCmproj{ \mathcal{D}_1 }  . \CICCmproj{ \mathcal{D}_2 }
\]

%% prod* %%
%%%%%%%%$%%
\[
\CICCmproj{
\infer[\m{prod}*]{ \Gamma \judgeCI (?\Pi x : T . U) : s 
}{ 
\overset{\mathcal{D}_1}{ 
\Gamma \vdash T : s_1
}
&
\overset{\mathcal{D}_2}{ 
\Gamma,x:T \vdash U : s_2
}
&
\cdots
}
}
:=
?\Pi x : \CICCmproj{ \mathcal{D}_1 }  . \CICCmproj{ \mathcal{D}_2 }
\]

%% gen %%
%%%%%%%%%
\begin{gather}
\CICCmproj{
\infer[\m{gen}]
{
\Gamma \judgeCI \lambda x : T . M : (\Pi x : T . U)
}
{
\cdots
&
\overset{\mathcal{D}_1}{
\Gamma , x : T \judgeCI M : U 
}
&
\infer[\m{prod}]{ \Gamma \judgeCI (\Pi x : T . U) : s 
}{ 
\overset{\mathcal{D}_2}{ 
\Gamma \vdash T : s_1
}
&
\overset{\mathcal{D}_3}{ 
\Gamma,x:T \vdash U : s_2
}
&
\cdots
}
}
}
\\
:=
\lambda x : \CICCmproj{ \mathcal{D}_2 }  . \CICCmproj{ \mathcal{D}_1 }
\end{gather}

%% gen* %%
%%%%%%%%%%
\begin{gather}
\CICCmproj{
\infer[\m{gen}*]
{
\Gamma \judgeCI ?\lambda x : T . M : (?\Pi x : T . U)
}
{
\cdots
&
\overset{\mathcal{D}_1}{
\Gamma , x : T \judgeCI M : U 
}
&
\infer[\m{prod}*]{ \Gamma \judgeCI (?\Pi x : T . U) : s 
}{ 
\overset{\mathcal{D}_2}{ 
\Gamma \vdash T : s_1
}
&
\overset{\mathcal{D}_3}{ 
\Gamma,x:T \vdash U : s_2
}
&
\cdots
}
}
}
\\
:=
?\lambda x : \CICCmproj{ \mathcal{D}_2 }  . \CICCmproj{ \mathcal{D}_1 }
\end{gather}

%% app %%
%%%%%%%%%
\begin{gather}
\CICCmproj{ 
\infer[\m{app}]
{
\Gamma \judgeCI M N : U [N/x]
}
{
\overset{\mathcal{D}_1}{ \Gamma \judgeCI M : \Pi x : T . U }
&
\overset{\mathcal{D}_2}{ \Gamma \judgeCI N : T }
}
}
\\
:=
\CICCmproj{ \mathcal{D}_1 } \; \CICCmproj{\mathcal{D}_2}
\end{gather}

%% inst/b %%
%%%%%%%%%%%%
\begin{gather}
\CICCmproj{ 
\infer[\m{inst/b}]
{
\Gamma \judgeCI M \{ x = N \} : U [N/x]
}
{
\overset{\mathcal{D}_1}{ \Gamma \judgeCI M : ?\Pi x :T . U }
&
\overset{\mathcal{D}_2}{ \Gamma \judgeCI N : T }
& 
\cdots
}
}
\\
:=
\CICCmproj{\mathcal{D}_1} \; \{ x : \CICCmproj{\Gamma \vdash T : \m{kind}} = \CICCmproj{\mathcal{D}_2} \}
\end{gather}

%% inst/b %%
%%%%%%%%%%%%
\begin{gather}
\CICCmproj{ 
\infer[\m{inst/f}]
{
\Gamma \judgeCI M : U [N/x]
}
{
\overset{\mathcal{D}_1}{ \Gamma \judgeCI M : ?\Pi x : T . U }
&
\overset{\mathcal{D}_2}{ \Gamma \judgeCI N : T }
&
\cdots
}
}
\\
:=
\CICCmproj{\mathcal{D}_1} \; \{ x = \CICCmproj{\mathcal{D}_2} \}
\end{gather}

%% strength %%
%%%%%%%%%%%%%%
\[
\CICCmproj{ 
\infer[\m{strength}]
{
\Gamma \judgeCI M : U
}
{
\overset{\mathcal{D}}{ \Gamma, x : T \judgeCI M : U }
&
\cdots
}
}
:=
\CICCmproj{\mathcal{D}}
\]


%% abs/f %%
%%%%%%%%%%%
\[
\CICCmproj{ 
\infer[\m{abs/f}]
{
\Gamma \judgeCI M : ?\Pi x : T . U
}
{
\overset{\mathcal{D}}{ \Gamma \judgeCI M : [N/x]U }
&
\cdots
}
}
:=
?\lambda x : T . \CICCmproj{\mathcal{D}}
\]
\label{cicc-:proj}
\end{definition}


\begin{theorem}

\textbf{(Soundness of extraction)}  

\begin{alignat}{4}
\Gamma &\judgeCI &  & \implies & \CICCproj{\Gamma \judgeCI}^c & \judgeCI &
\\
\Gamma &\judgeCI & A : T & \implies & \CICCproj{\Gamma \judgeCI}^c & \judgeCI & \CICCproj{ \Gamma \judgeCI A : T }
\end{alignat}

\label{cicc-:sound}
\end{theorem}

Since $CICC$ permits $\eta$ equivalence and $CICC^-$ does not, the extraction in the 
reverse direction is no longer sound.  For our purposes this is not objectionable 
since $CICC$ is consistent and there is no reason to convert back into $CICC^-$, as it is 
used entirely as a pre-elaboration language.  Once terms are typechecked and type infered in $CICC^-$
they are typechecked in $CICC$ and normalized in $CICC$.  While the reverse extraction is
in general not sound, normal terms with normal types are clearly typable in $CICC^-$.

%%%%%%%%%%%%%%%%%%%%%%%%%%%%%%%%%%%%%%%%%%%%%%%%%%%%%%%%%%%%%
%%% Subtyping %%%%%%%%%%%%%%%%%%%%%%%%%%%%%%%%%%%%%%%%%%%%%%%
%%%%%%%%%%%%%%%%%%%%%%%%%%%%%%%%%%%%%%%%%%%%%%%%%%%%%%%%%%%%%
\subsection{Subtyping}

Similar to $ICC$, these rules result in a subtyping relation, which will be of
importance during type inference and elaboration.

\begin{definition}
Subtyping relation:
$\Gamma \judgeCI T \leq T' \;\; \equiv \;\; \Gamma, x : T \judgeCI x : T'$  where $x$ is new.
\end{definition}

\begin{lemma}
Subtyping is a preordering:
\[
\begin{array}{lr}
\infer-[\m{refl}]{ 
\Gamma \judgeCI T \leq T
}{
\Gamma \judgeCI T : s
}
&
\infer-[\m{trans}]{ 
\Gamma \judgeCI T_1 \leq T_3
}{
\Gamma \judgeCI T_1 \leq T_2
&
\Gamma \judgeCI T_2 \leq T_3
}
\end{array}
\]

\[
\infer-[\m{sub}]{ 
\Gamma \judgeCI M : T'
}{
\Gamma \judgeCI M \leq T
&
\Gamma \judgeCI T \leq T'
}
\]
\end{lemma}

This theorem is an application of the substitution lemma.

\begin{lemma}
Domains of products are contravariant and codomains are covarient:

\[
\infer-[]{ 
\Gamma \judgeCI \Pi x : T . U \leq \Pi x : T' . U'
}{
\Gamma \judgeCI T' \leq T 
&
\Gamma,x : T' \judgeCI U \leq U'
}
\]

\[
\infer-[]{ 
\Gamma \judgeCI \forall x : T . U \leq \forall x : T' . U'
}{
\Gamma \judgeCI T' \leq T 
&
\Gamma,x : T' \judgeCI U \leq U'
}
\]
\end{lemma}

Unlike traditional subtyping relations where an explicit subtyping rule must be included in the type system,
this system's subtyping relation is much easier to manage during unification, as it is simply
a macro for a provability relation.  

This allows one to implement higher order unification almost exactly
as is usual without to much modification as would be the case in a lattice unification algorithm.  
Instead, the modification is made to the search procedure, and subtyping constraints 
are realized as search terms.  

However, with the addition of the strengthening rule, 
this kind of modification not entirely necessary, 
as it is provable that this subtyping relation is symetric \ref{ci:sym}, 
and thus an entirely symetric unification algorithm can be presented.

Theorem \ref{ci:sym} isn't exactly obvious upon first glance, 
so we will provide intuitive justification first.

In $CICC$, by uniqueness of types, 
$\Gamma \vdash x : A$ and 
$\Gamma \vdash x : B$ implies
$A \equiv_{\beta\eta*} B$.  
In $CICC^{-}$ however, there is no such uniqueness of 
types properties.  
Rather, the $\m{inst/f}$ and $\m{abs/f}$ 
rules permit you to repsectively,
add an initialize an implicit argument, 
abstract implicitely upon an unused argument. 

Thus if $\Gamma , x : ?\Pi z : T . A \judgeCI x : A$
by implicit instantiation of the argument $z:T$,
we might also
derive
$\Gamma , x : A \judgeCI x : ?\Pi z : T . A$
given that $z \notin FV(x)$ and that 
$\Gamma , x : ?\Pi z : T . A \judgeCI x : A$ 
implies $ \Gamma , x : ?\Pi z : T . A \judgeCI$ 
which implies $ \Gamma\judgeCI x : (?\Pi z : T . A) : K$.

\begin{theorem}
\textbf{(Symmetry)}
$\Gamma \judgeCI A \leq B $ implies 
$\Gamma \judgeCI B' \leq A' $. where $A \equiv_{\beta} A'$ and $B \equiv_{\beta} B'$
\label{ci:sym}
\end{theorem}

\begin{proof}

This is proved by induction on the structure of the proof
of $\Gamma, x : A \judgeCI x : B$.  
The fine details of this proof are out of the scope of this paper and can be found in the Thesis \citep{mirman}.

\end{proof}
