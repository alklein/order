In this thesis I present a higher order logic programming language, Caledon,
with a pure type system and a Turing complete type inference and implicit
argument system based on the same logic programming semantics. Because
the language has dependent types and type inference, terms can be
generated by providing type constraints. I design the dynamic semantics
of this language to be the same used to perform type inference,
such that there is no disparity between compilation and running. The lack
of distinction between compilation and execution permits certain metaprogramming
techniques which are normally either unavailable or only possible
with second thought extensions. The addition of control structures such
as implicit arguments, shared holes, polymorphism, and nondeterminism
control makes programming computation during type inference more natural.
As a consequence of these extensions, unification problems must be generated
to solve for terms in addition to the usual problems generated to solve
for types. Furthermore, because every result of execution is a term in the
consistent calculus of constructions, Caledon can be considered an interactive
theorem prover with a less orthogonal combination of proof search and
proof checking than has previously been designed or implemented.
