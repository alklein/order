\section{Results}

For this thesis, I designed a logic programming language with a type system based on the
``Calculus of Constructions'' which integrated the notion of an implicitly quantified type in
a manner useful for automating proof search. I demonstrated a series of reductions from
this language to the ``Calculus of Constructions'' where the output of the language could
be interpreted as meaningful theorems. I provided an abstract machine for the language
based on higher order unification with proof search, and I demonstrated an elaboration
method to this machine. 

The semantics of the language based on this compilation and
evaluation joined the notions of type inference and traditional evaluation in a way that
does not appear to have been examined in great detail in the past.

I provided a method to constrain proof search of a predicate to a small subset of the
axioms in the environment using families. I demonstrated a way to explicitly control
whether a predicate was searched in a breadth first or depth first manner, allowing
constructs similar to fork and join to be defined.

I gave examples of usage of the Caledon language and demonstrated functionality
equivalent to type classes and ways to extend the applicability of this feature using library defined linearity checking.  

Finally, I provided an implementation of Caledon in Haskell and provided a
standard library. Since previous dependently typed logic languages did not include
polymorphism, standard libraries were not reasonable or possible to include. However, I included 
polymorphism, so that useful generic lists, type logic, printing, monad and
functor libraries became possible.
