\section{Linear Predicates}

One major drawback of the type class paradigm outlined in the previous section is the
inability for a typeclass to uniquely determine membership of type in a type class based
on floating predicates in the environment with matching signatures. While ideal behavior
is possible for theorems in the ``Calculus of Constructions'' which exhibit ideal parametricity,
the type “type” has the trivial inhabitant “type.”  Thus implementations
will nearly always resolve to this version.

\begin{figure}[H]
\begin{lstlisting}

defn serializeable : [T] { serializer : T −> string −> type } type
   | serializable-auto-find = [T] [ Serializer : T -> string -> type ]
      serializable T { serializer = Serializer }

\end{lstlisting}
\caption{This would be nice}
\label{lin:nice}
\end{figure}

For example, it would be very nice if the predicates in figure \ref{lin:nice} automatically
resolved to a reasonable instance of serializable.

\begin{figure}[H]
\begin{lstlisting}

defn serialize : {T} { serializer : T −> string −> type } T -> string -> type
  as ?\T : type . ?\ serializer : T -> string -> type . \v : T . \s : string 
   . serializer v s

\end{lstlisting}
\caption{Nice implementation}
\label{lin:imp}
\end{figure}

We could implement serialize as above in the definition in figure \ref{lin:imp}.
However, this kind of implementation will fail to determine the correct instance of ``serializer.''


\begin{figure}[H]
\begin{lstlisting}

query writeBool = serialize true ``true''
===>
query writeBool = 
   serialize 
       {T = bool}
       { serializer = \ x : bool . \ y : string . type } 
       true ``true''

\end{lstlisting}
\caption{Trivial Failure}
\label{lin:fail}
\end{figure}

Rather, it will infer a trivial predicate, as seen in the query in figure \ref{lin:fail}

\begin{figure}[H]
\begin{lstlisting}

defn show : {T}{shower : T -> string } T -> string
  as ?\T : type . ?\ shower : T -> string . \ v : T . shower v 

query writeBool = print (show true)
===>
query writeBool = print (show {T = bool}{shower = \ x : bool . nil })

\end{lstlisting}
\caption{Functions also fail}
\label{lin:ffail}
\end{figure}

One might think that functions, as an alternative to predicates, are immune to inhabitation by trivial and
incorrect values as in the above scenario. However, unless specified with their properties (tedious), 
functions have similar drawbacks, as is seen in the program in figure \ref{lin:ffail}.


\begin{figure}[H]
\begin{lstlisting}

defn show : {T}
            {shower : T -> string}
            {reader : string -> maybe T}
            {comp1  : [v] reader (shower v) =:= just v}
            {comp2  : [v] fromJust (reader s) (\x . shower x =:= s) type}
            T -> string
   as ?\ T : type
    . ?\ shower : T -> string
    . ?\ reader : _
    . ?\ comp1  : _
    . ?\ comp2  : _
    . \ v : T
    . shower v

\end{lstlisting}
\caption{Kind of a success using proofs}
\label{lin:success}
\end{figure}

One can sometimes work around this by including metatheorems about the implicit
functions, as in the figure \ref{lin:success}. However, proving the metatheorems is often tedious,
impossible, or sometimes just plain slow for the compiler.

One method under investigation to solve this ambiguity problem uses substructural dependent
quantification, in which types indicate that the function argument can  be
used only once in the term, but unlimited times in types. While
this is a subject of ongoing work on my part, I have included a description of my ideas.

\begin{figure}[H]
\begin{lstlisting}

defn serializeBool : bool -o string -o type
   | serializeBool-true = serializeBool true ``true''
   | serializeBool-false = serializeBool false ``false''

defn serialize : {T}{ serializer : T -o string -o type } T -> string -> type
  as ?\ T 
   . ?\ serializer : T -o string -o type ]
   .  \ v : T 
   .  \ s : string 
   . serializer v s

\end{lstlisting}
\caption{Linear types would be useful here}
\label{lin:linideal}
\end{figure}

$F : A \lolli B$ shall mean that the function $F$ only consumes a single resource of type $A$.  
$F : \forall_o x : A . B$ shall mean the same in a dependent setting.


The problem is solved in the figure \ref{lin:linideal} since the only function which linearly 
consumes a single boolean and a single string in the program and outputs a type is ``serializeBool''.  
Other functions that do this might be added later, but functions of the form 
$(\lambda x : \m{bool}. \lambda s : \m{string} . \m{type})$ are not possible.  
Unfortunately, something alon gthe lines of 
$(\lambda x : \m{bool}. \lambda s : \m{string} . \m{isString} s \wedge \m{isBool} x)$ might be possible, 
but these are significantly more manageable provided one is careful.  

Fortunately, the fact that we are working with higher order abstract syntax in the ``Calculus of Constructions''
means that linear dependent products are actually implementable withing Caledon.


\begin{figure}[H]
\begin{lstlisting}

defn restriction : type
   | linear = restriction
   | unused = restriction

defn restrictor : restriction -> restriction -> restriction -> type
   | restrictor-linear1 = restrictor linear unused linear
   | restrictor-linear2 = restrictor linear linear unused 
   | restrictor-unused = restrictor unused unused unused


defn restrict : restriction -> [ T : type ] [ P : T -> type ] ( [ x : T ] P x ) -> type
   | restrict-unused = 
     restrict unused T (\ x : T . P) (\ x : T . G)

   | restrict-linear = 
     restrict linear T (\ x : T . T) (\ x : T . x)

   | restrict-app = 
      restrict Ba T (\ x : T . P x (G x)) (\ x : T . (F x) (G x))
   <- restrict Bb T (\ x : T . [z : Q x] P x z ) (\ x : T . F x)
   <- restrict Bc T (\ x : T . Q x ) (\ x : T . G x)

   | restrict-lam = 
      restrict B T (\ x : T . [y : A] P x y) (\ x : T . \ y : A . F y x)
   <- [ y : A ] restrict B T (\ x : T . P x y) (\ x : T . F y x)

   | restrict-eta = 
      restrict B T (\ x : T . [y : A x] P x y) (\ x : T . \ y : A x. F x y)
   <- restrict B T (\ x : T . [y : A x] P x y) (\ x : T . F x)

\end{lstlisting}
\caption{Linearity in Caledon}
\label{lin:linimp}
\end{figure}



\begin{figure}[H]
\begin{lstlisting}

fixity lambda lolli
fixity lambda llam

defn lolli : [ T : type ] ( T -> type ) -> type
   | llam = [T]{TyF}[F : [x : T] TyF x]
            restrict linear T TyF F => ( lolli x : T . TyF x)

defn lapp : {A : type} { T : A -> type } [ f : lolli x : A . T x ] [ a : A ] T a -> type
   | lapp-imp = lapp (llam _ _ F _) V (F V)

fixity arrow -o
defn -o : type -> type -> type
  as \t : type . \t2 : type . lolli t (\x : t . t2)

\end{lstlisting}
\caption{Linear Dependent Product}
\label{lin:lindep}
\end{figure}


In the figure \ref{lin:linimp} the predicate ``restrict linear'' encodes the test that an arbitrary function, even one with a dependent type, is linear \citep{benton1993term} if its argument is used
exactly once in the term and potentially many times in the type.  
``lolli'' is the linear dependent type constructor and ``llam'' is the linear function constructor.
Provided the code from figure \ref{lin:linimp} was in the environment, figure \ref{lin:linideal} in fact works.


More of these cases could be made less ambiguous through use of an ordered dependent type 
constructor, but this is significantly more complicated to define, although certainly possible.


\begin{figure}[H]
\begin{lstlisting}

defn sum : nat -o nat -o nat -o type
   | sum-zero = [N : nat]
                [Sum : nat -o nat -o type]
                [Sum' : nat -o type]
                [Sum'' : type]
     lapp sum zero Sum -> lapp Sum N Sum' -> lapp Sum' N Sum'' -> Sum''
   | sum-succ = [N M R : nat]
                [Sum1 Sum2 : nat -o nat -o type ]
                [ Sum1' Sum2' : nat -o type ]
                [ Sum1'' Sum2'' : type ]
       lapp sum N Sum1 ->
       lapp Sum1 M Sum1' -> 
       lapp Sum1' R Sum1'' -> Sum1''
    -> lapp sum (succ N) Sum2 -> 
       lapp Sum1 M Sum2' ->
       lapp Sum2' (succ R) Sum2'' -> Sum2''
   
\end{lstlisting}
\caption{Use of a linear type}
\label{lin:use}
\end{figure}


Of course, as seen in figure \ref{lin:use}, actually using this linear dependent product is a bit
absurd.  It requires flattening application into a logic programming form where the target
is a type variable.



\begin{figure}[H]
\begin{lstlisting}

fixity application lapp
defn sum : nat -o nat -o nat -o type
   | sum-zero = [N : nat] *APP=lapp* sum zero N N
   | sum-succ = [N M R : nat] 
        *APP=lapp* sum N M R 
     -> *APP=lapp* sum (succ N) M (succ R)
\end{lstlisting}
\caption{Example of a syntax for flattening application}
\label{lin:flat}
\end{figure}


That the end result is a type variable means that the family checking algorithm
is no longer applicable. In this case, it is helpful to either hard code
linearity or provide a syntax for flattening successive applications using a predicate, as
seen in figure \ref{lin:flat}

Syntax for applications could potentially extend the notion of a family such that
predicates using these applications could also be frozen. In this case, the applicator $x$ used with 
$*APP = x*$ would be declared, so that the compiler could check that its type matches the form $\{ A : \m{type} \}\{ T : A \rightarrow \m{type} \}[f : \m{some}-\m{product} x : A . T x] [ a : A] T a \rightarrow \m{type}$.
