\section{Typeclasses}
% -------------------------------------------------------------
\begin{frame}[fragile]
\frametitle{Serialize}

\begin{lstlisting}
defn serializeBool : bool -> string -> type
  >| serializeBool-t = 
       serializeBool true ``true''
  >| serializeBool-f = 
       serializeBool false ``false''

query readQ = 
   exists B : bool. serializeBool B ``true''

query printQ = 
   exists S : string . serializeBool false S
\end{lstlisting}

\end{frame}
% -------------------------------------------------------------

\begin{frame}[fragile]
\frametitle{Type Class for Serialize}

\begin{lstlisting}
defn serialize : {T}{serable : T -> string -> type} 
                 T -> string -> type
   | serializeImp = 
     [ Serer : T -> string -> type ]
     [ Serable : serializable T { serer = Serer }]
     serialize { serable = Serable } V S
     <- Serer V S

defn serializable : 
      [T]{serer : T -> string -> type} type
   | serialize-bool = 
      serializable bool { serer = serializeBool }
   | serialize-nat = 
      serializable nat { serer = serializeNat }

query readQueryBool = 
   exists B : bool . serialize B ``true''
query printQueryNat = 
   exists S . serialize (succ (succ zero)) S
\end{lstlisting}

\end{frame}


% -------------------------------------------------------------
\subsection[Linearity]{Putting it all Together}

\begin{frame}
\frametitle{Putting it all together}
\begin{itemize}
\item Can create a linearity predicate.
\item Uses notion of typeclass
\item acts on higher order abstract syntax
\item works on types by employing universes
\end{itemize}
\end{frame}


\begin{frame}[fragile]
\frametitle{Linearity}
\begin{lstlisting}
defn linear : 
     [T] [ P : T -> type ] 
     ( [ x : T ] P x ) -> type
   | linear-var = 
      linear T (\ x . T) (\ x : T . x)
   | restrict-lam = 
      linear T (\ x . [y : A] P x y) 
               (\ x . \ y : A . F y x)
      <- [ y : A ] 
         linear T (\ x . P x y) (\ x . F y x)

defn lolli : 
      [T] ( T -> type ) -> type
   | llam = 
      [T]{TyF}[F : [x : T] TyF x]
      linear T TyF F => lolli T TyF
\end{lstlisting}
\end{frame}


\begin{frame}
\frametitle{Uses of Linearity}
\begin{itemize}
\item Solves unique instance problem
\item Could be used to automate type class instancing
\item Application could get special syntax
\item Subject of future research
\end{itemize}
\end{frame}
