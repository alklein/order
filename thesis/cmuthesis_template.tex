%for a more compact document, add the option openany to avoid
%starting all chapters on odd numbered pages
\documentclass[12pt]{cmuthesis}

% This is a template for a CMU thesis.  It is 18 pages without any content :-)
% The source for this is pulled from a variety of sources and people.
% Here's a partial list of people who may or may have not contributed:
%
%        bnoble   = Brian Noble
%        caruana  = Rich Caruana
%        colohan  = Chris Colohan
%        jab      = Justin Boyan
%        josullvn = Joseph O'Sullivan
%        jrs      = Jonathan Shewchuk
%        kosak    = Corey Kosak
%        mjz      = Matt Zekauskas (mattz@cs)
%        pdinda   = Peter Dinda
%        pfr      = Patrick Riley
%        dkoes = David Koes (me)

% My main contribution is putting everything into a single class files and small
% template since I prefer this to some complicated sprawling directory tree with
% makefiles.

% some useful packages
\usepackage{times}
\usepackage{fullpage}
\usepackage{graphicx}
\usepackage{amsmath}
\usepackage[numbers,sort]{natbib}
\usepackage[backref,pageanchor=true,plainpages=false, pdfpagelabels, bookmarks,bookmarksnumbered,
%pdfborder=0 0 0,  %removes outlines around hyper links in online display
]{hyperref}
\usepackage{subfigure}

% Approximately 1" margins, more space on binding side
%\usepackage[letterpaper,twoside,vscale=.8,hscale=.75,nomarginpar]{geometry}
%for general printing (not binding)
\usepackage[letterpaper,twoside,vscale=.8,hscale=.75,nomarginpar,hmarginratio=1:1]{geometry}

% Provides a draft mark at the top of the document. 
\draftstamp{\today}{DRAFT}

\begin {document} 
\frontmatter

%initialize page style, so contents come out right (see bot) -mjz
\pagestyle{empty}

\title{ 
{\bf Metaprogramming via Type Inference and Logic Programming}}
\author{Matthew Mirman}
\date{May 2013}
\Year{2013}
\trnumber{}

\committee{
Frank Pfenning \\
Someone else \\  %% TODO: REMOVE
Yet another person \\ %% TODO: REMOVE
Someone from a strange and faraway land %% TODO: REMOVE
}

\support{}
\disclaimer{}

\keywords{Logic Programming, Pure Type System, 
Type Inference, Higher Order Unification, Caledon Language}

\maketitle

\begin{dedication}
For those who know. %% TODO: REMOVE
\end{dedication}

\pagestyle{plain} % for toc, was empty

%% Obviously, it's probably a good idea to break the various sections of your thesis
%% into different files and input them into this file...

\begin{abstract}

In this thesis I present a logic programming language, Caledon, with a pure type system
and a turing complete type inference and implicit argument system based on the semantics of the
object language.  Because the language has dependent types and type inference, terms can be generated 
by providing type constraints.  The addition of control structures such as implicits, costructor hiding,
shared holes, existential quantification, polymorphism, and nondeterminism make the language ideal for 
creating libraries for defining EDSLs.  Furthermore, the system can be made consistent and sound with 
the addition of totality, worlds, and universe checks. Proof irrelivance, universes, and uniqueness 
checking can be used to constrain the nondeterminism of type inference in a pure type system, and help 
programmers control coherence.



\end{abstract}

\begin{acknowledgments}
I thank the academy and Frank Pfenning. %% TODO: REMOVE
\end{acknowledgments}

\tableofcontents
\listoffigures
\listoftables

\mainmatter

%% Double space document for easy review:
\renewcommand{\baselinestretch}{1.66}\normalsize

% The other requirements Catherine has:
%
%  - avoid large margins.  She wants the thesis to use fewer pages, 
%    especially if it requires colour printing.
%
%  - The thesis should be formatted for double-sided printing.  This
%    means that all chapters, acknowledgements, table of contents, etc.
%    should start on odd numbered (right facing) pages.
%
%  - You need to use the department standard tech report title page.  I
%    have tried to ensure that the title page here conforms to this
%    standard.
%
%  - Use a nice serif font, such as Times Roman.  Sans serif looks bad.
%
% Other than that, just make it look good...

\chapter{Introduction}
``To be fair, even in last century's typed languages, 
types had a beneficial organisational effect on programmers. 
This century, it's just possible types will have a comparable effect on programs. 
Types are concepts and now mechanisms supporting program-discovery as well as error-discovery. 
I think that's more than just gravy.''
 - Conor McBride



\chapter{Background}
\chapter{Background}

\section{Logic Programming}

Logic programming languages such as Prolog were originally designed as part of the AI
program, in much the same way Lisp was. Automated reasoning’s natural goal would
be to be able to arbitrarily prove theorems. A logic programming language would be
a set of axioms and a predicate, and if the predicate could be proven through those axioms,
the automated theorem prover would halt. These proof search procedures were
then constrained into useful programming semantics. When performed in a backtracking
manner, the process of proof search represented an formulation of procedural code
with powerful pattern matching. The Caledon language is a higher order backtracking
logic programming language in the style of Twelf. In this section I present some basic
intution for logic programming, rather than explaining it technically, and demonstrate the 
descriptive power of the system implemented in Caledon.

\FloatBarrier
\subsection{Basics}
We begin by defining addition on unary numbers in Caledon shown in shown in \ref{code:add}.

\begin{figure}[H]
\begin{lstlisting}
defn add : nat -> nat -> nat -> prop
   | addZ = add zero A A
   | addS = add (succ A) B (succ C) 
             <- add A B C
\end{lstlisting}
\caption{Addition in Caledon}
\label{code:add}
\end{figure}

One might notice that this definition is incredibly similar to its Haskell counterpart shown in \ref{code:hask}.

\begin{figure}[H]
\begin{lstlisting}
add :: nat -> nat -> nat
add Zero a = a
add (Succ a) b = Succ c
   where c = add a b
\end{lstlisting}
\caption{Addition in Haskell}
\label{code:hask}
\end{figure}

We can read the logic programming definition as we would read the functional definition with pattern
matching, knowing that an intelligent compiler would be able to convert the first into the second.  
search allows one to define essentially nondeterministic programs. A common use for
logic programming has been to search for solutions to combinatorial games such as tic-tac-toe, 
without the programmer worrying about the order of the search. As this tends
to produce ineficient code, this use style is discouraged. Rather, a more procedural view of logic programming is encouraged where pattern match and search is performed
in the order it appears.

\begin{figure}[H]
\begin{lstlisting}
defn p : T_1 -> ... -> T_r -> prop
  >| n1 = p T_1 ... T_r <- p_1,1 ... <- p_1,k_1
...
  >| nN = p T_1 ... T_r <- p_n,1 ... <- p_n,k_n

query prg = p t1 ... tr
\end{lstlisting}
\caption{Format of a Caledon Logic Program}
\label{code:format}
\end{figure}

In this view, a program of the form \ref{code:format}
should be considered a program which first attempts to prove using axiom n1 by matching prg with ``p T1 ... Tr'' and then
attempting to prove $p_{1,1}$ and so on.


\FloatBarrier
\subsection{Higher Order Logic Programming}

Just as imperative programs benefit from the addition of higher order functions, logic programs benefit from the addition of both
higher order predicates and patterns.  

A common request by functional programming language users is that they would like to be able to abstract patterns even more than just over
arguments.  

\begin{figure}[H]
\begin{lstlisting}
func (Var a) = code1
func (Forall var val) = Exists var code
func (Exists var val) = Forall var code
func (And a b) = code2
\end{lstlisting}
\caption{Unecessarily verbose code}
\label{code:verbose}
\end{figure}

A serious amount of code is repeated in lines 2 and 3 of example \ref{code:verbose}, 
but in common languages it is impossible to simplify this.
What a programmer would actually like to express is shown in \ref{code:Fideal}.

\begin{figure}[H]
\begin{lstlisting}
func (Var a) = code1
func (f var val) = f var code
func (And a b) = code2
\end{lstlisting}
\caption{Ideal code. $f$ is a constructor in strong head normal form.}
\label{code:Fideal}
\end{figure}

In higher order logic programming languages like $\lambda$Prolog, this sort of simplification 
is in fact possible to express as shown in \ref{code:lprolog}.

\begin{figure}[H]
\begin{lstlisting}
defn func : term -> term -> prop 
  | f1 = func (var A) R <- [code1]
  | f2 = func (F Var Val) (f Var R) <- [code]
  | f3 = func (and A B) R <- [code2]
\end{lstlisting}
\caption{Expressing constructor variables in patterns}
\label{code:lprolog}
\end{figure}

\FloatBarrier
\subsection{Higher Order Programming}

Fortunately, higher order functions need not be restricted to patterns.  Macros provide even more ways to generalize code. 

A great example is the function application operator from Haskell.  
We can define this in Caledon as shown in \ref{code:macros}

\begin{figure}[H]
\begin{lstlisting}
fixity right 0 @
defn @ : (At -> Bt) -> At -> Bt
  as ?\ At Bt . \ f : At -> Bt . \ a : At . f a

\end{lstlisting}
\caption{Definitions for expressive syntax}
\label{code:macros}
\end{figure}

In many cases, allowing these definitions allows for significant simplification of syntax.
The reader familiar with languages like Twelf, Haskell, and Agda might notice the 
implicit abstraction of the type variables At and Bt in the type of @ in \ref{code:macros}. The rest of this
paper is concerned with formalizing these implicit abstractions and letting them have
as much power as possible. For example, one might make these abstractions explicit by
instead declaring at the beginning \ref{code:expHask}

\begin{figure}[H]
\begin{lstlisting}
infixr 0 @
(@) :: forall At Bt . (At -> Bt) -> At -> Bt
(@) = \ f . \ a . f a
\end{lstlisting}
\caption{Explicit Haskell style abstractions}
\label{code:expHask}
\end{figure}

However, in a dependently typed language, every function type is a dependent product
(forall).  This makes it necessary to provide a new (explicit) implicit dependent product - $?\forall$ or $?\Pi$.


\begin{figure}[H]
\begin{lstlisting}
fixity right 0 @
defn @ : {At Bt:prop} (At -> Bt) -> At -> Bt
  as ?\ At Bt . \ f . \ a . f a

\end{lstlisting}
\caption{The (explicit) implicit equivalent of \ref{code:macros}}
\label{code:expimp}
\end{figure}

Haskell also has type classes. For example, the type of “show” can be seen in \ref{code:showty}.

\begin{figure}[H]
\begin{lstlisting}
show :: Show a => a -> String
\end{lstlisting}
\caption{The type of show}
\label{code:showty}
\end{figure}

In Caledon, these can be written similarly as in \ref{code:cshowty}

\begin{figure}[H]
\begin{lstlisting}
defn show : showC A => A -> string
defn show : {unused : showC A } A -> string
\end{lstlisting}
\caption{Equivalent types for show in Caledon}
\label{code:cshowty}
\end{figure}

However, since implicit arguments are a natural extension of the dependent type
system in Caledon, no restrictions are made on the number of arguments, or difficulty
of computing. Unfortunately, since computation is primarily accomplished by the logic
programming fragment of the language rather than the functional fragment of the language,
the correspondence between these programmable implicit arguments and type
classes is not one to one. It is possible to replicate virtually all of the functionality of
type classes in the implicit argument system, but the syntax required to do so can become
verbose. Rather than attempting to simulate type classes, more creative uses are
possible, such as computing the symbolic derivative of a type for use in a (albeit, slow
and unnecessary) generic zipper library, or writing programs that compile differently
with different types in different environments.




\section{Initial Examples}


In the previous section I gave an introduction to the notion of logic programming using
both the familiar language of Haskell and the new language of Caledon. In this section
I will build upon these ideas by introducing logic programming with polymorphism
through building a set of standard polymorphic type logic library.

There are a few ways of defining sums in Caledon.


\begin{figure}[H]
\begin{lstlisting}
defn and : type -> type -> type
   | pair = and A B <- A <- B

defn fst : and A B -> A -> type
   | fstImp = fst (pair Av Bv) Av

defn snd : and A B -> B -> type
   | sndImp = snd (pair Av Bv) Bv
\end{lstlisting}
\caption{Logical conjunction}
\label{code:lconj}
\end{figure}


In this first, simplest way (as seen in figure \ref{coe:lconj}, we define a predicate for “and” and
predicates for construction and projection. This method has the advantage of doubling
as a form of sequential predicate.

\begin{figure}[H]
\begin{lstlisting}

query main = and (print ``hello ``) (print `` world!'')
\end{lstlisting}
\caption{Use of logical conjunction}
\label{code:lconjuse}
\end{figure}

In the figure \ref{code:lconjuse} the query will output ``hello world!''.

\begin{figure}[H]
\begin{lstlisting}

defn and : type -> type -> type
  as \ a : type . \ b : type . 
      [ c : type ] (a -> b -> c) -> c

defn pair : A -> B -> and A B
  as ?\ A B : type .
      \ a b .
      \ c : type .
      \ proj : A -> B -> c .
        proj a b

defn fst : and A B -> A
  as ?\ A B : type .
      \ pair : [c : type] (A -> B -> c) -> c .
        pair A (\ a b . a)

defn snd : and A B -> A
  as ?\ A B : type .
      \ pair : [c : type] (A -> B -> c) -> c .
        pair B (\ a b . b)

\end{lstlisting}
\caption{Church style conjunction}
\label{code:cconj}
\end{figure}

In the case of figure \ref{code:cconj}, we do not add any axioms without their proofs.   
In this example we also introduce the dependent product written in the form $[ a : t_1 ] t_2$.

This case mimics the version usually seen in the Calculus of Constructions and has the advantage of
the projections being functions rather than predicates.


\begin{figure}[H]
\begin{lstlisting}

defn churchList : type −> type
  as \A : type . [ lst : type] lst −> (A −> lst −> lst ) −> lst

defn consCL : [ B : type ] B -> churchList B -> churchList B
  as \ C : type .
     \ V : C .
     \ cl : churchList C .
     \ lst : type .
     \ nil : lst .
     \ cons : C -> lst -> lst .
     cons V (cl lst nil cons)

defn nilCL : [ B : type ] churchList B
  as \ C : type .
     \ lst : type .
     \ nil : lst .
     \ cons : C -> lst -> lst .
       nil

defn mapCL : { A B } ( A -> B) -> churchList A -> churchList B
  as ?\ A B : type .
      \ F : A -> B.
      \ cl : churchList A .
      \ lst : type .
      \ nil : lst .
      \ cons : B -> lst -> lst .
        cl lst nil (\ v . cons (F v))

defn foldrCL : { A B } ( A -> B -> A) -> A -> churchList B -> A 
  as ?\ A B : type . 
      \ F : A -> B -> A .
      \ bc : A .
      \ cl : churchList B .
        cl A bc (\ v : B . \ c : A . F c v)

\end{lstlisting}
\caption{Church style list}
\label{code:clist}
\end{figure}

In the Church form of a list, folds and maps are possible to implement as functions
rather than predicates. However, their implementation is verbose and doesn’t permit
more complex functions.


\begin{figure}[H]
\begin{lstlisting}

defn list : type -> type
   | nil = list A
   | cons = A -> list A -> list A

defn concatList : list A -> list A -> list A -> type
   | concatListNil = [ L : list A ] concatList nil L L
   | concatListCons = 
          concatList (cons (V : T) A) B (cons V C) 
       <- concatList A B C

defn concatList : list A -> list A -> list A -> type
   | concatListNil = [ L : list A ] concatList nil L L
   | concatListCons = 
          concatList (cons (V : T) A) B (cons V C) 
       <- concatList A B C

defn mapList : (A -> B -> type ) -> list A -> list B -> type
   | mapListNil = [ F : A -> B ] mapList F nil nil
   | mapListCons = [F : A -> B ] 
            mapList F (cons V L) (cons (F V) L')
            <- mapList F L L'

defn pmapList : (A -> B -> type) -> list A -> list B -> type
   | pmapListNil = [ F : A -> B -> type] pmapList F nil nil
   | pmapListCons = [F : A -> B -> type] 
            pmapList F (cons V L) (cons V' L')
            <- F V V'
            <- mapList F L L'
\end{lstlisting}
\caption{Logic List}
\label{code:llist}
\end{figure}

The logic programming version can be seen in figure \ref{code:llist}. It is important to note that
we can now map a predicate over a list rather than just mapping a function over a list.

\section{Pure Type Systems}

The type system for caledon is a pure type system \citep{mckinna1993pure} 
extended with explicit recursive types and implicit types.  In this section,
I discuss what a pure type system is and what its properties are.

Pure type systems are generalizations of the lambda cube
\citep{barendregt1991introduction} which allow for arbitrary 
relationships between terms and types.
With proper selection of constants, sorts, axioms, and relations,
pure type systems can embed the ``Calculus of Constructions'' \citep{coquand1986calculus}
and many other type systems one might want to construct.

As generalizations, these systems are important, as it has been proven by Jutting \citep{jutting1993typing} that 
type checking for normalizing pure type systems with finite axiom sets are decidable.  Thus, by showing how a system is a pure type system and is normalizing, 
you get decidability of type checking nearly for free.

It has also been shown that these systems have utility.  
Roorda \citep{roorda2001pure} gave an implementation of a functional programming language with 
pure type system and demonstrated its utility.

A pure type system is a set $S$ of sorts, 
$A\subseteq S \times S$ of axioms, and a relation 
$R \subseteq S \times S \times S$ along with the following grammar and inference rules:

\begin{definition}
\textbf{(PTS Syntax)}
\[ 
E ::=  V 
 \orr S 
 \orr E\;E 
 \orr \lambda V : E . E 
 \orr \Pi V : E . E 
\]

\label{pt:syntax}
\end{definition}

\begin{definition}

\textbf{(PTS Typing Rules)}

\[ \begin{array}{lr}
\infer[\m{WF-E}]
{
\cdot \vdash
}
{}
&
\infer[\m{WF-S}]
{
\Gamma, x : T \vdash
}
{
\Gamma \vdash T : s
&
x \notin DV(\Gamma)
}
\end{array} \]

\[
\infer[\m{axioms}]
{
\Gamma \vdash c : s
}
{
\Gamma \vdash
&
(c,s) \in A
}
\]

\[
\infer[\m{start}]
{
\Gamma,x:A \vdash x :A
}
{
\Gamma \vdash A:s
&
s \in S
}
\]

\[
\infer[\m{weakening}]
{
\Gamma,x:C \vdash A:B
}
{
\Gamma \vdash A:B
&
\Gamma \vdash C:s
&
s \in S
}
\]


\[
\infer[\m{product}]
{
\Gamma \vdash (\Pi x : A . B) : s_3
}
{
\Gamma \vdash A : s_1
&
\Gamma,x:A \vdash B : s_2
&
(s_1,s_2,s_3) \in R
}
\]

\[
\infer[\m{application}]
{
\Gamma \vdash F V : [V/x] B
}
{
\Gamma \vdash F : (\Pi x : A . B)
&
\Gamma \vdash V : A
&
\text{V is free for x in B}
}
\]

\[
\infer[\m{abstraction}]
{
\Gamma \vdash (\lambda x : A . F) : (\Pi x : A . B)
}
{
\Gamma , x : A\vdash F : B
&
\Gamma \vdash (\Pi x : A . B) : s
&
s \in S
}
\]

\[
\infer[\m{conversion}]
{
\Gamma \vdash A : B'
}
{
\Gamma \vdash A : B
&
\Gamma \vdash B \equiv_{\beta\eta\nu*} B'
&
\Gamma \vdash B' : s
&
s \in S
}
\]

\label{pt:typing}
\end{definition}


As Barendgregt\citep{barendregt1991introduction} points out, the common type theories can be recast as pure type systems
by choice of axioms.  
In the simplest example, the only axioms chosen are $(*,\Box)$ along with 
the single relationship $(*,*,*)$. This system describes the simply typed lambda calculus, 
where only terms can depend on terms.  We say that $A \rightarrow B \equiv \Pi x : A . B$ iff $ x \notin FV(B)$.

\begin{theorem} 
Subject Reduction: If $\Gamma \vdash A : T$ and $A \Rightarrow_\beta B$ then $\Gamma \vdash B : T$
\end{theorem}

Geuvers and Nederhof \citep{geuvers1991modular} proved subject reduction for any calculus on the $\lambda$ cube.
This property can be proved syntactically by induction on the structure of the typing derivation 
and there exist Twelf and Agda verified proofs of this property.  
Note that this is a useful property 
to maintain, even in the face of inconsistency of a system, because at the very least, the 
property allows for a consistent understanding of typing terms.

\begin{theorem}
Uniqueness of Types: If $\Gamma \vdash A : T$ and $\Gamma \vdash A : T'$ then $T \equiv_\beta T'$
\end{theorem}

The uniqueness of types with respect to $\beta$ reduction has also been shown for any system on the $\lambda$ cube.  
This last property is important to showing the decidability of type inference in 
the caledon language without implicits.



\begin{lemma}
Strengthening
\[
\infer-[\m{strength}]
{
\Gamma \vdash M : U
}
{
\Gamma , x : T \vdash M : U
&
x \notin FV(M) \cup FV(U)
}
\]
\end{lemma}

As it turns out, the strengthening lemma has important implications to the generation of bindings 
during proof search.

\section{Dynamics}

Unification lies at the heart of the semantics of the Caledon language.  
In this section we present the syntax of the unification problems, as well as a modified version of the algorithm 
presented in \citep{pfenning1991logic} and \citep{pfenning1991unification} suited for implicit argument search.

Checking for the reducability of two full lambda terms has long been known to be only semidecidable.  
The matter becomes even more complicated when checking for the equality of terms with variables bound
by both existential and universal quantifiers.  Research from the past thirty years has constrained
the problem to a decidable subset known as the pattern fragment.

\subsection{Unification Terms}

\begin{definition}
Unification Terms:

\[
U ::= U \wedge U 
 \orr \forall V : T . U
 \orr \exists V : T . U 
 \orr T \doteq T
 \orr T \in T
 \orr \top
\]
\label{def:hou:syn}
\end{definition}

When $\doteq$ is taken to mean $\equiv_{\beta\eta\alpha*}$, the unification problem is to determine 
whether a statement $U$ is ``true'' in the common sense, and provide a proof of the truth of the statement. 

While we do not discuss the semantics of $T \in T$ or $T \in T >> T \in T$ here, 
one can refer to \citep{pfenning1991logic} for a complete presentation.

Unification problems of the form 
$\forall x : T_1 . \exists y : T_2 . U $ can be converted to the form
$\exists y : \Pi x : T_1 . T_2 . \forall x : T_1 . [y\; x / y ]U $ 
in the process known as raising. Unification
statements can always quantified over unused variables: $U \implies Q x : T . U$.  

Thus, statements can always be converted to the form
\[
\exists y_1 \cdots y_n . \forall x_1 \cdots x_k . S_1 \doteq V_1 \wedge \cdots S_r \doteq V_r
\]

\subsection{Typed Implicit Hereditary Substitution}

Before we can properly specify the semantics of this language, 
we must define typed hereditary substitution for this calculus.  This is important, as higher order unification for the calculus of constructions 
requires that terms maintain $\eta$-long normal form after substitutions have been performed.
Here, we do describe typed hereditary substituion in full, as a more complete presentation can be found in \citep{keller2010normalization}.

The formulation of hereditary substitution in the presence of 
implicit arguments is not that unlike the presentation of
hereditary substitution without implicit arguments \citep{miller1991uniform}, 
but for the additional checks required.
The main difficulty is the notion of an acceptable substitution. 
Because implicit bindings are not $\alpha$ convertible, 
certain substitutions are not permitted.  
Because as many substitions should be permitted as possible, 
the situation becomes significantly more complex in the 
hereditary case, where substitutions might not carry types.  
The easiest way to define substitution in this case is with an ``illegal'' alpha substitution, 
which maps implicitly bound variables to fresh names, 
and produces a memory to map them back. 


In this case, we can say that a substitution $[S/x] M$ is legal if 
$FV(S) \subseteq FV(\alpha_I^-1( [\alpha_I(S)/x] M) )$.

\begin{definition}
\textbf{(Implicit Typed Hereditary Substitution)}

\[
[S / x : A]^n_{\Gamma } (?\lambda y : B . N) := ?\lambda y:B . [S / x : A]^n_{\Gamma, y : B} N
\] 

\[
\eta^{-1}_{?\Pi x : A . B}(N) := ?\lambda x : A . N \; \{ x = \eta^{-1}_A(x) \}
\] since $N$ being typable by $?\Pi x $ means that $x$ can not appear free in $N$

\[
\m{H}_{\Gamma}(P \downarrow ?\Pi y : B_1 . B_2 , \{ v := N \} ) := P\; \{ v := N \} \downarrow [N/y : B_1]^n_{\Gamma}B_2
\]

\[
\m{H}_{\Gamma} ((?\lambda v : A_1 . N) \uparrow ?\Pi v : A_1 . A_2 , \{ v := P \}) 
:= [P/v]^n_{\Gamma \vdash v : A_1} N \uparrow A_2
\]

\[ 
\m{H}(?\lambda v : T . P \uparrow \_ , A) := ?\lambda v : T . \m{H}(P,A)
\]

\label{def:hered}
\end{definition}


\subsection{Unification Term Meaning}

We can provide an provability relation of a unification formula
based on the obvious logic.

\begin{definition}
$\Gamma \Vdash F $ can be interpreted as $\Gamma$ implies $F$ 
is provable.

\[ \begin{array}{lr}
\infer[\m{equiv}]{
\Gamma \Vdash M \doteq N
}{
\Gamma \vdash M : A
&
M \equiv_{\beta\eta\alpha*} N
&
\Gamma \vdash N : A
}
&
\infer[\m{true}]{
\Gamma \Vdash \top
}{}
\end{array} \]

\[
\infer[\m{conj}]{
\Gamma \Vdash F \wedge G
}{
\Gamma \Vdash F
&
\Gamma \Vdash G
}
\]

\[ \begin{array}{lr}
\infer[\m{exists}]{
\Gamma \Vdash \exists x : A . F
}{
\Gamma \Vdash [M/x] F
&
\Gamma \vdash M : A
}
&
\infer[\m{forall}]{
\Gamma \Vdash \forall x : A . F
}{
\Gamma, x : A \Vdash F
}
\end{array} \]

\label{def:hou:prf}
\end{definition}

While a truly superb logic programming language might 
be able to convert this very declarative 
specification into a runnable program, 
the essentially nondeterministic rule for existential
quantification in a unification formula prevents an 
obvious deterministic algorithm from being extracted.


\subsection{Higher Order Unification for CC}

\newcommand{\UnifiesTo}{\;\longrightarrow\;}

We now present an algorithm, similar to that presented in 
\citep{pfenning1991logic} for unification in the 
Calculus of Constructions.  Because we have already 
presented typed hereditary substitution with $\eta$-expansion, 
the presentation here will not add much 
but for types in the substitutions.  

$F \UnifiesTo F'$ shall mean that $F$ can be transformed to $F'$
without modifying the provability. 
An equation $F[G]$ will stand as notation for highlighting $G$
under the formulae context $F$.  
As an example, if we were to examine the formula 
$\forall x . \forall n . \exists y . ( y \doteq x \wedge \forall z . \exists r . [ x z \doteq r] )$
but were only interested in the last portion, we might instead write it as
$\forall x . F[\forall z . \exists r . [ x z \doteq r]]$
Again, $\phi$ shall be an injective partial permutation. 

Furthermore, rather than explicitly writing down the result of unification, 
we shall use $\exists x. F \UnifiesTo \exists x . [ L / x] F$ 
to stand for $\exists x. F \UnifiesTo \exists x . x \doteq L \wedge [ L / x] F$

The unification rules are symetric, so any rule of the form 
$M \doteq N \sim N \doteq M$ practically.

Also, for the purpose of typed normalizing heredetary substitution, 
a formula prefix $F[e]$ of the form 
$Qx_1:A_1 . E_1\wedge \cdots Qx_n : A_n . e$ shall be considered as a context
$x_1 : A_1 ,\cdots ,x_n : A_n$ when written $\nu^-1(F)$.

\setcounter{tcase}{0}

\begin{tcase}
Lam-Any
\end{tcase}

\[
F[\lambda x : A . M \doteq N]
\UnifiesTo
F[\forall x : A . M \doteq \m{H}_{\nu^-1(F),x:A}(N , x)]
\]

Because application is normalizing, this can cover the case where both $N$ is also a $\lambda$ 
abstraction.

\begin{tcase}
Lam-Lam
\end{tcase}

\[
\lambda x : A . M \doteq \lambda x : A . N
\UnifiesTo
\forall x : A . M \doteq N
\]

While this rule is not explicitly necessary as it is covered by the Lam-Any rule, 
when working
in a substitutive system with explicit names rather than DeBruijn indexes, 
this helps to reduce the number of substitutions from an original name. 

Lastly, these reductions make the 
assumption that no variable name is bound more than once.
This can seem restrictive, but it is possible to 
work past by alpha converting everywhere and annotating
new variables with their original names, and alpha converting
back to the original after unification. The other option is again
to use DeBruijn indexes.  DeBruijn indexes have their own drawbacks
here, as certain transformation such as ``Raising''
or the ``Gvar-Uvar'' rules involve insertion of multiple
variables into the context at an arbitrary point, 
which requiring the lifting of many variable names.  
It is possible to implement higher order unification
with DeBruijn indexes safetly and efficiently, 
but this is out of the scope of the thesis.

\begin{tcase}
Ilam-Ilam-Same
\end{tcase}

This case behaves just as the lam-lam case does, but only for implicit abstractions with the same names.

\begin{tcase}
Ilam-Other
\end{tcase}

if the constraint is of the form $F[(?\lambda x : A . M) A_1 \cdots A_n \doteq R]$, where $x$ is not constrained
in a prefix of $R$, we transition to 
\[
F[\exists x' : A .  x' \in A \wedge H(\cdots , H([x' / x : A]_F M, A_1), \cdots A_n) 
\]
\[
\doteq H(R, \{ x : A = x' \}) ]
\]

\begin{tcase}
Iforall-Iforall-Same
\end{tcase}

Because universal quantification is also subject to these subtyping rules, we do a similar thing to what 
we do in the case of implicit lambda abstraction: if the names match on both side, we unify.  

\begin{tcase}
Iforall-Other
\end{tcase}

This case is a bit different however, since in the constraint $F[?\Pi x : A . M \doteq R]$, $R$ is no longer an implicit
abstraction - rather it should be a type.  Here, we simply transition to 

\[
F[\exists x' : A .  x' \in A \wedge H(\cdots , H([x' / x : A]_F M, A_1), \cdots A_n) \doteq R ]
\]

\begin{tcase}
Uvar-Uvar
\end{tcase}

In the traditional case without implicit constraints, we have the following transition:

\[
F[\forall y : A . G[y M \doteq y N  ]]
\UnifiesTo
F[\forall y : A . G[ M_1 \doteq \wedge N_1 \cdots]]
\]

However, when implicit constraints are permitted, the same universally quantified variables might take different numbers of arguments.  
In this case, we must remove argument the unnecessary implicit constraints.  

We might then be presented with the following constraint: 
\[
F[\forall y : A . G[y M_1 \cdots M_m \doteq y N_1 \cdots N_n  ]]
\]
We define the following matching function $\uplus$


\[
\infer{
M_i M \uplus N_j N
\Rightarrow 
M_i \doteq N_i \wedge (M \uplus N)
}{
M_i \neq \{ a : A = B \}
&
N_i \neq \{ a : A = B \}
}
\]

\[
\{ a : A = M_i \} M \uplus \{ a : A' = N_j \} N
\Rightarrow
M_i \doteq N_i \wedge (M \uplus N)
\]


\[
\infer{
\{ a : A = M_i \} M \uplus N
\Rightarrow 
M \uplus N
}{
a \notin CN(N)
}
\]

Using this dropping match, we can define the transition as follows.

\[
\infer{ 
F[\forall y : A . G[y M_1 \cdots M_m \doteq y N_1 \cdots N_n  ]]
\UnifiesTo
F[\forall y : A . G[ Q  ]]
}{
M_1 \cdots M_m \uplus y N_1 \cdots N_n \Rightarrow Q
}
\]

The rest of the transitions are a bit mundane in comparison, mostly just ensuring they use the correct
quantifier when necessary.  Thus, we exclude the presentation of the Gvar-Gvar, and Gvar-Uvar-Inside and Gvar-Uvar-Outside cases.

\begin{tcase}
Forall-And
\end{tcase}

While in a good implementation, this case is not necessary, we take note of it here as it is a potential source of 
bugs when implementing such a language.  Unfortunately moving the universal quantifier to
capture a conjunction is not as simple, since
if done incorrectly, existential variables might be able
to be defined with respect to universal quantifiers that they
were not previously in the scope of.

\[
F[(\forall x : A . E_1) \wedge E_2]
\UnifiesTo
F[\forall x : A . E_1 \wedge E_2]
\]
provided no existential variables are declared in $E_2$.

While this restriction prevents most application of this rule, 
equations can still be flattened to the form
\[
Qx_1:A_1\cdots Qx_n : A_n . M_1 \doteq N_1 \wedge M_m \doteq N_n
\]

transforming $E_2$ first with the Raising rule untill 
an Exists-And transformation is possible, then repeating  
until $E_2$ no longer contains any existentially 
quantified variables.  This process is always terminating,
although potentially significantly slower.   

\subsection{Implementation}

Because typed substitution is necessary, we must now keep track of existential variable's
types.  This can significantly complicate the implementation of the unification algorithm
as the common technique of maintaining unbound existential variables with restrictions
can no longer be blindly used, as existential variables must be maintained in the 
formula.  The most advisable option is to maintain the type of the existential variable 
with each mention of the existential variable.  

After experimentation, good performance has been observed when this structure is implemented
as a zipper \citep{huet1997functional}. We have found exceptional performence when implementing this structure
as a zipper using a finger tree indexed lookups. Unfortunately, since variables are best 
implemented via DeBruijn indexes, general variable reconstruction is no longer trivial.  It is fortunate that general variable reconstruction 
is not necessary in the final implementation since an existential variable representing the body is always used at the top level.

Another option is to perform unification with untyped substitution.
While there is no proof at the moment that unification on the pattern subset of 
the calculus of constructions with untyped substitution for only the existential substitutions
is total, it is not unbelievable. Furthermore, omitting typed substitution does not alter
the correctness of the algorithm, only the potential totality.  

Ideally, knowledge that type checking terminated would be convincing enough
so it is not necessary to continue with the reconstruction.  However, reconstruction
is necessary for implementing the multi-pass proof search described previously.  
Furthermore, reconstruction is usefull since the exposed typing rules do not admit 
coherence.  In these cases, it is desirable to see what was infered by type inference.


\section{Proof Search}

In a traditional logic programming language, the order of declaration of quantified arguments is irrelevant, 
and the context can be considered an unordered set (even though for implementation reasons it is not). 
In a dependently typed logic programming language where types direct proof search, types must be maintained in the context
and the context thus must maintain order.   Since search dynamically poses unification problems, which may not be 
entirely solvable until later in the search, unification and proof search are naturally mutually recursive procedures.
As it is important to maintain the mixed quantifier prefix thought proof search, it is desirable to provide a version 
of the algorithm where unification and proof search are not distinct procedures. 
\citep{pfenning1991logic} gave a succinct formulation where inhabitance and immediate implication were represented
directly in the unification calculus.  

\subsection{Proof Sharing}

\begin{definition}
Unification Calculus with Search

\[
U ::= U \wedge U 
 \orr \forall V : T . U
 \orr \exists V : T . U 
 \orr U \doteq U
 \orr \top
  \orr T \in T 
  \orr T \in T >> T \in T
\]

\end{definition}

The following new transformations are added to represent proof search:

\[
G_\Pi : M \in \Pi x : A . B   \rightarrow \forall x : A . \exists y : B . y \doteq M x \wedge y \in B
\]

\[
G^1_\m{Atom} : \forall x : A . F[M\in C]  \rightarrow \forall x : A . F[x \in A >> M \in C]
\]

\[
G^2_\m{Atom} : F[M\in C]  \rightarrow \forall x : A . F[c_0 \in A >> M \in C]
\]  where $c_0 : A$ is a constant

\[
D_\Pi : N\in \Pi x : A . B >> M \in C \rightarrow \exists x : A ( N x \in B >> M \in C) \wedge x \in A
\]

\[
D_\m{Atom} : N\in a N_1 \cdots N_n >> M \in a M_1 \cdots M_n \rightarrow N_1 \doteq M_1 \wedge \cdots \wedge N_n \doteq M_n \wedge N \doteq M
\]

\subsection{Proof Sharing}

In a pure setting, significant improvements to the efficiency of the system can be made by 
extending the quantifiers of the unification calculus to include forced inhabitant existential quantification.

\[
U ::= U \wedge U 
 \orr \forall V : T . U
 \orr \exists V : T . U 
 \orr \exists_f V : T . U 
 \orr U \doteq U
 \orr \top
 \orr T \in T >> T \in T
\]

\[
G^1_\m{Atom} : \forall x : A . F[\exists_f V : T . \top]  \rightarrow \forall x : A . F[x \in A >> M \in C]
\]

\[
G^2_\m{Atom} : \exists_f x : A . F[\exists_f V : T . \top]  \rightarrow \forall x : A . F[x \in A >> M \in C]
\]

\[
D_\Pi : N\in \Pi x : A . B >> M \in C \rightarrow \exists_f x : A ( N x \in B >> M \in C)
\]

In this situation, it is permitted to use the results of future searches for the solution of the current search.
While this sharing is optimal from an operational standpoint, it can make reasoning about the behavior 
of impure logic programs very difficult.  Given that Caledon is an impure programming language, reasoning about program
behavior comes before optimizing proof search.  It is the subject of future research to determine proof sharing teqchniques
that do not interfere with I/O. 

\section{Type Inference}

    
\chapter{Caledon Language}
In this chapter I discuss the some of the details of the 
specification details of the Caledon language.

\chapter{Metaprogramming with Caledon}
\chapter{Metaprogramming with Caledon}

\section{Typeclasses}


I've mentioned previously that implicit arguments along side polymorphism and proof search can subsume Haskell style type classes.

The easiest way to see this is through an implementation of what is known as the ``Show'' type class in Haskell.  
In a logic programming language, a predicate that can be used to print a datatype can also be used to read a datatype, so here we shall discuss a ``serialize'' type class.


\begin{figure}[H]
\begin{lstlisting}

defn serializeBool : bool -> string -> type
  >| serializeBool-true = serializeBool true ``true''
  >| serializeBool-false = serializeBool false ``false''

\end{lstlisting}
\caption{Serializing booleans}
\label{prog:serializing}
\end{figure}


\begin{figure}[H]
\begin{lstlisting}

query readQuery = exists B : bool. serializeBool B ``true''
query printQuery = exists S : string . serializeBool false S

\end{lstlisting}
\caption{Bidirectional serializing}
\label{prog:bidi}
\end{figure}

Notice that the predicate \ref{prog:serialize} in the sense that both of the queries in 
\ref{prog:bidi} will resolve. 

The serialize predicate is a useful one and we would like it to be polymorphic in all types for which we've implemented a serialize function.  This is possible using implicit arguments.

We'd first create an open type for the type class serializable.

\begin{figure}[H]
\begin{lstlisting}

open serializable : [T]{ serializer : T -> string -> type } type

\end{lstlisting}
\caption{The type of the type class serializable }
\label{prog:sty}
\end{figure}

We'd then define a function ``serialize'' which unpacks the the implicit dependency of the type serializable.


\begin{figure}[H]
\begin{lstlisting}

defn serialize : {T}{ serializable : T -> string -> type } T -> string -> type
   | serializeImp = 
       [ Serializer : T -> string -> type ]
       [ Serializable : serializable T { serializer = Serializer }]
       serialize { serializable = Serializable } V S
     <- Serializer V S

\end{lstlisting}
\caption{The implementation of the function serialize }
\label{prog:imp}
\end{figure}


\begin{figure}[H]
\begin{lstlisting}

instance serialize-bool = serializable bool { serializer = serializeBool }
instance serialize-nat = serializable nat { serializer = serializeNat }

\end{lstlisting}
\caption{ Instances of serializable }
\label{prog:inst}
\end{figure}


To implement an instance of the serializable type class, one would add an instance axiom to the environment as in \ref{prog:inst}

Use of the function would then omit the implementation of the ``serializable'' argument and type argument such that they might be resolved automatically as in \ref{prog:uses}


\begin{figure}[H]
\begin{lstlisting}

query readQueryBool = exists B . serialize B ``true''
query printQueryBool = exists S . serialize false S

query printQueryNat = exists S . serialize (succ (succ zero)) S
query readQueryNat = exists S : nat . serialize S ``(succ (succ zero))''

\end{lstlisting}
\caption{ Instances of serializable }
\label{prog:inst}
\end{figure}


This process can be extended to not only open type classes, but closed typeclasses where 
resolution involves arbitrary computation.  
While it is difficult to point to uses of this capability that can be discussed in these confines,
type class computation has been known to the Haskell community for quite some time and has been used in application ranging from embedding an imperative computation monad with local variable use and 
assignment rules similar to those of C, to an RPC framework which creates end points based on functions with arbitrarily complex type signatures.

\section{Flexible Syntax}
\subsection{Program transformations}

\section{DSLs}



\chapter{Conclusion}
\chapter{Conclusion}

\section{Results}


\section{Future Work}

Universe checking would be an ideal way to ensure separation of phases and allow compilation. 
Modal provability logics might also be used for this purpose.  
In general, running of these programs in the current implementation is excruciatingly slow, 
as types need to be recorded and searched during runtime.  
Algorithms that take advantage of totality checking \citep{altenkirch2010termination}, 
uniqueness checking \citep{anderson2004verifying}, 
worlds checking\citep{anderson2004verifying}, 
mode checking\citep{anderson2004verifying}, 
and universe checking \citep{harper1991type}, 
could be implemented and applied as they were for twelf and agda.  


\appendix
\include{appendix}

\backmatter

%\renewcommand{\baselinestretch}{1.0}\normalsize

% By default \bibsection is \chapter*, but we really want this to show
% up in the table of contents and pdf bookmarks.
\renewcommand{\bibsection}{\chapter{\bibname}}
%\newcommand{\bibpreamble}{This text goes between the ``Bibliography''
%  header and the actual list of references}
\bibliographystyle{plainnat}
\nocite{*}
\bibliography{MyBib} %your bib file

\end{document}
