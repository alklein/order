%for a more compact document, add the option openany to avoid
%starting all chapters on odd numbered pages

\documentclass[12pt]{cmuthesis}

% This is a template for a CMU thesis.  It is 18 pages without any content :-)
% The source for this is pulled from a variety of sources and people.
% Here's a partial list of people who may or may have not contributed:
%
%        bnoble   = Brian Noble
%        caruana  = Rich Caruana
%        colohan  = Chris Colohan
%        jab      = Justin Boyan
%        josullvn = Joseph O'Sullivan
%        jrs      = Jonathan Shewchuk
%        kosak    = Corey Kosak
%        mjz      = Matt Zekauskas (mattz@cs)
%        pdinda   = Peter Dinda
%        pfr      = Patrick Riley
%        dkoes = David Koes (me)

% My main contribution is putting everything into a single class files and small
% template since I prefer this to some complicated sprawling directory tree with
% makefiles.

% some useful packages
\usepackage{times}
\usepackage{fullpage}
\usepackage{graphicx}
\usepackage{amsmath}
\usepackage{verbatim} 

\usepackage{float}
\floatstyle{boxed}
\restylefloat{figure}
\usepackage{placeins} 

\renewcommand{\rmdefault}{ppl}
\renewcommand{\sfdefault}{phv}

\usepackage{dashrule}
\usepackage{proof-dashed}
\usepackage{verbatim}


\newtheorem{theorem}{Theorem}
\newtheorem{lemma}[theorem]{Lemma}
\newtheorem{definition}[theorem]{Definition}
\newtheorem{corollary}[theorem]{Corollary}

\newenvironment{proof}{\trivlist \item[\hskip \labelsep{\bf
Proof:}]}{\hfill$\Box$ \endtrivlist}
\newenvironment{attempt}{\trivlist \item[\hskip \labelsep{\bf
Proof attempt:}]}{\hfill$\Diamond$ \endtrivlist}

\newcommand{\ednote}[1]{\footnote{\it #1}\message{ednote!}}
\newenvironment{note}{\begin{quote}\message{note!}\it}{\end{quote}}
\newcommand{\highlight}[1]{\par\vspace{\abovedisplayskip}%
\framebox{\addtolength{\linewidth}{-1em}\begin{minipage}{\linewidth}#1\end{minipage}}%
\par\vspace{\belowdisplayskip}}



\usepackage{latexsym}
\usepackage{amssymb}            % for \multimap (-o)
\usepackage{stmaryrd}           % for \binampersand (&), \bindnasrepma (\paar)

\newcommand{\m}[1]{\mathsf{#1}}
\newcommand{\f}[1]{\framebox{#1}}
\newcommand{\DD}{\mathcal{D}}
\newcommand{\EE}{\mathcal{E}}
\newcommand{\FF}{\mathcal{F}}
\newcommand{\palign}[1]{\raisebox{-0.75em}{#1}}
\newcommand{\ih}[1]{\mbox{i.h.}(#1)}

% judgments of linear logic
\newcommand{\eph}{\mathit{eph}}
\newcommand{\pers}{\mathit{pers}}
\newcommand{\um}[1]{\underline{\m{#1}}}

\newcommand{\seq}{\vdash}
\newcommand{\cfseq}{\Rightarrow}
\newcommand{\cseq}{\rightarrow}
\newcommand{\semi}{\mathrel{;}}
\newcommand{\lequiv}{\mathrel{\dashv\vdash}}
\newcommand{\cut}{\m{cut}}
\newcommand{\cutbang}{\m{cut}{!}}
\newcommand{\id}{\m{id}}
\newcommand{\defeq}{\triangleq}
\newcommand{\vvdash}{\mathrel{\vdash\kern-0.8ex\vdash}}

% symbols of linear logic
\newcommand{\lolli}{\multimap}
\newcommand{\tensor}{\otimes}
\newcommand{\with}{\mathbin{\binampersand}}
\newcommand{\paar}{\mathbin{\bindnasrepma}}
\newcommand{\one}{\mathbf{1}}
\newcommand{\zero}{\mathbf{0}}
\newcommand{\bang}{{!}}
\newcommand{\whynot}{{?}}
\newcommand{\bilolli}{\mathrel{\raisebox{1pt}{\ensuremath{\scriptstyle\circ}}{\lolli}}}
% \oplus, \top, \bot

% symbols of pi-calculus
\newcommand{\FN}{\mathsf{fn}} % free names
% \mid for parallel composition
\newcommand{\recv}[2]{#1(#2)}
\newcommand{\send}[2]{\overline{#1}\langle #2\rangle}
% \newcommand{\zero}{\boldsymbol{0}} % already above
\newcommand{\fwd}{\leftrightarrow}
\newcommand{\case}{\mathsf{case}}
\newcommand{\inl}{\mathsf{inl}}
\newcommand{\inr}{\mathsf{inr}}

% symbols of linear lambda-calculus
\newcommand{\pair}[2]{\langle #1, #2\rangle}
\newcommand{\llet}{\m{let}\;}
\newcommand{\iin}{\mathrel{\m{in}}}
\newcommand{\ccase}{\m{case}\;}
\newcommand{\oof}{\mathrel{\m{of}}}
\newcommand{\unit}{\langle\,\rangle}
\newcommand{\abort}{\m{abort}}

% substructural operational semantics
\newcommand{\eval}{\m{eval}}
\newcommand{\retn}{\m{retn}}
\newcommand{\cont}{\m{cont}}
\newcommand{\hole}{\mbox{\tt\char`\_}}
\newcommand{\rep}[1]{\ulcorner #1\urcorner}

% ordered logic
\newcommand{\gnab}{\mbox{!`}}
\newcommand{\fuse}{\bullet}
\newcommand{\esuf}{\circ}
\newcommand{\rimp}{\twoheadrightarrow}
\newcommand{\limp}{\rightarrowtail}


\usepackage[numbers,sort]{natbib}

\usepackage[backref,pageanchor=true,plainpages=false, pdfpagelabels, bookmarks,bookmarksnumbered,
%pdfborder=0 0 0,  %removes outlines around hyper links in online display
]{hyperref}
\usepackage{subfigure}
  
\usepackage{bussproofs}

\usepackage{color}
\usepackage{listings}
\lstset{ %
language=Haskell,               % choose the language of the code
basicstyle=\footnotesize,       % the size of the fonts that are used for the code
numbers=left,                   % where to put the line-numbers
numberstyle=\footnotesize,      % the size of the fonts that are used for the line-numbers
stepnumber=1,                   % the step between two line-numbers. If it is 1 each line will be numbered
numbersep=5pt,                  % how far the line-numbers are from the code
backgroundcolor=\color{white},  % choose the background color. You must add \usepackage{color}
showspaces=false,               % show spaces adding particular underscores
showstringspaces=false,         % underline spaces within strings
showtabs=false,                 % show tabs within strings adding particular underscores
frame=none,           % adds a frame around the code
tabsize=2,          % sets default tabsize to 2 spaces
captionpos=b,           % sets the caption-position to bottom
breaklines=true,        % sets automatic line breaking
breakatwhitespace=false,    % sets if automatic breaks should only happen at whitespace
escapeinside={\%*}{*)}          % if you want to add a comment within your code
}

 
% Approximately 1" margins, more space on binding side
%\usepackage[letterpaper,twoside,vscale=.8,hscale=.75,nomarginpar]{geometry}
%for general printing (not binding)
\usepackage[letterpaper,twoside,vscale=.8,hscale=.75,nomarginpar,hmarginratio=1:1]{geometry}

% Provides a draft mark at the top of the document. 
\draftstamp{\today}{DRAFT}

  
\begin{document} 
\frontmatter

%initialize page style, so contents come out right (see bot) -mjz
\pagestyle{empty}

\title{ 
{\bf Logic Programming and Type Inference with the Calculus of Constructions }}
\author{Matthew Mirman}
\date{May 2013}
\Year{2013}
\trnumber{}

\committee{
Frank Pfenning \\
Karl Crary
}

\support{}
\disclaimer{}

\keywords{Logic Programming, Pure Type System, 
Type Inference, Higher Order Unification, Caledon Language, Higher Order Abstract Syntax,
Metaprogramming, Universe Checking}

\maketitle

\begin{dedication}
For my grandfather.
\end{dedication}

\pagestyle{plain} % for toc, was empty

%% Obviously, it's probably a good idea to break the various sections of your thesis
%% into different files and input them into this file...

\begin{abstract}

In this thesis I present a logic programming language, Caledon, with a pure type system
and a turing complete type inference and implicit argument system based on the semantics of the
object language.  Because the language has dependent types and type inference, terms can be generated 
by providing type constraints.  The addition of control structures such as implicits, costructor hiding,
shared holes, existential quantification, polymorphism, and nondeterminism make the language ideal for 
creating libraries for defining EDSLs.  Furthermore, the system can be made consistent and sound with 
the addition of totality, worlds, and universe checks. Proof irrelivance, universes, and uniqueness 
checking can be used to constrain the nondeterminism of type inference in a pure type system, and help 
programmers control coherence.



\end{abstract}

\begin{acknowledgments}
I thank my advisor, Frank Pfenning for listening patiently to all my outlandish ideas.
\end{acknowledgments}

\tableofcontents
\mainmatter

%% Double space document for easy review:
\renewcommand{\baselinestretch}{1.66}\normalsize

% The other requirements Catherine has:
%
%  - avoid large margins.  She wants the thesis to use fewer pages, 
%    especially if it requires colour printing.
%
%  - The thesis should be formatted for double-sided printing.  This
%    means that all chapters, acknowledgements, table of contents, etc.
%    should start on odd numbered (right facing) pages.
%
%  - You need to use the department standard tech report title page.  I
%    have tried to ensure that the title page here conforms to this
%    standard.       
%
%  - Use a nice serif font, such as Times Roman.  Sans serif looks bad.
%
% Other than that, just make it look good...

\newtheorem{tcase}{Case}
      
\chapter{Introduction}
    ``To be fair, even in last century's typed languages, 
types had a beneficial organisational effect on programmers. 
This century, it's just possible types will have a comparable effect on programs. 
Types are concepts and now mechanisms supporting program-discovery as well as error-discovery. 
I think that's more than just gravy.''
 - Conor McBride


    \section{Logic Programming}

Logic programming languages such as Prolog were originally designed as part of the AI
program, in much the same way Lisp was. Automated reasoning’s natural goal would
be to be able to arbitrarily prove theorems. A logic programming language would be
a set of axioms and a predicate, and if the predicate could be proven through those axioms,
the automated theorem prover would halt. These proof search procedures were
then constrained into useful programming semantics. When performed in a backtracking
manner, the process of proof search represented an formulation of procedural code
with powerful pattern matching. The Caledon language is a higher order backtracking
logic programming language in the style of Twelf. In this section I present some basic
intution for logic programming, rather than explaining it technically, and demonstrate the 
descriptive power of the system implemented in Caledon.

\FloatBarrier
\subsection{Basics}
We begin by defining addition on unary numbers in Caledon shown in shown in \ref{code:add}.

\begin{figure}[H]
\begin{lstlisting}
defn add : nat -> nat -> nat -> prop
   | addZ = add zero A A
   | addS = add (succ A) B (succ C) 
             <- add A B C
\end{lstlisting}
\caption{Addition in Caledon}
\label{code:add}
\end{figure}

One might notice that this definition is incredibly similar to its Haskell counterpart shown in \ref{code:hask}.

\begin{figure}[H]
\begin{lstlisting}
add :: nat -> nat -> nat
add Zero a = a
add (Succ a) b = Succ c
   where c = add a b
\end{lstlisting}
\caption{Addition in Haskell}
\label{code:hask}
\end{figure}

We can read the logic programming definition as we would read the functional definition with pattern
matching, knowing that an intelligent compiler would be able to convert the first into the second.  
search allows one to define essentially nondeterministic programs. A common use for
logic programming has been to search for solutions to combinatorial games such as tic-tac-toe, 
without the programmer worrying about the order of the search. As this tends
to produce ineficient code, this use style is discouraged. Rather, a more procedural view of logic programming is encouraged where pattern match and search is performed
in the order it appears.

\begin{figure}[H]
\begin{lstlisting}
defn p : T_1 -> ... -> T_r -> prop
  >| n1 = p T_1 ... T_r <- p_1,1 ... <- p_1,k_1
...
  >| nN = p T_1 ... T_r <- p_n,1 ... <- p_n,k_n

query prg = p t1 ... tr
\end{lstlisting}
\caption{Format of a Caledon Logic Program}
\label{code:format}
\end{figure}

In this view, a program of the form \ref{code:format}
should be considered a program which first attempts to prove using axiom n1 by matching prg with ``p T1 ... Tr'' and then
attempting to prove $p_{1,1}$ and so on.


\FloatBarrier
\subsection{Higher Order Logic Programming}

Just as imperative programs benefit from the addition of higher order functions, logic programs benefit from the addition of both
higher order predicates and patterns.  

A common request by functional programming language users is that they would like to be able to abstract patterns even more than just over
arguments.  

\begin{figure}[H]
\begin{lstlisting}
func (Var a) = code1
func (Forall var val) = Exists var code
func (Exists var val) = Forall var code
func (And a b) = code2
\end{lstlisting}
\caption{Unecessarily verbose code}
\label{code:verbose}
\end{figure}

A serious amount of code is repeated in lines 2 and 3 of example \ref{code:verbose}, 
but in common languages it is impossible to simplify this.
What a programmer would actually like to express is shown in \ref{code:Fideal}.

\begin{figure}[H]
\begin{lstlisting}
func (Var a) = code1
func (f var val) = f var code
func (And a b) = code2
\end{lstlisting}
\caption{Ideal code. $f$ is a constructor in strong head normal form.}
\label{code:Fideal}
\end{figure}

In higher order logic programming languages like $\lambda$Prolog, this sort of simplification 
is in fact possible to express as shown in \ref{code:lprolog}.

\begin{figure}[H]
\begin{lstlisting}
defn func : term -> term -> prop 
  | f1 = func (var A) R <- [code1]
  | f2 = func (F Var Val) (f Var R) <- [code]
  | f3 = func (and A B) R <- [code2]
\end{lstlisting}
\caption{Expressing constructor variables in patterns}
\label{code:lprolog}
\end{figure}

\FloatBarrier
\subsection{Higher Order Programming}

Fortunately, higher order functions need not be restricted to patterns.  Macros provide even more ways to generalize code. 

A great example is the function application operator from Haskell.  
We can define this in Caledon as shown in \ref{code:macros}

\begin{figure}[H]
\begin{lstlisting}
fixity right 0 @
defn @ : (At -> Bt) -> At -> Bt
  as ?\ At Bt . \ f : At -> Bt . \ a : At . f a

\end{lstlisting}
\caption{Definitions for expressive syntax}
\label{code:macros}
\end{figure}

In many cases, allowing these definitions allows for significant simplification of syntax.
The reader familiar with languages like Twelf, Haskell, and Agda might notice the 
implicit abstraction of the type variables At and Bt in the type of @ in \ref{code:macros}. The rest of this
paper is concerned with formalizing these implicit abstractions and letting them have
as much power as possible. For example, one might make these abstractions explicit by
instead declaring at the beginning \ref{code:expHask}

\begin{figure}[H]
\begin{lstlisting}
infixr 0 @
(@) :: forall At Bt . (At -> Bt) -> At -> Bt
(@) = \ f . \ a . f a
\end{lstlisting}
\caption{Explicit Haskell style abstractions}
\label{code:expHask}
\end{figure}

However, in a dependently typed language, every function type is a dependent product
(forall).  This makes it necessary to provide a new (explicit) implicit dependent product - $?\forall$ or $?\Pi$.


\begin{figure}[H]
\begin{lstlisting}
fixity right 0 @
defn @ : {At Bt:prop} (At -> Bt) -> At -> Bt
  as ?\ At Bt . \ f . \ a . f a

\end{lstlisting}
\caption{The (explicit) implicit equivalent of \ref{code:macros}}
\label{code:expimp}
\end{figure}

Haskell also has type classes. For example, the type of “show” can be seen in \ref{code:showty}.

\begin{figure}[H]
\begin{lstlisting}
show :: Show a => a -> String
\end{lstlisting}
\caption{The type of show}
\label{code:showty}
\end{figure}

In Caledon, these can be written similarly as in \ref{code:cshowty}

\begin{figure}[H]
\begin{lstlisting}
defn show : showC A => A -> string
defn show : {unused : showC A } A -> string
\end{lstlisting}
\caption{Equivalent types for show in Caledon}
\label{code:cshowty}
\end{figure}

However, since implicit arguments are a natural extension of the dependent type
system in Caledon, no restrictions are made on the number of arguments, or difficulty
of computing. Unfortunately, since computation is primarily accomplished by the logic
programming fragment of the language rather than the functional fragment of the language,
the correspondence between these programmable implicit arguments and type
classes is not one to one. It is possible to replicate virtually all of the functionality of
type classes in the implicit argument system, but the syntax required to do so can become
verbose. Rather than attempting to simulate type classes, more creative uses are
possible, such as computing the symbolic derivative of a type for use in a (albeit, slow
and unnecessary) generic zipper library, or writing programs that compile differently
with different types in different environments.




\section{Initial Examples}


In the previous section I gave an introduction to the notion of logic programming using
both the familiar language of Haskell and the new language of Caledon. In this section
I will build upon these ideas by introducing logic programming with polymorphism
through building a set of standard polymorphic type logic library.

There are a few ways of defining sums in Caledon.


\begin{figure}[H]
\begin{lstlisting}
defn and : type -> type -> type
   | pair = and A B <- A <- B

defn fst : and A B -> A -> type
   | fstImp = fst (pair Av Bv) Av

defn snd : and A B -> B -> type
   | sndImp = snd (pair Av Bv) Bv
\end{lstlisting}
\caption{Logical conjunction}
\label{code:lconj}
\end{figure}


In this first, simplest way (as seen in figure \ref{coe:lconj}, we define a predicate for “and” and
predicates for construction and projection. This method has the advantage of doubling
as a form of sequential predicate.

\begin{figure}[H]
\begin{lstlisting}

query main = and (print ``hello ``) (print `` world!'')
\end{lstlisting}
\caption{Use of logical conjunction}
\label{code:lconjuse}
\end{figure}

In the figure \ref{code:lconjuse} the query will output ``hello world!''.

\begin{figure}[H]
\begin{lstlisting}

defn and : type -> type -> type
  as \ a : type . \ b : type . 
      [ c : type ] (a -> b -> c) -> c

defn pair : A -> B -> and A B
  as ?\ A B : type .
      \ a b .
      \ c : type .
      \ proj : A -> B -> c .
        proj a b

defn fst : and A B -> A
  as ?\ A B : type .
      \ pair : [c : type] (A -> B -> c) -> c .
        pair A (\ a b . a)

defn snd : and A B -> A
  as ?\ A B : type .
      \ pair : [c : type] (A -> B -> c) -> c .
        pair B (\ a b . b)

\end{lstlisting}
\caption{Church style conjunction}
\label{code:cconj}
\end{figure}

In the case of figure \ref{code:cconj}, we do not add any axioms without their proofs.   
In this example we also introduce the dependent product written in the form $[ a : t_1 ] t_2$.

This case mimics the version usually seen in the Calculus of Constructions and has the advantage of
the projections being functions rather than predicates.


\begin{figure}[H]
\begin{lstlisting}

defn churchList : type −> type
  as \A : type . [ lst : type] lst −> (A −> lst −> lst ) −> lst

defn consCL : [ B : type ] B -> churchList B -> churchList B
  as \ C : type .
     \ V : C .
     \ cl : churchList C .
     \ lst : type .
     \ nil : lst .
     \ cons : C -> lst -> lst .
     cons V (cl lst nil cons)

defn nilCL : [ B : type ] churchList B
  as \ C : type .
     \ lst : type .
     \ nil : lst .
     \ cons : C -> lst -> lst .
       nil

defn mapCL : { A B } ( A -> B) -> churchList A -> churchList B
  as ?\ A B : type .
      \ F : A -> B.
      \ cl : churchList A .
      \ lst : type .
      \ nil : lst .
      \ cons : B -> lst -> lst .
        cl lst nil (\ v . cons (F v))

defn foldrCL : { A B } ( A -> B -> A) -> A -> churchList B -> A 
  as ?\ A B : type . 
      \ F : A -> B -> A .
      \ bc : A .
      \ cl : churchList B .
        cl A bc (\ v : B . \ c : A . F c v)

\end{lstlisting}
\caption{Church style list}
\label{code:clist}
\end{figure}

In the Church form of a list, folds and maps are possible to implement as functions
rather than predicates. However, their implementation is verbose and doesn’t permit
more complex functions.


\begin{figure}[H]
\begin{lstlisting}

defn list : type -> type
   | nil = list A
   | cons = A -> list A -> list A

defn concatList : list A -> list A -> list A -> type
   | concatListNil = [ L : list A ] concatList nil L L
   | concatListCons = 
          concatList (cons (V : T) A) B (cons V C) 
       <- concatList A B C

defn concatList : list A -> list A -> list A -> type
   | concatListNil = [ L : list A ] concatList nil L L
   | concatListCons = 
          concatList (cons (V : T) A) B (cons V C) 
       <- concatList A B C

defn mapList : (A -> B -> type ) -> list A -> list B -> type
   | mapListNil = [ F : A -> B ] mapList F nil nil
   | mapListCons = [F : A -> B ] 
            mapList F (cons V L) (cons (F V) L')
            <- mapList F L L'

defn pmapList : (A -> B -> type) -> list A -> list B -> type
   | pmapListNil = [ F : A -> B -> type] pmapList F nil nil
   | pmapListCons = [F : A -> B -> type] 
            pmapList F (cons V L) (cons V' L')
            <- F V V'
            <- mapList F L L'
\end{lstlisting}
\caption{Logic List}
\label{code:llist}
\end{figure}

The logic programming version can be seen in figure \ref{code:llist}. It is important to note that
we can now map a predicate over a list rather than just mapping a function over a list.

     
\chapter{Type System}
    In this section I introduce the specifics of the \textit{Caledon Implicit Calculus of Constructions} ($CICC$).
The internal Caledon type system is an extension of the well known Calculus of Constructions with the 
addition of implicit bindings, and explicit type constraints for implicit instantiation.  
While the inspiration for this comes from the theorem prover Agda, it appears as though no formal treatment
has been provided. In this section provides a background on pure type systems, the history of the calculus of constructions
and introduces a formal defintion and treatment of $CICC$ with $\eta$ conversions for the purpose of type checking
and proof search.

The type system of Caledon is designed after two different formalisms for working with implicit arguments:  
the Bicolored Calculus of Constructions ($CC^{Bi}$) \citep{luther2001more}, 
the Implicit Calculus of Constructions ($ICC$) \citep{miquel2001implicit}.

It is comprised of two parts: The Caledon Implicit Calculus of Constructions ($CICC$), 
and the Implicit Caledon Implicit Calculus of Constructions ($CICC^-$).

$CICC$ is a custom modification of $CC^{Bi}$ with Church style binders and explicit constraints 
where alpha conversion is not available for certain binders.

$CICC^-$ Is intended to be a combination of the first two, or a partial 
erasure system for $CICC$. 

    \section{Pure Type Systems}

The type system for caledon is a pure type system \citep{mckinna1993pure} 
extended with explicit recursive types and implicit types.  In this section,
I discuss what a pure type system is and what its properties are.

Pure type systems are generalizations of the lambda cube
\citep{barendregt1991introduction} which allow for arbitrary 
relationships between terms and types.
With proper selection of constants, sorts, axioms, and relations,
pure type systems can embed the ``Calculus of Constructions'' \citep{coquand1986calculus}
and many other type systems one might want to construct.

As generalizations, these systems are important, as it has been proven by Jutting \citep{jutting1993typing} that 
type checking for normalizing pure type systems with finite axiom sets are decidable.  Thus, by showing how a system is a pure type system and is normalizing, 
you get decidability of type checking nearly for free.

It has also been shown that these systems have utility.  
Roorda \citep{roorda2001pure} gave an implementation of a functional programming language with 
pure type system and demonstrated its utility.

A pure type system is a set $S$ of sorts, 
$A\subseteq S \times S$ of axioms, and a relation 
$R \subseteq S \times S \times S$ along with the following grammar and inference rules:

\begin{definition}
\textbf{(PTS Syntax)}
\[ 
E ::=  V 
 \orr S 
 \orr E\;E 
 \orr \lambda V : E . E 
 \orr \Pi V : E . E 
\]

\label{pt:syntax}
\end{definition}

\begin{definition}

\textbf{(PTS Typing Rules)}

\[ \begin{array}{lr}
\infer[\m{WF-E}]
{
\cdot \vdash
}
{}
&
\infer[\m{WF-S}]
{
\Gamma, x : T \vdash
}
{
\Gamma \vdash T : s
&
x \notin DV(\Gamma)
}
\end{array} \]

\[
\infer[\m{axioms}]
{
\Gamma \vdash c : s
}
{
\Gamma \vdash
&
(c,s) \in A
}
\]

\[
\infer[\m{start}]
{
\Gamma,x:A \vdash x :A
}
{
\Gamma \vdash A:s
&
s \in S
}
\]

\[
\infer[\m{weakening}]
{
\Gamma,x:C \vdash A:B
}
{
\Gamma \vdash A:B
&
\Gamma \vdash C:s
&
s \in S
}
\]


\[
\infer[\m{product}]
{
\Gamma \vdash (\Pi x : A . B) : s_3
}
{
\Gamma \vdash A : s_1
&
\Gamma,x:A \vdash B : s_2
&
(s_1,s_2,s_3) \in R
}
\]

\[
\infer[\m{application}]
{
\Gamma \vdash F V : [V/x] B
}
{
\Gamma \vdash F : (\Pi x : A . B)
&
\Gamma \vdash V : A
&
\text{V is free for x in B}
}
\]

\[
\infer[\m{abstraction}]
{
\Gamma \vdash (\lambda x : A . F) : (\Pi x : A . B)
}
{
\Gamma , x : A\vdash F : B
&
\Gamma \vdash (\Pi x : A . B) : s
&
s \in S
}
\]

\[
\infer[\m{conversion}]
{
\Gamma \vdash A : B'
}
{
\Gamma \vdash A : B
&
\Gamma \vdash B \equiv_{\beta\eta\nu*} B'
&
\Gamma \vdash B' : s
&
s \in S
}
\]

\label{pt:typing}
\end{definition}


As Barendgregt\citep{barendregt1991introduction} points out, the common type theories can be recast as pure type systems
by choice of axioms.  
In the simplest example, the only axioms chosen are $(*,\Box)$ along with 
the single relationship $(*,*,*)$. This system describes the simply typed lambda calculus, 
where only terms can depend on terms.  We say that $A \rightarrow B \equiv \Pi x : A . B$ iff $ x \notin FV(B)$.

\begin{theorem} 
Subject Reduction: If $\Gamma \vdash A : T$ and $A \Rightarrow_\beta B$ then $\Gamma \vdash B : T$
\end{theorem}

Geuvers and Nederhof \citep{geuvers1991modular} proved subject reduction for any calculus on the $\lambda$ cube.
This property can be proved syntactically by induction on the structure of the typing derivation 
and there exist Twelf and Agda verified proofs of this property.  
Note that this is a useful property 
to maintain, even in the face of inconsistency of a system, because at the very least, the 
property allows for a consistent understanding of typing terms.

\begin{theorem}
Uniqueness of Types: If $\Gamma \vdash A : T$ and $\Gamma \vdash A : T'$ then $T \equiv_\beta T'$
\end{theorem}

The uniqueness of types with respect to $\beta$ reduction has also been shown for any system on the $\lambda$ cube.  
This last property is important to showing the decidability of type inference in 
the caledon language without implicits.



\begin{lemma}
Strengthening
\[
\infer-[\m{strength}]
{
\Gamma \vdash M : U
}
{
\Gamma , x : T \vdash M : U
&
x \notin FV(M) \cup FV(U)
}
\]
\end{lemma}

As it turns out, the strengthening lemma has important implications to the generation of bindings 
during proof search.
 
    \section{The Calculus of Constructions}

The type system for caledon is based on the Calculus of Constructions as defined by Coquand \citep{coquand1986calculus}.
Since caledon might extend the calculus of constructions, it is important to view it as a pure type system. 
As Barendgregt points out, the common type theories can be recast as pure type systems
by choice of axioms. 

Jutting \citep{jutting1993typing} gave a proof that both typechecking and inference
in a pure type system with a finite axiom set is decidable.

Roorda \citep{roorda2001pure} gave an implementation of a functional programming language with 
pure type system and demonstrated its utility.

This is the pure type system where.

\begin{definition}
\textbf{(PTS for $CC$)}

\[
A = \{ *, \Box \}
\]

\[
S = \{ (* : \Box) \}
\]

\[ 
R = \{ (*,*,*),(*,\Box,\Box),(\Box,\Box,\Box),(\Box,*,*)\}
\]  
\label{coc:types}
\end{definition}

In this sytem, terms can depend on terms and types, 
and types can depend on types and terms.  

It has the well known strong normalization property, implying the termination of 
all lambda terms typable by $CC$ \citep{Geuvers94ashort} \citep{geuvers1991modular}.

It is necessary to be carefull with the types of equalities allowed in the conversion rule, 
as with more allowed equalities certain proofs
become significantly more complex.  For example, if $\eta$ reduction is permited, 
the church rosser theorem becomes no longer easily provable.

\begin{definition}
If $\Gamma \vdash_{CC} P : T : K$ means $\Gamma \vdash_{CC} P : T$ and $\Gamma \vdash_{CC} T : K$
\end{definition}


\subsection{Consistency of the Calculus of Constructions}

It is useful to first describe a few theorems and definitions relevant to $CC$.  

It is important to show that we can classify \citep{Geuvers94ashort} terms of $CC$.

\begin{definition}
Classifications:

\[
\m{Kind} := \{ A | \exists \Gamma . \Gamma \vdash_{CC} A : \Box \}
\]

\[
\m{Type} := \{ A | \exists \Gamma . \Gamma \vdash_{CC} A : * \}
\]

\[
\m{Constr} := \{ A | \exists T,\Gamma . \Gamma \vdash_{CC} A : T:\Box \}
\]

\[
\m{Obj} := \{ A | \exists T,\Gamma . \Gamma \vdash_{CC} A : T:* \}
\]

\end{definition}

\begin{theorem}
Classification:
$\m{Kind}\cap \m{Type} = \emptyset$ 
and
$\m{Constr}\cap \m{Obj} = \emptyset$ 
\end{theorem}

Write proof here.

As Geuvers points out, this can also proved using the Church-Rosser theorem, 
Subject Reduction and Uniqueness of Types theorem.  
The theorem defines an important technique for breaking down judgements into
different cases.

\begin{definition}
$ \m{Term}_{CC}  = \{ M : \exists T,\Gamma . \Gamma \vdash_{cc} M : T \}$
\end{definition}

\begin{theorem}
\textbf{(Strong Normalization)} $\forall M \in \m{Term}_{CC}. SN(M)$
\end{theorem}

The easiest to digest proof is also due to Geuvers \citep{Geuvers94ashort} 
but other proofs have also been proposed.  Geuvers' proof has the convenient
property that it does not depend too much on the definition of the set $SN$ 
(strongly normalizing). The only properties required are that $S \subseteq SN$ 
where $S$ is the set of sorts in the system. In the case of $CC$ $\Box,* \in SN$.
It also requires that $\Pi x : A . B \in SN$, and $\lambda x : A . B \in SN$ 
for any $A,B$.  Lastly it requires that saturated subsets of $SN$ are closed under
arbitrary intersection, and function space generation.

However, this proof is restricted to normalization in the calculus where only $\beta$ reduction 
is considered and not $\eta$ conversion.  The proof for the calculus with full reduction properties is
in Guevers' thesis \citep{geuvers1993logics}.  While evaluation of terms in Caledon without considering proof search does
not involve $\eta$ conversion, unification makes significant use of $\eta$ reduction and it is important to note
the consistency of the calculus of constructions even when $\eta$ reduction is considered.  
It is also important to note that in the Curry style calculus where types are omited from lambda abstractions, 
the church rosser theorem under $\eta$ conversion appears essentially for free\citep{miquel2001implicit}. Strong normalization 
follows, and by replacement of types, strong normalization follows for the Tarski style calculus.

\subsection{Impredicativity in the Calculus of Constructions}

It is also important to note that while the term language of the calculus of constructions is strongly normalizing, 
the predicates in the calculus of constructions are impredicative, meaning that small types can be generalized over all small types.
This adds yet another layer of computation into the Caledon language which might not terminate.  
Furthermore, the predicate language is insufficient to prove properties of larger types. 
Luo \citep{luo1989ecc} solves this by introducing universes into the Extended Calculus of Constructions.  
As defined by Luo, the Extended Calculus of Constructions is no longer a pure type system.

Harper \citep{harper1991type} provided an analysis of a few pure type systems with universes.
Agda, Idris, CoQ,
and Plastic \citep{callaghan2001implementation} all implement this teqchnique. 
Idris implementation uses a topological sort to allow for implicit universes
and thus does not allow the universes to become an annoyance in code. 
As Caledon is not intended to be a theorem prover, the impredicativity of the theorem layer is ignored and no 
predicative universe heirarchy is provided to construct metatheorems natively.

With the addition of inconsistent impredicative inductive types, it is possible to provide an implementation 
of the universe heirachy as a higher order datatype who's inhabitants are consistent.  This kind of implementation 
allows for metatheorem verification in a restricted subset of the language, albeit with the loss of 
simple cumulative universe inference.

\begin{lstlisting}
fixity pre 1 %* $\diamond$ *)
fixity lambda %* $\Pi$ *)
fixity lambda lam
unsound tm : {S : tm ty} tm S %* $\rightarrow$ *) prop
   | ty  = tm ty
   | %* $\diamond $ *)   = tm ty %* $\rightarrow$ *) tm ty
   | %* $\Pi$ *)   = [T : tm ty] (tm T %* $\rightarrow$ *) tm T) %* $\rightarrow$ *) tm $ %* $\diamond $ *) T
   | lam = [T : tm ty][F : tm T %* $\rightarrow$ *) tm T] tm {S = %* $\diamond$ *) T} (%* $\Pi$ *) A : T . F A)
   | raise = tm T %* $\rightarrow$ *) tm $ %* $\diamond$ *) T
\end{lstlisting}

\subsection{Theorems in Caledon}

It is important to note that in the programming language caledon, programs might not terminate.  
The search language itself is not consistent, and is not a theorem verification language like twelf.  
Rather, it is language for writing theorem proving programs.  

\begin{definition}
In the Caledon language, $\m{prop} = *$ and $\m{type} = \Box$
\end{definition}

It is important to note that in Caledon code, $\m{type}$ will never appear.  This is because $\m{type} : T$ for any $T$
is not provable in the Calculus of Constructions, and every term that appears in the language must have a provable type.

\begin{definition}
If $\Gamma \vdash_{CICC} P : T : \m{prop}$ in the caledon language, then $T$ is a theorem, and $P$ is a proof.
\end{definition}

Terms can be considered this way since the consistency of the calculus of constructions implies the strong normalization
of terms such as $P$.  No unbounded proof search is necessary to evaluate $P$.

When $CC$ was first developed, therems were proven and generated by explicitly defining
constructors and destructors for records and sum types.  Later, the inductive calculus of constructions was developed 
\citep{coquand1990inductively} which more accurately forms the basis of the CoQ programming language.  These types of inductive
constructions have been omited from Caledon to allow the addition of inductively defined, universally quantified predicates.
This omits the confusion that would be generated from having both inductively defined data and predicates in the system, as 
dependently typed logic programming treats predicates as both data and code.

Unfortunately, simple $CC$ does not make for a powerfull theorem proving language.  This can significantly 
limit the utility of powerfull theorem searching techniques.

The original calculus of constructions with a universe heirarchy included an impredicative type.

\begin{definition}
\textbf{(PTS for $CC_\omega$)}

\[
A = \{ \m{prop} \} \cup \{ \m{type}_i | i \in \mathbb{N} \}
\]

\[
S =   \{ (\m{prop} : \m{type}_0 ) \}
 \cup \{ (\m{type}_i : \m{type}_{i+1} ) | i \in \mathbb{N} \}
\]

\[ 
R = \{ (\m{type}_j,\m{type}_i,\m{type}_k) | j \leq k \wedge i \leq k \}
\cup \{ (\m{prop},t,t) | t \in A \}
\cup \{ (t,\m{prop},t) | t \in A \}
\]

\label{coc:types}
\end{definition}

While type checking with the addition of the impredicative universe is theoretically equivalently as difficult 
as without it, in practice it turns out to not be very usefull for proof or program writing, and 
languages like CoQ, Agda and Idris now omit it for the system with only predicative universes.

The addition of a universe heirachy into the Caledon Language is possible without too much added difficulty. 
Experimentation shows that the omitting the predicative $\m{prop}$ allows for a simpler implementation of type inference.
In this case, typechecking and unification are performed with the usually inconsistent assumption that
$\m{type} : \m{type}$, and a cycle checker is later applied to ensure that there is no instance of 
$\m{type}_i : \m{type}_i$ necessary.  

In the case where an impredicative universe is included, the extra axiom $\m{prop} : \m{type}$ would
need to be included, which would mean searching among two possibilities, rather than a single possibility 
when attempting to find a type for a metatheoretic hole.  The added nondeterminism tends to cause
an explosion in the time complexity of type inference.
    \section{Caledon Implicit Calculus of Constructions}

Caledon's exposed type system is a varient of $ICC$,
which for the rest of the paper I will refer to as $CICC^{-}$.  
This sytem contains two products and two binders - one each for implicit and explicit arguments. 
Unlike $ICC$, there is no rule that allows for an unmarked value to obtain an implicit product type, which
makes type checking somewhat simpler.  
In addition, there is a new form of application to allow for the explicit
selection of an implicit argument to constrain. 
As implicit quantification is permitted on object level variables, semantics must be given with respect to 
the elaborated theory without implicit application.

\begin{figure}[H]
\[ 
E ::= 
V 
\orr S 
\orr E\;E 
\orr \lambda V : T. E 
\orr ?\lambda V : T. E 
\orr \Pi V : E . E 
\orr ?\Pi V : E . E 
\orr E \{ V : E = E \}
\]

\caption{Syntax of $CICC$}
\label{cicc:syntax}
\end{figure}

The \textit{non-dependent} explicit and implicit products are written $T \rightarrow T$ 
and $T \Rightarrow T$ respectively.

Note, that in $CICC$ system, $?\lambda x . A \neg\equiv_\alpha ?\lambda y . [x / y] A$ 
and $?\Pi x . A \neg\equiv_\alpha ?\Pi y . [x / y] A$  if $x \neq y$.  
This implies that the behavior of caledon's implicit argument is something akin to a structural dependent product.  

Before the typing rules of this system can be given, the notion of a constrained name of a term must be defined.

\begin{definition}
The constrained names on a term, written $CN(M)$ is a set defined as follows:

\[
CN(M \{ x = E \}) = \{ x \} \cup CN(M)
\]

\[ 
CN(\m{otherwise}) = \emptyset
\]

\end{definition}

\begin{definition}
The generalized names for a term, written $GN(M)$ is a set defined as follows:

\[ 
GN(?\Pi x : T . M) = \{ x \} \cup GN(M) \cup GN(T)
\]

\[ 
GN(\m{otherwise}) = \emptyset
\]

\end{definition}

\begin{definition}
The bound names for a term, written $BN(M)$ is a set defined as follows:

\[ 
BN(?\lambda x : T . M) = \{ x \} \cup BN(M) \cup BN(T)
\]

\[ 
BN(\m{otherwise}) = \emptyset
\]

\end{definition}

As defined above, bound names and bound variables can no longer be treated the same in the semantics.  
Specifically, $?\lambda x : A . B$ does not have the same semantics as $?\lambda y : A . [y / x] B$.  
This implies that alpha conversion is now severely limited.  

There are a few ways to deal with this.  
The most attractive possibility is to interpret names as a kind of record modifier.
This can be seen as saying $\{ x : T = N \} : \{ x : N \}$, 
and $?\lambda x : N . B$ is really just $\lambda y : \{ x : N \} . [ y.x / x ] N$ where $ .x : \{ x : N \} \rightarrow N$.

\begin{figure}[H]
\begin{lstlisting}
defn nat : prop 
  as [a : prop] a -> (a -> a) -> a

defn nat_1 : nat -> prop
  as \ N : nat . [a : nat -> prop] a zero -> succty a -> a N

defn rec : nat -> prop -> prop
  as \ nm : nat . \ N : kind . nat_1 nm * N

defn get : [N : kind] [nm : nat] nat_1 nm * N -> N
  as \ N : kind . \nm : nat . \ c : (string2 nm, N) . snd c

defn put : [N : kind] [nm : nat] nat_1 nm ->  N -> nat_1 nm * N
  as \ N : kind . \nm : nat . \nmnm : nat_1 nm . \ c : N . pair nmnm N
\end{lstlisting}
\caption{Definitions for extraction}
\label{code:ideal}
\end{figure}

We can further convert this into traditional dependent types by constructing
 type invariants as seen in \ref{code:ideal}.

Then $?\lambda x : A . B$ and $?\Pi x : A . B$ 
becomes $\lambda y : \m{rec}\;\bar{x}\;A . [get A \bar{x} y / x ] B$
and $?\Pi y : \m{rec}\;\bar{x}\;A . [get A \bar{x} y / x ] B$.

Similarly, $N \{ x\; : \; T \; = \;A \} $
would become $N\;( \m{put} \; T \; \bar{x} \; \bar{\bar{x}} \; A)$.

One might notice that $N$ in $\m{get}$ is of type $\m{kind}$.  
In simple $CC$, this is unfortunately not an actual type. 
Rather, it refers to the use of either $\m{type}$ or $\m{prop}$.  
Allowing $N : \m{type}$ is not permited in the standard $CC$ since it is the subject of quantification.
This is possible in $CC_\omega$ however.
In this case, $\m{kind}$ would always refer to the next universe after the highest
universe mentioned in the program.
In the Caledon language implementing simple $CC$ we are free to define 
$\m{kind}$ as $\m{type}_1$, since it will always be larger than any
type or kind mentioned in a Caledon program.  
Fortunately, Geuvers' proof \citep{geuvers1993logics} of strong normalization in the presence of 
$\eta$ conversion applies to the Calculus of Construction with one impredicative universe and two predicative
universes.

This intuitive conversion leads to the following typing rules for $CICC$.

\begin{definition}
\textbf{(Typing for $CICC$)}

\[ \begin{array}{lr}
\infer[\m{wf/e}]
{
\cdot \vdash_{ci} 
}{}
&
\infer[\m{wf/s}]
{
\Gamma, x : T \vdash_{ci} 
}
{
\Gamma \vdash_{ci} T : s
&
x \notin DV(\Gamma)
}
\end{array} \]

%% axioms %%
%%%%%%%%%%%%
\[
\infer[\m{axioms}]
{
\Gamma \vdash_{ci} c : s
}
{
\Gamma \vdash_{ci}
&
(c,s) \in A
}
\]

%% start %%
%%%%%%%%%%%
\[
\infer[\m{start}]
{
\Gamma,x:A \vdash_{ci} x :A
}
{
\Gamma \vdash_{ci} A:s
&
s \in S
}
\]

%% prod %%
%%%%%%%%%%
\[
\infer[\m{prod}]
{
\Gamma \vdash (\Pi x : A . B) : s_3
}
{
\Gamma \vdash A : s_1
&
\Gamma,x:A \vdash B : s_2
&
(s_1,s_2,s_3) \in R
}
\]

%% prod* %%
%%%%%%%%%%%
\[
\infer[\m{prod}*]
{
\Gamma \vdash (?\Pi x : A . B) : s_3
}
{
\Gamma \vdash A : s_1
&
\Gamma,x:A \vdash B : s_2
&
(s_1,s_2,s_3) \in R
}
\]

%% gen %%
%%%%%%%%%
\[
\infer[\m{gen}]
{
\Gamma \vdash_{ci} \lambda x : T . M : (\Pi x : T . U)
}
{
\Gamma , x : T \vdash_{ci} M : U
&
\Gamma \vdash_{ci} (\Pi x : T . U) : s
&
s \in S
&
x \notin FV(M) \cup BN(M) \cup GN(U)
}
\]

%% gen* %%
%%%%%%%%%%
\[
\infer[\m{gen}*]
{
\Gamma \vdash_{ci} ?\lambda x : T . M : (?\Pi x : T . U)
}
{
\Gamma , x : T \vdash_{ci} M : U
&
\Gamma \vdash_{ci} (?\Pi x : T . U) : s
&
s \in S
&
x \notin FV(M) \cup BN(M) \cup GN(U)
}
\]

%% app %%
%%%%%%%%%
\[
\infer[\m{app}]
{
\Gamma \vdash_{ci} M N : U [N/x]
}
{
\Gamma \vdash_{ci} M : \Pi x : T . U
&
\Gamma \vdash_{ci} N : T
}
\]

%% inst/b %%
%%%%%%%%%%%%
\[
\infer[\m{inst/b}]
{
\Gamma \vdash_{ci} M \{ x : T = N \} : U [N/x]
}
{
\Gamma \vdash_{ci} M : ?\Pi x : T . U
&
\Gamma \vdash_{ci} N : T
& 
x \notin GN(M)
&
x \notin BN(U)
}
\]

\label{cicc:typing}
\end{definition}


\subsection{Main Results}

Most of the theorems relating to $CICC$ can be obtained by a simple projection into $CC$.

\newcommand{\CICCproj}[1]{ \left\llbracket #1 \right\rrbracket_{ci}}

\begin{definition}
\textbf{ (Projection from $CICC$ to $CC$) }

\[
\CICCproj{v} := v
\]

\[
\CICCproj{s} := s
\]

\[
\CICCproj{E_1 \; E_2} := \CICCproj{E_1} \; \CICCproj{E_2}
\]

\[
\CICCproj{E_1 \; \{ x : T = E \}} := \CICCproj{E_1} \; (\m{put}\;\CICCproj{T}\;\bar{x}\; \bar{\bar{x}}; \CICCproj{E_2} )
\]

\[
\CICCproj{\lambda v : T . E } := \lambda v : \CICCproj{T} . \CICCproj{E}
\]

\[
\CICCproj{?\lambda v : T . E } := \lambda y : \m{rec}\;\bar{v}\; \CICCproj{T} . \CICCproj{ [ \m{get}\; \CICCproj{T}\; \bar{v}\; y  / v ] E}
\;\text{ where $y$ is fresh}
\] 

\[
\CICCproj{\Pi v : T . E } := \Pi v : \CICCproj{T} . \CICCproj{E}
\]

\[
\CICCproj{?\Pi v : T . E } := \Pi y : \m{rec}\;\bar{v}\;\CICCproj{T} . \CICCproj{ [ \m{get}\;\CICCproj{T}\; \bar{v}\; y  / v ] E}
\;\text{ where $y$ is fresh}
\]

\label{cicc:proj}
\end{definition}

It is significant that church numerals be used for the representation of the name in the record, 
as no extra axioms need to be included in the context of the translation for the translation to be valid.  
This necessity is seen in \ref{ci:sound}.

\begin{lemma}

If $\Gamma \vdash_{ci} A : T$ then $\Gamma \vdash_{ci}$

\label{ci:wfctxt}
\end{lemma}

\begin{lemma}

If $\Gamma \vdash_{ci} A : T$ then $\Gamma \vdash_{ci} T : s$ for some sort $s$

\label{ci:wtt}
\end{lemma}

\begin{theorem}

\textbf{(Soundness of extraction)}  

\begin{alignat}{4}
\Gamma &\vdash_{ci}&  & \implies & \CICCproj{\Gamma} & \vdash_{cc} &
\\
\Gamma &\vdash_{ci}& A : T & \implies & \CICCproj{\Gamma} & \vdash_{cc} & \CICCproj{A} : \CICCproj{ T }
\end{alignat}

\label{ci:sound}
\end{theorem}

It is easy to see how this proof follows from the 
definition \ref{cicc:proj}, and the two lemmas \ref{ci:wfctxt} and \ref{ci:wtt}.

\begin{definition}
$ \m{Term}_{ci}  = \{ M : \exists T,\Gamma . \Gamma \vdash_{ci} M : T \}$
\end{definition}

\begin{theorem}
\textbf{(Consistency)}  $\not \exists M \in \m{Term}_{ci}. \vdash M : \Pi x : \m{prop} . x$
\label{ci:cons}
\end{theorem}

This follows clearly from the soundness of the extraction, 
\ref{ci:sound}, the consistency of $CC$ \ref{cc:cons}, and the fact that we 
have obviously more possible reductions in $CC$ than in $CICC$.  

Before we can prove the strong normalization theorem, we need to show that reductions in the extracted calculus 
correspond to reductions in $CICC$. 

\begin{theorem}
\textbf{(Semantic Equivalence)} 
$\forall M \in \m{Term}_{cicc}$ such that $\CICCproj{M} \rightarrow_{\beta\eta*} N'$
implies 
$\exists M' \in \m{Term}_{cicc}$ such that $M \rightarrow_{\beta\eta\alpha} M'$ and $\CICCproj{M'} \equiv N'$
\label{ci:se}
\end{theorem}

While the proof here is somewhat technical, the result is obvious, as the extraction and calculus
have been written so as to ensure that the result is true.

\begin{theorem}
\textbf{(Subject Reduction)} If $\Gamma \vdash_{ci} M : T$ and $M \rightarrow_{\beta\eta\alpha*} M'$ then $\Gamma \vdash_{ci} M' : T$
\end{theorem}

In non dependent type theories, this theorem is rather trivial.  However, in $CC$, this theorem depends heavily on the Church-Rosser 
theorem.  Fortunately, $CICC$ can be considered a bicoloring of the syntax of $CC$, and the proof of subject reduction 
for $CC$ can be leveraged to $CICC$ using a forgetfull extraction.

\begin{theorem}
\textbf{(Strong Normalization)} $\forall M \in \m{Term}_{ci}. SN(M)$
\label{ci:sn}
\end{theorem}

That we can cleanly translate into the calculus of constructions without loss or gain of 
semantical translation implies strong normalization for $CICC$ with $\beta\eta$ equivalence, 
and $\eta$ expansion.  This is the most important theorem of the section, as it implies that typchecking a Caledon 
statement will allow that statement to be compiled to pattern form to be used in proof search.  
Without this theorem typechecking is a weak property and valid programs might not be useable.  
While type safety in the presence of the strong normalization theorem does not ensure bounded proof search, 
it does ensure bounded unification.
The strong normalization theorem is to logic programming languages as 
the progress theorem is to functional programming languages.

That $CICC$ is simply an extention of $CC$ and not a modification of $CC$ implies we have the completeness theorem, \ref{ci:comp}.

\begin{theorem}
\textbf{(Completeness)}  $\forall M,T \in \m{Term}_{cc}. \vdash_{cc} M : T \implies \vdash_{ci} M : T$
\label{ci:comp}
\end{theorem}

This theorem is trivial since the syntax of $CC$ is a subset of the syntax of $CICC$.

  
\chapter{Semantics}
    In this chapter I lay out and justify a specification and semantics 
for Caledon.

    \section{History}

Huet \citep{Huet75} gave the first semi-decision algorithm for unification of 
terms in the lambda calculus.  
Later, Miller \citep{miller1986higher} proved
that for terms in the pattern fragment of the lambda calculus, unification was decidable.  
Pfenning \citep{pfenning1988partial} \citep{pfenning1988higher} demonstrated unification 
for the typed lambda calculus and considered solving the dynamic pattern fragment where non pattern equations
are postponed.
Elliott\citep{elliott1989higher} gave a more efficient algorithm for unification in the context 
of dependent types. Later Pfenning \citep{pfenning1991unification}
did the same thing for unification in the ``Calculus of Constructions'', although without a mixed prefix.  
The most succinct presentation is from the 1991 paper describing the workings of Twelf 
\citep{pfenning1991logic}.  While the unification algorithm implemented in the interpreter for Caledon is 
an extension of that presented in Pfenning's 1991 paper on unification for 
the ``Calculus of Constructions''\citep{pfenning1991unification}, 
I briefly cover here the main ideas from the presentation of the paper describing the workings of Twelf, 
and extend those ideas later.



    \section{Forms for Unification}

\begin{definition}
Spine Form
\begin{align}
N &::= P
   \orr \lambda V : N . N 
\\
P &::= V 
  \orr P N 
\end{align}
\label{def:spine}
\end{definition}

Note that we will write $\Pi V : N . P$ as a synonm for 
$\Pi\; N \; (\lambda V : N . P)$ in the rest of the paper.
This simplifies the presentation of the unification algorithm, 
as then $\Pi$ can be considered a traditional constructor
that can also be used to direct the unification procedure.

Spine terms have the incredibly useful property that they are always in head normal form, 
meaning that the head of every term is a constructor, 
and every argument is either a constructor or lambda term.

\subsection{Higher Order Patterns}

While spine form is restrictive enough that its terms are always in head normal form, 
it is not yet restrictive enough for unification problems to be decidable.  
Miller \citep{miller1991logic} showed that for any unification instance given in 
the pattern fragment shown in \ref{def:pattern}, unification is decidable.  

Pattern form is specified with respect to partial permuations $\phi$, 
which are injective mappings from finite domains to finite domains.

\newcommand{\Pat}{\;\m{ Pat }\;}
\begin{definition}
Pattern Form:  Note that $\Delta$ is the existential context and 
$\Gamma$ is the universal context.

\[
\infer[\m{P/ABS}]{
\Delta ; \Gamma \vdash \lambda u : A . M \Pat
}{
\Delta ; \Gamma \vdash A \Pat
&
\Delta ; \Gamma,u \vdash M \Pat
} \]


\[ \begin{array}[2]{lr}
\infer[\m{P/CON}]{
\Delta ; \Gamma \vdash c \;M_1\cdots M_m \Pat
}{
\Delta ; \Gamma \vdash M_1 \Pat
&
\cdots
&
\Delta ; \Gamma \vdash M_m \Pat
}
&
\infer[\m{P/VAR}]{
\Delta ; \Gamma \vdash u \;M_1\cdots M_m \Pat
}{
\Delta ; \Gamma \vdash M_1 \Pat
&
\cdots
&
\Delta ; \Gamma \vdash M_m \Pat
&
u \in \Gamma
}
\end{array} \]


\[ \begin{array}[2]{lr}
\infer[\m{P/PROP}]{
\Delta ; u_1 ,\cdots u_p 
\vdash x \;u_{\phi(1)}\cdots u_{\phi(m)} \Pat
}{
\phi \text{ is a partial permutation}
&
x \in \Delta
}
&
\infer[\m{P/VAR}]{
\Delta ; \Gamma \vdash M \Pat
}{
\Delta ; \Gamma \vdash M' \Pat
&
M \equiv_{\eta} M'
}
\end{array} \]

\label{def:pattern}
\end{definition}


\subsection{Canonical Forms}

Pfenning's \citep{pfenning1991unification} unification algorithm for the 
calculus of constructions, which the metatheory of Caledon is based on, 
relies on the fact that expressions are also presented in 
\textit{long $\beta\eta$-normal form} (or \textit{canonical form}), 
meaning that they are $\eta$ expanded to conform to their type signature.  
In the initial publication of this paper, it was taken as a hypothesis that 
every well-typed term in $CC$ has a unique $\beta$-normal $\eta$-long form.  This is now known
to be the case \citep{abel2010towards}.

\newcommand{\FormFor}{\;\Rightarrow\;}
\begin{definition}
Canonical Forms

\[ \begin{array}{lr}
\infer[\m{F/ax}]{
\Gamma \vdash s_1 \FormFor s_2
}{
(s_1,s_2) \in A
}
&
\infer[\m{F/prod}]{
\Gamma \vdash \Pi x : A . B \FormFor s_3
}{
\Gamma \vdash A \FormFor s_1
&
\Gamma, x : A \vdash B \FormFor s_2
&
(s_1,s_2,s_3) \in R
}
\end{array} \]

\[
\infer[\m{F/lam}]{
\Gamma \vdash \lambda x : A . M \FormFor \lambda x : A . B
}{
\Gamma,x : A\vdash M \FormFor B
&
\Gamma\vdash A \FormFor s
} 
\]

\[
\infer[\m{F/app}]{
\Gamma \vdash h\;M_1 \cdots M_n \FormFor D
}{
\Gamma \vdash h\;M_1 \cdots M_n : D
&
\Gamma \vdash M_1 \FormFor A_1
&
\cdots
&
\Gamma \vdash M_n \FormFor A_n
&
} 
\]
where $D$ is atomic
\label{def:canonical}
\end{definition}

It has been proven that the standard calculus of constructions is 
long $\beta\eta$-normal form strongly normalizing.  
Unfortunately, normalzation into this form is not possible without type 
information.  Later, a typed substitution algorithm will be given which 
ensures normalization into this form.  

The notions of canonical form and of a higher order pattern are also trivially
extendable into church-style $CC^{Bi}$ (ie, ICC* from \citep{barras2008implicit}), 
where stong normalization into long $\beta\eta$-normal form is also provable, as is shown by Bernardo.

    \section{Substitution}

The higher order unification algorithm described is only defined on the pattern form provided in the previous section.
As the pattern fragment is a restriction on $\beta\eta$ long-normal form, it is necessary to provide a normalizing
substitution that preserves $\beta\eta$ long-normal form.  The presentation here
revolves around heredetary substitution, a notion first described by Pfenning et al. \citep{pfenning1991logic} for
a calculus with only $\beta$ reduction.

Other presentations are possible, such as traditional substitution followed by Normalization by Evaluation 
for $\beta\eta$ conversions\citep{abel2010towards}. 
While this is a proven total method in a typed setting, it's mechanics
are complex and not particularly enlightening.  
Keller et al. \citep{keller2010normalization}
extended Heredetary substitution to $\eta$ expansion, but only in the simply typed case. 
The presentation here extends this version into $CC$.

\subsection{Untyped Substitution}

\begin{definition}
\textbf{(Hereditary Substitution)} 

\[ \begin{array}[3]{llr}
[ S / x ] x := S
&
[S / x] y := y
&
[S / x] P\;T := \m{H}([S/x] P, [S/x]T)
\end{array} \]

\[
[S / x] \lambda v : T . P := \lambda v' : [S/x]T . [S/x][v'/v]P
\;
\text{  where $v'$ is new.}
\]

\[ 
\m{H}(\lambda v : T . P , A) := [A/v] P
\]

\[ \begin{array}[2]{lr}
\m{H}(P A_1 , A_2) := P\;A_1\;A_2
&
\m{H}(V , A) := V\; A
\end{array} \]

\label{def:shered}
\end{definition}

It is important to note the alpha conversion in the $\lambda$ case, as alpha conversion will be lost on 
some terms when implicits are added.

A hereditary substitution is not necessarily terminating as is shown by
substitution replicating the $\omega$-combinator.

\[
[(\lambda x : T . x\; x) / x ] ( x \; (\lambda x : T. x\; x) )
\]

This is not defined, as it expands to the well known $\m{H}(\lambda x . x \; x , \lambda x . x \; x)$.

If the pattern and substitution are well typed terms in the Calculus of Constructions, by 
strong normalization, this version of hereditary substitution is total.


\begin{theorem} Substitution Theorem:

If $\Gamma,x : T \vdash A : T' : \m{prop}$  and $\Gamma \vdash S : T : \m{prop}$ then
$ \Gamma \vdash [S/x]_o A : [S/x]_o T' : \m{prop}$
\end{theorem}

By consistency, $[S/x]_o A $ can be normalized to strong head normal form.  Thus, the 
hereditary substitution $[S/x] A$ is defined.



\subsection{Typed Substitution}

Performing substitutions that maintain long $\beta\eta$-normal 
form is important to ensuring decidability of unification.  
Unfortunately this is not possible without some type information, 
as arbitrary $\eta$ expansion has no stop condition.
Keller \citep{keller2010normalization} gave a heredetary substitution algorithm
that results in canonical forms for simply typed lambda calculus. 
\ref{def:tyhered} solves this by performing a typed substitution under a context, 
generating type information.

\begin{theorem}
\textbf{(Reduction Decomposition)}
If $\Gamma \vdash A \Rightarrow_{\beta\eta*} B$ then there exists some $C$ such that
$A \Rightarrow_{\beta*} C$ and $\Gamma \vdash C \Rightarrow_{\eta*} B$

\label{thm:betaeta}
\end{theorem}

The property in \ref{thm:betaeta} stating that any reduction can be 
shown equivalent to first a series of $\beta$ reductions and then a series of
$\eta$ expansions forms the basis of the following algorithm.

For this section it is convenient to include $\Pi$ in the spine form: 
\[
N ::= P 
\orr \lambda V : N . N 
\orr \Pi V : N . N 
\]


\begin{definition}
\textbf{(Typed Hereditary Substitution)}

\[
[S / x : A]^n_{\Gamma } P := \m{E}([S / x : A]^p_{\Gamma } P)
\]

\[
[S / x : A]^n_{\Gamma } (\lambda y : B . N) := \lambda y:B . [S / x : A]^n_{\Gamma, y : B} N
\]

\[
\m{E}^M(M \uparrow P) := M
\]

\[
\m{E}(N \downarrow A) := \eta^{-1}_A (N)
\]

\[
\eta^{-1}_{\Pi x : A . B}(N) := \lambda z : A . N \; \eta^{-1}_A(z)
\] where $z$ is fresh

\[
\eta^{-1}_{P}(N) := N
\]

\[ 
[ N / x : A]^p_{\Gamma} x := N \uparrow A
\]

\[
[S / x : A]^p_{\Gamma} y := y \downarrow \Gamma(x)
\] 

\[
[S / x : A]^p_{\Gamma } P\;N := 
\m{H}_{\Gamma} ( [S/x : A]^p_{\Gamma} P , [S/x : A]^n_{\Gamma} N) 
\]

\[
\m{H}_{\Gamma} ((\lambda v : A_1 . N) \uparrow \Pi v' : A_1 . A_2 , P) 
:= [P/v : A_1]^n_{\Gamma} N \uparrow [P/v']^n A_2
\]

\[
\m{H}_{\Gamma}(P \downarrow \Pi y : B_1 . B_2 , N) := P\; N \downarrow [N/y]^nB_2
\]

\label{def:tyhered}
\end{definition} 

In these last two rules, when initializing the universal type, it is important that the substitution be non typed
or $\eta$ expanding as this would lead to an unecessary circularity.  Types need not be in $\eta$-long form
when used for $\eta$ expanding in substitution since they are not unified with.

While we could omit the distinction between $\uparrow$ and $\downarrow$ from
these rules, it would in practice result in excess computation.

Also, note that in the hereditary part of the recurrence, 
it is possible to omit cases for improperly formated terms.  $A \downarrow T$ has the invarient that
$A$ must be already in a cononical form within the context it is used since it existed previously in the
equation.  Similarly, $A \uparrow T$ has the invarient that $A$ must be locally in a canonical form.
This permits significant amounts of reduction of steps.

The usefull property of this hereditary substitution is spelled in \ref{thm:hed-sound}

\begin{theorem}
\textbf{(Soundness of Heredetary Substitution)}
If  $\Gamma \vdash S : A$ 
and $\Gamma \vdash x : A$ 
and $\Gamma \vdash S \; \m{pattern}$ 
and $\Gamma \vdash N \; \m{pattern}$ 
and $x$ is free for $S$ in $N$
and $[S / x : A]^n N$ is defined, 
then
and $\Gamma \vdash ([S / x : A]^n N) \; \m{pattern}$ 
\label{thm:hed-sound}
\end{theorem}

In general, it is provable that for any PTS that is strongly normalizing for $\beta$ reduction and 
$\eta$ expansion, this algorithm will terminate and substitution will be defined.  
This is a direct consequence of \ref{thm:betaeta}.

    \section{Dynamics}

Unification lies at the heart of the semantics of the Caledon language.  
In this section we present the syntax of the unification problems, as well as a modified version of the algorithm 
presented in \citep{pfenning1991logic} and \citep{pfenning1991unification} suited for implicit argument search.

Checking for the reducability of two full lambda terms has long been known to be only semidecidable.  
The matter becomes even more complicated when checking for the equality of terms with variables bound
by both existential and universal quantifiers.  Research from the past thirty years has constrained
the problem to a decidable subset known as the pattern fragment.

\subsection{Unification Terms}

\begin{definition}
Unification Terms:

\[
U ::= U \wedge U 
 \orr \forall V : T . U
 \orr \exists V : T . U 
 \orr T \doteq T
 \orr T \in T
 \orr \top
\]
\label{def:hou:syn}
\end{definition}

When $\doteq$ is taken to mean $\equiv_{\beta\eta\alpha*}$, the unification problem is to determine 
whether a statement $U$ is ``true'' in the common sense, and provide a proof of the truth of the statement. 

While we do not discuss the semantics of $T \in T$ or $T \in T >> T \in T$ here, 
one can refer to \citep{pfenning1991logic} for a complete presentation.

Unification problems of the form 
$\forall x : T_1 . \exists y : T_2 . U $ can be converted to the form
$\exists y : \Pi x : T_1 . T_2 . \forall x : T_1 . [y\; x / y ]U $ 
in the process known as raising. Unification
statements can always quantified over unused variables: $U \implies Q x : T . U$.  

Thus, statements can always be converted to the form
\[
\exists y_1 \cdots y_n . \forall x_1 \cdots x_k . S_1 \doteq V_1 \wedge \cdots S_r \doteq V_r
\]

\subsection{Typed Implicit Hereditary Substitution}

Before we can properly specify the semantics of this language, 
we must define typed hereditary substitution for this calculus.  This is important, as higher order unification for the calculus of constructions 
requires that terms maintain $\eta$-long normal form after substitutions have been performed.
Here, we do describe typed hereditary substituion in full, as a more complete presentation can be found in \citep{keller2010normalization}.

The formulation of hereditary substitution in the presence of 
implicit arguments is not that unlike the presentation of
hereditary substitution without implicit arguments \citep{miller1991uniform}, 
but for the additional checks required.
The main difficulty is the notion of an acceptable substitution. 
Because implicit bindings are not $\alpha$ convertible, 
certain substitutions are not permitted.  
Because as many substitions should be permitted as possible, 
the situation becomes significantly more complex in the 
hereditary case, where substitutions might not carry types.  
The easiest way to define substitution in this case is with an ``illegal'' alpha substitution, 
which maps implicitly bound variables to fresh names, 
and produces a memory to map them back. 


In this case, we can say that a substitution $[S/x] M$ is legal if 
$FV(S) \subseteq FV(\alpha_I^-1( [\alpha_I(S)/x] M) )$.

\begin{definition}
\textbf{(Implicit Typed Hereditary Substitution)}

\[
[S / x : A]^n_{\Gamma } (?\lambda y : B . N) := ?\lambda y:B . [S / x : A]^n_{\Gamma, y : B} N
\] 

\[
\eta^{-1}_{?\Pi x : A . B}(N) := ?\lambda x : A . N \; \{ x = \eta^{-1}_A(x) \}
\] since $N$ being typable by $?\Pi x $ means that $x$ can not appear free in $N$

\[
\m{H}_{\Gamma}(P \downarrow ?\Pi y : B_1 . B_2 , \{ v := N \} ) := P\; \{ v := N \} \downarrow [N/y : B_1]^n_{\Gamma}B_2
\]

\[
\m{H}_{\Gamma} ((?\lambda v : A_1 . N) \uparrow ?\Pi v : A_1 . A_2 , \{ v := P \}) 
:= [P/v]^n_{\Gamma \vdash v : A_1} N \uparrow A_2
\]

\[ 
\m{H}(?\lambda v : T . P \uparrow \_ , A) := ?\lambda v : T . \m{H}(P,A)
\]

\label{def:hered}
\end{definition}


\subsection{Unification Term Meaning}

We can provide an provability relation of a unification formula
based on the obvious logic.

\begin{definition}
$\Gamma \Vdash F $ can be interpreted as $\Gamma$ implies $F$ 
is provable.

\[ \begin{array}{lr}
\infer[\m{equiv}]{
\Gamma \Vdash M \doteq N
}{
\Gamma \vdash M : A
&
M \equiv_{\beta\eta\alpha*} N
&
\Gamma \vdash N : A
}
&
\infer[\m{true}]{
\Gamma \Vdash \top
}{}
\end{array} \]

\[
\infer[\m{conj}]{
\Gamma \Vdash F \wedge G
}{
\Gamma \Vdash F
&
\Gamma \Vdash G
}
\]

\[ \begin{array}{lr}
\infer[\m{exists}]{
\Gamma \Vdash \exists x : A . F
}{
\Gamma \Vdash [M/x] F
&
\Gamma \vdash M : A
}
&
\infer[\m{forall}]{
\Gamma \Vdash \forall x : A . F
}{
\Gamma, x : A \Vdash F
}
\end{array} \]

\label{def:hou:prf}
\end{definition}

While a truly superb logic programming language might 
be able to convert this very declarative 
specification into a runnable program, 
the essentially nondeterministic rule for existential
quantification in a unification formula prevents an 
obvious deterministic algorithm from being extracted.


\subsection{Higher Order Unification for CC}

\newcommand{\UnifiesTo}{\;\longrightarrow\;}

We now present an algorithm, similar to that presented in 
\citep{pfenning1991logic} for unification in the 
Calculus of Constructions.  Because we have already 
presented typed hereditary substitution with $\eta$-expansion, 
the presentation here will not add much 
but for types in the substitutions.  

$F \UnifiesTo F'$ shall mean that $F$ can be transformed to $F'$
without modifying the provability. 
An equation $F[G]$ will stand as notation for highlighting $G$
under the formulae context $F$.  
As an example, if we were to examine the formula 
$\forall x . \forall n . \exists y . ( y \doteq x \wedge \forall z . \exists r . [ x z \doteq r] )$
but were only interested in the last portion, we might instead write it as
$\forall x . F[\forall z . \exists r . [ x z \doteq r]]$
Again, $\phi$ shall be an injective partial permutation. 

Furthermore, rather than explicitly writing down the result of unification, 
we shall use $\exists x. F \UnifiesTo \exists x . [ L / x] F$ 
to stand for $\exists x. F \UnifiesTo \exists x . x \doteq L \wedge [ L / x] F$

The unification rules are symetric, so any rule of the form 
$M \doteq N \sim N \doteq M$ practically.

Also, for the purpose of typed normalizing heredetary substitution, 
a formula prefix $F[e]$ of the form 
$Qx_1:A_1 . E_1\wedge \cdots Qx_n : A_n . e$ shall be considered as a context
$x_1 : A_1 ,\cdots ,x_n : A_n$ when written $\nu^-1(F)$.

\setcounter{tcase}{0}

\begin{tcase}
Lam-Any
\end{tcase}

\[
F[\lambda x : A . M \doteq N]
\UnifiesTo
F[\forall x : A . M \doteq \m{H}_{\nu^-1(F),x:A}(N , x)]
\]

Because application is normalizing, this can cover the case where both $N$ is also a $\lambda$ 
abstraction.

\begin{tcase}
Lam-Lam
\end{tcase}

\[
\lambda x : A . M \doteq \lambda x : A . N
\UnifiesTo
\forall x : A . M \doteq N
\]

While this rule is not explicitly necessary as it is covered by the Lam-Any rule, 
when working
in a substitutive system with explicit names rather than DeBruijn indexes, 
this helps to reduce the number of substitutions from an original name. 

Lastly, these reductions make the 
assumption that no variable name is bound more than once.
This can seem restrictive, but it is possible to 
work past by alpha converting everywhere and annotating
new variables with their original names, and alpha converting
back to the original after unification. The other option is again
to use DeBruijn indexes.  DeBruijn indexes have their own drawbacks
here, as certain transformation such as ``Raising''
or the ``Gvar-Uvar'' rules involve insertion of multiple
variables into the context at an arbitrary point, 
which requiring the lifting of many variable names.  
It is possible to implement higher order unification
with DeBruijn indexes safetly and efficiently, 
but this is out of the scope of the thesis.

\begin{tcase}
Ilam-Ilam-Same
\end{tcase}

This case behaves just as the lam-lam case does, but only for implicit abstractions with the same names.

\begin{tcase}
Ilam-Other
\end{tcase}

if the constraint is of the form $F[(?\lambda x : A . M) A_1 \cdots A_n \doteq R]$, where $x$ is not constrained
in a prefix of $R$, we transition to 
\[
F[\exists x' : A .  x' \in A \wedge H(\cdots , H([x' / x : A]_F M, A_1), \cdots A_n) 
\]
\[
\doteq H(R, \{ x : A = x' \}) ]
\]

\begin{tcase}
Iforall-Iforall-Same
\end{tcase}

Because universal quantification is also subject to these subtyping rules, we do a similar thing to what 
we do in the case of implicit lambda abstraction: if the names match on both side, we unify.  

\begin{tcase}
Iforall-Other
\end{tcase}

This case is a bit different however, since in the constraint $F[?\Pi x : A . M \doteq R]$, $R$ is no longer an implicit
abstraction - rather it should be a type.  Here, we simply transition to 

\[
F[\exists x' : A .  x' \in A \wedge H(\cdots , H([x' / x : A]_F M, A_1), \cdots A_n) \doteq R ]
\]

\begin{tcase}
Uvar-Uvar
\end{tcase}

In the traditional case without implicit constraints, we have the following transition:

\[
F[\forall y : A . G[y M \doteq y N  ]]
\UnifiesTo
F[\forall y : A . G[ M_1 \doteq \wedge N_1 \cdots]]
\]

However, when implicit constraints are permitted, the same universally quantified variables might take different numbers of arguments.  
In this case, we must remove argument the unnecessary implicit constraints.  

We might then be presented with the following constraint: 
\[
F[\forall y : A . G[y M_1 \cdots M_m \doteq y N_1 \cdots N_n  ]]
\]
We define the following matching function $\uplus$


\[
\infer{
M_i M \uplus N_j N
\Rightarrow 
M_i \doteq N_i \wedge (M \uplus N)
}{
M_i \neq \{ a : A = B \}
&
N_i \neq \{ a : A = B \}
}
\]

\[
\{ a : A = M_i \} M \uplus \{ a : A' = N_j \} N
\Rightarrow
M_i \doteq N_i \wedge (M \uplus N)
\]


\[
\infer{
\{ a : A = M_i \} M \uplus N
\Rightarrow 
M \uplus N
}{
a \notin CN(N)
}
\]

Using this dropping match, we can define the transition as follows.

\[
\infer{ 
F[\forall y : A . G[y M_1 \cdots M_m \doteq y N_1 \cdots N_n  ]]
\UnifiesTo
F[\forall y : A . G[ Q  ]]
}{
M_1 \cdots M_m \uplus y N_1 \cdots N_n \Rightarrow Q
}
\]

The rest of the transitions are a bit mundane in comparison, mostly just ensuring they use the correct
quantifier when necessary.  Thus, we exclude the presentation of the Gvar-Gvar, and Gvar-Uvar-Inside and Gvar-Uvar-Outside cases.

\begin{tcase}
Forall-And
\end{tcase}

While in a good implementation, this case is not necessary, we take note of it here as it is a potential source of 
bugs when implementing such a language.  Unfortunately moving the universal quantifier to
capture a conjunction is not as simple, since
if done incorrectly, existential variables might be able
to be defined with respect to universal quantifiers that they
were not previously in the scope of.

\[
F[(\forall x : A . E_1) \wedge E_2]
\UnifiesTo
F[\forall x : A . E_1 \wedge E_2]
\]
provided no existential variables are declared in $E_2$.

While this restriction prevents most application of this rule, 
equations can still be flattened to the form
\[
Qx_1:A_1\cdots Qx_n : A_n . M_1 \doteq N_1 \wedge M_m \doteq N_n
\]

transforming $E_2$ first with the Raising rule untill 
an Exists-And transformation is possible, then repeating  
until $E_2$ no longer contains any existentially 
quantified variables.  This process is always terminating,
although potentially significantly slower.   

\subsection{Implementation}

Because typed substitution is necessary, we must now keep track of existential variable's
types.  This can significantly complicate the implementation of the unification algorithm
as the common technique of maintaining unbound existential variables with restrictions
can no longer be blindly used, as existential variables must be maintained in the 
formula.  The most advisable option is to maintain the type of the existential variable 
with each mention of the existential variable.  

After experimentation, good performance has been observed when this structure is implemented
as a zipper \citep{huet1997functional}. We have found exceptional performence when implementing this structure
as a zipper using a finger tree indexed lookups. Unfortunately, since variables are best 
implemented via DeBruijn indexes, general variable reconstruction is no longer trivial.  It is fortunate that general variable reconstruction 
is not necessary in the final implementation since an existential variable representing the body is always used at the top level.

Another option is to perform unification with untyped substitution.
While there is no proof at the moment that unification on the pattern subset of 
the calculus of constructions with untyped substitution for only the existential substitutions
is total, it is not unbelievable. Furthermore, omitting typed substitution does not alter
the correctness of the algorithm, only the potential totality.  

Ideally, knowledge that type checking terminated would be convincing enough
so it is not necessary to continue with the reconstruction.  However, reconstruction
is necessary for implementing the multi-pass proof search described previously.  
Furthermore, reconstruction is usefull since the exposed typing rules do not admit 
coherence.  In these cases, it is desirable to see what was infered by type inference.


    \section{Proof Search}

In a traditional logic programming language, the order of declaration of quantified arguments is irrelevant, 
and the context can be considered an unordered set (even though for implementation reasons it is not). 
In a dependently typed logic programming language where types direct proof search, types must be maintained in the context
and the context thus must maintain order.   Since search dynamically poses unification problems, which may not be 
entirely solvable until later in the search, unification and proof search are naturally mutually recursive procedures.
As it is important to maintain the mixed quantifier prefix thought proof search, it is desirable to provide a version 
of the algorithm where unification and proof search are not distinct procedures. 
\citep{pfenning1991logic} gave a succinct formulation where inhabitance and immediate implication were represented
directly in the unification calculus.  

\subsection{Proof Sharing}

\begin{definition}
Unification Calculus with Search

\[
U ::= U \wedge U 
 \orr \forall V : T . U
 \orr \exists V : T . U 
 \orr U \doteq U
 \orr \top
  \orr T \in T 
  \orr T \in T >> T \in T
\]

\end{definition}

The following new transformations are added to represent proof search:

\[
G_\Pi : M \in \Pi x : A . B   \rightarrow \forall x : A . \exists y : B . y \doteq M x \wedge y \in B
\]

\[
G^1_\m{Atom} : \forall x : A . F[M\in C]  \rightarrow \forall x : A . F[x \in A >> M \in C]
\]

\[
G^2_\m{Atom} : F[M\in C]  \rightarrow \forall x : A . F[c_0 \in A >> M \in C]
\]  where $c_0 : A$ is a constant

\[
D_\Pi : N\in \Pi x : A . B >> M \in C \rightarrow \exists x : A ( N x \in B >> M \in C) \wedge x \in A
\]

\[
D_\m{Atom} : N\in a N_1 \cdots N_n >> M \in a M_1 \cdots M_n \rightarrow N_1 \doteq M_1 \wedge \cdots \wedge N_n \doteq M_n \wedge N \doteq M
\]

\subsection{Proof Sharing}

In a pure setting, significant improvements to the efficiency of the system can be made by 
extending the quantifiers of the unification calculus to include forced inhabitant existential quantification.

\[
U ::= U \wedge U 
 \orr \forall V : T . U
 \orr \exists V : T . U 
 \orr \exists_f V : T . U 
 \orr U \doteq U
 \orr \top
 \orr T \in T >> T \in T
\]

\[
G^1_\m{Atom} : \forall x : A . F[\exists_f V : T . \top]  \rightarrow \forall x : A . F[x \in A >> M \in C]
\]

\[
G^2_\m{Atom} : \exists_f x : A . F[\exists_f V : T . \top]  \rightarrow \forall x : A . F[x \in A >> M \in C]
\]

\[
D_\Pi : N\in \Pi x : A . B >> M \in C \rightarrow \exists_f x : A ( N x \in B >> M \in C)
\]

In this situation, it is permitted to use the results of future searches for the solution of the current search.
While this sharing is optimal from an operational standpoint, it can make reasoning about the behavior 
of impure logic programs very difficult.  Given that Caledon is an impure programming language, reasoning about program
behavior comes before optimizing proof search.  It is the subject of future research to determine proof sharing teqchniques
that do not interfere with I/O. 

          
\chapter{Type Inference} 
    In this section, I introduce the type inference system for Caledon. 
I first discuss the inference rules and 
erasure form of $CICC$ dubbed $CICC^-$, 
a system based on the ``Implicit Calculus of Constructions'' ($ICC$) \citep{pollack1990implicit}. 
I then introduce a unification algorithm for handling constraints generated by $CICC^-$ and finally 
describe the construction of these constraints and the elaboration technique.

$ICC$ is an extention to the standard ``Calculus of Constructions'' which allows
a declaration that in all uses of a function, the argument be omitted 
and chosen during typechecking based on a provability relation.

Standard $CC$, and even standard LF 
can be unnecessarily verbose, as seen in the example \ref{code:long}.

\begin{figure}[h]
\begin{lstlisting}
defn churchList : prop -> prop
  as \ A : prop . [lst : prop -> prop] ([C] lst C) -> (A -> [C] lst C -> lst C) -> [C] lst C

defn mapCL : [A : prop] [B : prop] (A -> B) -> churchList A -> churchList B
  as \ A    : prop                    . 
     \ B    : prop                    .
     \ F    : A -> B                  . 
     \ cl   : churchList A            .
     \ lst  : prop -> prop            .
     \ nil  : [B] lst B               .
     \ cons : B -> [B] lst B -> lst B .

          cl lst nil (\v . cons (F v))

defn mapResult : churchList natural
  as mapCL natural boolean (\ a : natural . isZero a) someList

\end{lstlisting}
\caption{Maping over the church encoding of a list}
\label{code:long}
\end{figure}

Ideally, one omits redundant types whose values are parameterized 
and can be inferred from context. 

Omiting these types gives rise to the notion of an implicit type system.  
The Hindley-Milner \citep{hindley1969principal} system for inferring principle types in system F
is a special case of the system where implicit universally quantified type variables are automatically
resolved.

    \section{Implicit Calculus of Constructions}

Miquel \citep{miquel2001implicit} provides a more general system than that seen 
in Hindley-Milner, ICC to allow for implicit arguments.
Here, I will briefly explain the system and some of the relevant theoretical results that have been obtained.
As maintaining the flexibility of the system is important to future extentions of the Caledon language, 
I will present the implicit calculus in terms of Pure Type Systems.

\begin{figure}[h]
\[ 
E ::= V 
 \orr S 
 \orr E\;E 
 \orr \lambda V . E 
 \orr \Pi V : E . E 
 \orr \forall V : E . E 
\]
\caption{Syntax of ICC}
\label{icc:syntax}
\end{figure}

Note that Miquel's presentation of ICC uses curry style $\lambda$ bindings where types are ommitted.  
The typing rules for ICC are mostly the same as those for Pure Type Systems except that an extra rule
needs to be proided for abstraction, application, and formation of implicitly quantification.

\begin{figure}[h]
\[
\infer[\m{gen}]
{
\Gamma \vdash M : (\forall x : T . U)
}
{
\Gamma , x : T\vdash M : U
&
\Gamma \vdash (\forall x : T . U) : s
&
s \in S
&
x \notin FV(M)
}
\]

\[
\infer[\m{inst}]
{
\Gamma \vdash M : U [N/x]
}
{
\Gamma \vdash M : \forall x :T . U
&
\Gamma \vdash N : T
}
\]

\[
\infer[\m{imp-prod}]
{
\Gamma \vdash (\forall x : A . B) : s_3
}
{
\Gamma \vdash A : s_1
&
\Gamma,x:A \vdash B : s_2
&
(s_1,s_2,s_3) \in R
}
\]


\[
\infer[\m{strength}]
{
\Gamma \vdash M : U
}
{
\Gamma , x : T \vdash M : U
&
x \notin FV(M) \cup FV(U)
}
\]

\[
\infer[\m{ext}]
{
\Gamma \vdash M : T
}
{
\Gamma\vdash \lambda x . (M x)  : T 
&
x \notin FV(M)
}
\]
\caption{Typing for ICC}
\label{icc:typing}
\end{figure}

Note that in the formulation in \ref{icc:typing}, 
there is absolutely no way to control the type of the argument used explicitly.
Similarly, there is no mechanism for this in the syntax shown in \ref{icc:syntax}.
In the implemented version, this is not the case, as a notion of explicit binding has been provided.

Also note that in the formulation, neither strengthening rule nor 
the rule of extenionality are not admissible.  
These rules are however required to show subject reduction in this calculus.


\subsection{Subtyping}

\begin{definition}
Subtyping relation:
$\Gamma \vdash T \leq T' \;\; \equiv \;\; \Gamma, x : T \vdash x : T'$ 
\end{definition}

\begin{lemma}
Subtyping is a preordering:

\begin{tabular}{lrc}
$
\infer-[\m{sym}]{ 
\Gamma \vdash T \leq T
}{
\Gamma \vdash T : s
}
$
&
$
\infer-[\m{trans}]{ 
\Gamma \vdash T_1 \leq T_3
}{
\Gamma \vdash T_1 \leq T_2
&
\Gamma \vdash T_2 \leq T_3
}
$
&
$
\infer-[\m{sub}]{ 
\Gamma \vdash M : T'
}{
\Gamma \vdash M \leq T
&
\Gamma \vdash T \leq T'
}
$
\end{tabular}

\end{lemma}

\begin{lemma}
Domains of products are contravariant and codomains are covarient:

\begin{tabular}{lrc}
$
\infer[]{ 
\Gamma \vdash \Pi x : T . U \leq \Pi x : T' . U'
}{
\Gamma \vdash T' \leq T 
&
\Gamma,x : T' \vdash U \leq U'
}
$
&
$
\infer[]{ 
\Gamma \vdash \forall x : T . U \leq \forall x : T' . U'
}{
\Gamma \vdash T' \leq T 
&
\Gamma,x : T' \vdash U \leq U'
}
$
\end{tabular}
\end{lemma}




\subsection{Results}

There are two main results that follow from this calculus.

\begin{theorem}
Subject Reduction:
If $\Gamma \vdash M : T$ and $M \rightarrow_{\beta\eta*} M'$ then $\Gamma \vdash M' : T$
\end{theorem}

\begin{theorem}
Strong Normalization:
$\forall M \in \m{Term}. SN(M)$
\end{theorem}

It is important to note that because this calculus is Curry style and thus Church-Rosser is provable.  
While the internal representation and external presentation of Caledon is not necessarily Curry style, 
it is possible to mimic a non Curry style encoding into a curry style encoding through use of type ascriptions,
and evaluation delaying terms.  Technically, the calculus will no longer have the church-rosser property if 
evaluation delaying terms are included, but these are irrelevant in the presence of the strong normalization theorem
for the underlying calculus without them.
 
    \section{Elaboration}

In order to make use of the implicit system of $CICC$, an inference
relation must be provided.  
This is accomplished by extending the typing rules and providing
a mapping from the extended type derivation and term to 
an original type derivation and term. 

We only have one syntactic difference in this calculus:  $E\; \{ V : A = E \}$ 
is now simply $ E \; \{ V  = E \}$.  
We might also include Curry style binders in this presentation, but they shed little 
extra light on the workings of type inference.

\newcommand{\judgeCI}[0]{ \vdash_{i^-}}

Let $\Gamma \vdash A : T \wedge B : T'$ stand for $\Gamma \vdash A : T$ and $\Gamma \vdash B : T'$.
 
\begin{definition}
\textbf{($CICC^-$ Extended Typing Rules)}

%% inst/f %%
%%%%%%%%%%%%
\[
\infer[\m{inst/f}]
{
\Gamma \judgeCI M : [N/x]U 
}
{
\Gamma \judgeCI M : ?\Pi x :T . U
&
\Gamma \judgeCI N : T
&
x \notin DV(\Gamma)
}
\]

%% abs2 %%
%%%%%%%%%%
\[
\infer[\m{abs/f}]
{
\Gamma\judgeCI M : ?\Pi x : T . U
}
{
\Gamma, x : T\judgeCI M : [N/x]U \wedge N : T
&
\Gamma \judgeCI (?\Pi x : T . U) : K
&
x \notin FV(M) \cup DV(\Gamma)
}
\]

%% strength %%
%%%%%%%%%%%%%%
\[
\infer[\m{strength}]
{
\Gamma\judgeCI M : U
}
{
\Gamma, x : T  \judgeCI M : U
&
x \notin FV(M) \cup FV(U)  \cup DV(\Gamma)
}
\]

%% inst/b %%
%%%%%%%%%%%%
\[
\infer[\m{inst/b}]
{
\Gamma \judgeCI M \{ x = N \} : [N/x]U 
}
{
\Gamma \judgeCI M : ?\Pi x : T . U
&
\Gamma \judgeCI N : T
& 
x \notin GN(M)
&
x \notin BN(U)
}
\]

\end{definition}

In $CICC$, as in $CC$, the strengthening rule is admissible,
while in $CICC^{-}$, it is not.  

Note that the rule $\m{abs/f}$ might appear to not make sense at first
glance since it abstracts to a known term, but it can be considered
equivalent to an existential pack without the pack proof term, since
$x \notin FV(M)$  

Further note that conversion is now restricted to $\beta$ conversion in order to 
allow for the Church Rosser theorem which is necessary to prove subject reduction.

While we no longer care about the semantics of this language since we will be
elaborating to the sublanguage $CICC$ before evaluating and type checking further, we do not need to 
semantic related properties.  

However, it is still important to note that substitution holds.

\begin{theorem}
\textbf{(Substitution)}
\[
\infer-[\m{subst}]{ 
\Gamma, [N/x]\Gamma' \judgeCI [N / x]M : [N/x]T_2
}{
\Gamma, x : T_1, \Gamma' \judgeCI M : T_2
&
\Gamma \judgeCI N : T_1
}
\]
\label{ci:sub}
\end{theorem}

\begin{theorem}
\textbf{(Subject Reduction)} If $\Gamma \judgeCI M : T$ and $M \rightarrow_{\beta*} M'$ then $\Gamma \judgeCI M' : T$

\label{ci:sr}
\end{theorem}

\ref{ci:sr} is at the moment beleived to be true, 
although no full formalization of them exists.  Provided reductions are restricted to $\beta$ conversion, the church rosser 
theorem is simply provable and the proof of subject reduction is similar to that in the traditional calculus of constructions.

Without the $\m{abs/f}$ rule, subject reduction becomes unnecessary for the metatheory since 
the single direction subtyping relation is sufficient.  However, unification becomes difficult to implement.  

Unfortunately, the projection function now requires more information than is available syntactically, 
and thus must be given on the typing derivation.

\begin{definition}
\textbf{ (Projection from $CICC^{-}$ to $CICC$) }

\newcommand{\CICCmproj}[1]{ \left\llbracket #1 \right\rrbracket_{ci^{-}}}

\[
\CICCmproj{
\infer[\m{wf/e}]
{
\cdot \judgeCI 
}{}
}^{c}
:= \cdot
\]

\[
\CICCmproj{
\infer[\m{wf/s}]
{
\Gamma, x : T \judgeCI 
}
{
\overset{\mathcal{D}}{ 
\Gamma \judgeCI x : T 
}
&
\cdots
}
}^{c}
:= \CICCmproj{\Gamma \judgeCI}^c, \CICCmproj{\mathcal{D}} 
\]

\[
\CICCmproj{
\infer[\m{start}]
{
\Gamma,x:A \judgeCI x :A
}
{
\cdots
}
}
:= x
\]


\[
\CICCmproj{
\infer[\m{axioms}]
{
\Gamma,x:A \judgeCI c : s
}
{
\cdots
}
}
:= c
\]

%% prod %%
%%%%%%%%%%
\[
\CICCmproj{
\infer[\m{prod}]{ \Gamma \judgeCI (\Pi x : T . U) : s 
}{ 
\overset{\mathcal{D}_1}{ 
\Gamma \vdash T : s_1
}
&
\overset{\mathcal{D}_2}{ 
\Gamma,x:T \vdash U : s_2
}
&
\cdots
}
}
:=
\Pi x : \CICCmproj{ \mathcal{D}_1 }  . \CICCmproj{ \mathcal{D}_2 }
\]

%% prod* %%
%%%%%%%%$%%
\[
\CICCmproj{
\infer[\m{prod}*]{ \Gamma \judgeCI (?\Pi x : T . U) : s 
}{ 
\overset{\mathcal{D}_1}{ 
\Gamma \vdash T : s_1
}
&
\overset{\mathcal{D}_2}{ 
\Gamma,x:T \vdash U : s_2
}
&
\cdots
}
}
:=
?\Pi x : \CICCmproj{ \mathcal{D}_1 }  . \CICCmproj{ \mathcal{D}_2 }
\]

%% gen %%
%%%%%%%%%
\[
\CICCmproj{
\infer[\m{gen}]
{
\Gamma \judgeCI \lambda x : T . M : (\Pi x : T . U)
}
{
\overset{\mathcal{D}_1}{
\Gamma , x : T \judgeCI M : U 
}
&
\infer[\m{prod}]{ \Gamma \judgeCI (\Pi x : T . U) : s 
}{ 
\overset{\mathcal{D}_2}{ 
\Gamma \vdash T : s_1
}
&
\overset{\mathcal{D}_3}{ 
\Gamma,x:T \vdash U : s_2
}
&
\cdots
}
&
\cdots
}
}
:=
\lambda x : \CICCmproj{ \mathcal{D}_2 }  . \CICCmproj{ \mathcal{D}_1 }
\]

%% gen* %%
%%%%%%%%%%
\[
\CICCmproj{
\infer[\m{gen}*]
{
\Gamma \judgeCI ?\lambda x : T . M : (?\Pi x : T . U)
}
{
\overset{\mathcal{D}_1}{
\Gamma , x : T \judgeCI M : U 
}
&
\infer[\m{prod}*]{ \Gamma \judgeCI (?\Pi x : T . U) : s 
}{ 
\overset{\mathcal{D}_2}{ 
\Gamma \vdash T : s_1
}
&
\overset{\mathcal{D}_3}{ 
\Gamma,x:T \vdash U : s_2
}
&
\cdots
}
&
\cdots
}
}
:=
?\lambda x : \CICCmproj{ \mathcal{D}_2 }  . \CICCmproj{ \mathcal{D}_1 }
\]

%% app %%
%%%%%%%%%
\[
\CICCmproj{ 
\infer[\m{app}]
{
\Gamma \judgeCI M N : U [N/x]
}
{
\overset{\mathcal{D}_1}{ \Gamma \judgeCI M : \Pi x : T . U }
&
\overset{\mathcal{D}_2}{ \Gamma \judgeCI N : T }
}
}
:=
\CICCmproj{ \mathcal{D}_1 } \; \CICCmproj{\mathcal{D}_2}
\]

%% inst/b %%
%%%%%%%%%%%%
\[
\CICCmproj{ 
\infer[\m{inst/b}]
{
\Gamma \judgeCI M \{ x = N \} : U [N/x]
}
{
\overset{\mathcal{D}_1}{ \Gamma \judgeCI M : ?\Pi x :T . U }
&
\overset{\mathcal{D}_2}{ \Gamma \judgeCI N : T }
& 
\cdots
}
}
:=
\CICCmproj{\mathcal{D}_1} \; \{ x : \CICCmproj{\Gamma \vdash T : \m{kind}} = \CICCmproj{\mathcal{D}_2} \}
\]

%% inst/b %%
%%%%%%%%%%%%
\[
\CICCmproj{ 
\infer[\m{inst/f}]
{
\Gamma \judgeCI M : U [N/x]
}
{
\overset{\mathcal{D}_1}{ \Gamma \judgeCI M : ?\Pi x : T . U }
&
\overset{\mathcal{D}_2}{ \Gamma \judgeCI N : T }
&
\cdots
}
}
:=
\CICCmproj{\mathcal{D}_1} \; \{ x = \CICCmproj{\mathcal{D}_2} \}
\]

%% strength %%
%%%%%%%%%%%%%%
\[
\CICCmproj{ 
\infer[\m{strength}]
{
\Gamma \judgeCI M : U
}
{
\overset{\mathcal{D}}{ \Gamma, x : T \judgeCI M : U }
&
\cdots
}
}
:=
\CICCmproj{\mathcal{D}}
\]


%% abs/f %%
%%%%%%%%%%%
\[
\CICCmproj{ 
\infer[\m{abs/f}]
{
\Gamma \judgeCI M : ?\Pi x : T . U
}
{
\overset{\mathcal{D}}{ \Gamma \judgeCI M : [N/x]U }
&
\cdots
}
}
:=
?\lambda x : T . \CICCmproj{\mathcal{D}}
\]
\label{cicc-:proj}
\end{definition}


\begin{theorem}

\textbf{(Soundness of extraction)}  

\begin{alignat}{4}
\Gamma &\judgeCI &  & \implies & \CICCproj{\Gamma \judgeCI}^c & \judgeCI &
\\
\Gamma &\judgeCI & A : T & \implies & \CICCproj{\Gamma \judgeCI}^c & \judgeCI & \CICCproj{ \Gamma \judgeCI A : T }
\end{alignat}

\label{cicc-:sound}
\end{theorem}

Since $CICC$ permits $\eta$ equivalence and $CICC^-$ does not, the extraction in the 
reverse direction is no longer sound.  For our purposes this is not objectionable 
since $CICC$ is known to be consistent and there is no reason to convert back into $CICC^-$, as it is 
used entirely as a pre-elaboration language.  Once terms are typechecked and type infered in $CICC^-$
they are typechecked in $CICC$ and normalized in $CICC$.  While the reverse extraction is
in general not sound, normal terms with normal types are clearly typable in $CICC^-$.

%%%%%%%%%%%%%%%%%%%%%%%%%%%%%%%%%%%%%%%%%%%%%%%%%%%%%%%%%%%%%
%%% Subtyping %%%%%%%%%%%%%%%%%%%%%%%%%%%%%%%%%%%%%%%%%%%%%%%
%%%%%%%%%%%%%%%%%%%%%%%%%%%%%%%%%%%%%%%%%%%%%%%%%%%%%%%%%%%%%
\subsection{Subtyping}

Similar to $ICC$, these rules result in a subtyping relation, which will be of
importance during type inference and elaboration.

\begin{definition}
Subtyping relation:
$\Gamma \judgeCI T \leq T' \;\; \equiv \;\; \Gamma, x : T \judgeCI x : T'$  where $x$ is new.
\end{definition}

\begin{lemma}
Subtyping is a preordering:
\[
\begin{array}{lr}
\infer-[\m{refl}]{ 
\Gamma \judgeCI T \leq T
}{
\Gamma \judgeCI T : s
}
&
\infer-[\m{trans}]{ 
\Gamma \judgeCI T_1 \leq T_3
}{
\Gamma \judgeCI T_1 \leq T_2
&
\Gamma \judgeCI T_2 \leq T_3
}
\end{array}
\]

\[
\infer-[\m{sub}]{ 
\Gamma \judgeCI M : T'
}{
\Gamma \judgeCI M \leq T
&
\Gamma \judgeCI T \leq T'
}
\]
\end{lemma}

This theorem is an application of the substitution lemma.

\begin{lemma}
Domains of products are contravariant and codomains are covarient:

\[
\begin{array}{lr}
\infer-[]{ 
\Gamma \judgeCI \Pi x : T . U \leq \Pi x : T' . U'
}{
\Gamma \judgeCI T' \leq T 
&
\Gamma,x : T' \judgeCI U \leq U'
}
&
\infer-[]{ 
\Gamma \judgeCI \forall x : T . U \leq \forall x : T' . U'
}{
\Gamma \judgeCI T' \leq T 
&
\Gamma,x : T' \judgeCI U \leq U'
}
\end{array}
\]
\end{lemma}

Unlike traditional subtyping relations where an explicit subtyping rule must be included in the type system,
this system's subtyping relation is much easier to manage during unification, as it is simply
a macro for a provability relation.  

This allows one to implement higher order unification almost exactly
as is usual without to much modification as would be the case in a lattice unification algorithm.  
Instead, the modification is made to the search procedure, and subtyping constraints 
are realized as search terms.  

However, with the addition of the strengthening rule, 
this kind of modification not entirely necessary, 
as it is provable that this subtyping relation is symetric \ref{ci:sym}, 
and thus an entirely symetric unification algorithm can be presented.

Theorem \ref{ci:sym} isn't exactly obvious upon first glance, 
so I will provide intuitive justification first.

In $CICC$, by uniqueness of types, 
$\Gamma \vdash x : A$ and 
$\Gamma \vdash x : B$ implies
$A \equiv_{\beta\eta*} B$.  
In $CICC^{-}$ however, there is no such uniqueness of 
types properties.  
Rather, the $\m{inst/f}$ and $\m{abs/f}$ 
rules permit you to repsectively,
add an initialize an implicit argument, 
abstract implicitely upon an unused argument. 

Thus if $\Gamma , x : ?\Pi z : T . A \judgeCI x : A$
by implicit instantiation of the argument $z:T$,
we might also
derive
$\Gamma , x : A \judgeCI x : ?\Pi z : T . A$
given that $z \notin FV(x)$ and that 
$\Gamma , x : ?\Pi z : T . A \judgeCI x : A$ 
implies $ \Gamma , x : ?\Pi z : T . A \judgeCI$ 
which implies $ \Gamma\judgeCI x : (?\Pi z : T . A) : K$.

\begin{theorem}
\textbf{(Symmetry)}
$\Gamma \judgeCI A \leq B $ implies 
$\Gamma \judgeCI B' \leq A' $. where $A \equiv_{\beta} A'$ and $B \equiv_{\beta} B'$
\label{ci:sym}
\end{theorem}

\begin{proof}

This is proved by induction on the structure of the proof
of $\Gamma, x : A \judgeCI x : B$.  Here I only consider 
the cases relevant to the new fragment.

\setcounter{tcase}{0}

%%%%%%% STRENGTH %%%%%%%%%
\begin{tcase}
We begin with the non admissible strengthening rule.
\end{tcase}

\begin{prooftree}
\AxiomC{$\Gamma, x : A, z : T \judgeCI x : B$}
\AxiomC{$z \notin FV(x)\cup FV(B) \cup DV(G)$}
\RightLabel{strength}
\BinaryInfC{$\Gamma, x : A \judgeCI x : B $}
\end{prooftree}

from this we can derive via the induction hypothesis, $\Gamma, x : B', z : T \judgeCI x : A'$ 
and then reapply strengthening.

\begin{prooftree}
\AxiomC{$\Gamma, x : B', z : T \judgeCI x : A'$}
\AxiomC{$z \notin FV(x)\cup FV(A') \cup DV(G)$}
\RightLabel{strength}
\BinaryInfC{$\Gamma, x : B' \judgeCI x : A' $}
\end{prooftree}

%%%%%%% INST %%%%%%%%%
\begin{tcase}
In this case we cover implicit instantiation.
\end{tcase}

\begin{prooftree}
\AxiomC{$\Gamma, x : A \judgeCI x : ?\Pi z :T . B$}
\AxiomC{$\Gamma, x : A \judgeCI N : T$}
\AxiomC{$z \notin DV(\Gamma, x:A)$}
\RightLabel{inst/f}
\TrinaryInfC{$\Gamma,x : A \judgeCI x : [N/z]B $}
\end{prooftree}

Suppose $z$ is not in $FV(B)$ then $[N/z]B \equiv B$.   
From this the following proof is possible:

The first steps are the following few derivations:

\begin{prooftree}
  \AxiomC{$\Gamma,\judgeCI B' : K $} 
  \AxiomC{$z \notin FV(B')\cup FV(K) $} 
\RightLabel{streng}
\BinaryInfC{$\Gamma,z:T',  \judgeCI B' : K $} 
\end{prooftree}

\begin{prooftree}
  \AxiomC{$\Gamma \judgeCI T': K' $} 
  \AxiomC{$\Gamma,z:T' \judgeCI B' : K $} 

\RightLabel{form/f}
\BinaryInfC{$\Gamma \judgeCI ?\Pi z : T' . B' : K$}
\end{prooftree}

\begin{prooftree}
 \AxiomC{$B \equiv_{\beta}B'$}
 \AxiomC{$z \notin B'$}
\BinaryInfC{$z \notin FV(B')$}
\end{prooftree}

From these, we can derive the following result about $B'$.

\begin{prooftree}
    \AxiomC{$\Gamma,z:T' \judgeCI B' : K $} 

   \RightLabel{start}
   \UnaryInfC{$\Gamma,z:T', x : B' \judgeCI x : B'$}

   \AxiomC{$\Gamma \judgeCI ?\Pi z : T' . B' : K$}

   \AxiomC{$z \notin FV(x)\cup DV(\Gamma)$}
\RightLabel{abs/f}
\TrinaryInfC{$\Gamma, x : B' \judgeCI x : ?\Pi z : T' . B'$}
\end{prooftree}

Finally, we can derive the desired result:

\begin{prooftree}
   \AxiomC{$\Gamma, x : B' \judgeCI x : ?\Pi z : T' . B'$}

      \AxiomC{$IH(\Gamma , x : A \judgeCI x : ?\Pi z : T . B)$}
    \UnaryInfC{$\Gamma, x : ?\Pi z : T' . B' \judgeCI x : A'$}

  \RightLabel{subst}
  \BinaryInfC{$\Gamma, x : B' \judgeCI x : A'$}

  \AxiomC{$z \notin FV(B')$}
\BinaryInfC{$\Gamma, x : [N/z]B' \judgeCI x : A'$}
\end{prooftree}

Note that the only unexplained axioms here are 
$\Gamma \judgeCI B': K$ and $\Gamma \judgeCI T' : K$ in this proof.
Because $\Gamma, x : A' \judgeCI x : ?\Pi z:T' . B'$ is true, we know that 
$\Gamma, x : A' \judgeCI ?\Pi z : T' . B' : K$ and thus that 
$\Gamma,x : A' \judgeCI ?\Pi T' : K' $ and $\Gamma, x : A', z : T' \judgeCI B' : K$.  
By strengthening we can infer $\Gamma \judgeCI B': K$ and $\Gamma \judgeCI T' : K$.

%%%%%%%%%%%%%%%%%%%%%%%%%%%

On the other hand, if $z$ is in $FV(B)$ we achieve different proofs.  
Now we can infer that $x \notin FV(N)$, but we can not show that $[N/z]B' \equiv B'$.

By the induction hypothesis, we can infer $\Gamma , x : ?\Pi z : T' . B'\judgeCI x : A'$.

First, we know that $\Gamma,x : A \judgeCI [N/z]B : K$ by well formedness of the judgement 
$\Gamma,x : A \judgeCI x : [N/z]B' $ and the conversion rule.

\begin{prooftree}
\AxiomC{$\Gamma, x : A \judgeCI [N/z]B' : K$}
\AxiomC{$x \notin FV(B')$}
\RightLabel{strength}
\BinaryInfC{$\Gamma \judgeCI [N/z]B' : K$}
\RightLabel{start}
\UnaryInfC{$\Gamma, x : [N/z]B'\judgeCI x : [N/z]B'$}
\AxiomC{$z \notin FV(N)$}
\RightLabel{strength}
\BinaryInfC{$\Gamma, x : [N/z]B', z : T\judgeCI x : [N/z]B'$}
\end{prooftree}

thus, we can use the $\m{abs/f}$ rule to construct a form we can use in substitution.

\begin{prooftree}
\AxiomC{$\Gamma, x : [N/z]B', z : T \judgeCI (x : [N/z]B') \wedge N : T'$}
\AxiomC{$\Gamma, x : [N/z]B' \judgeCI ?\Pi z : T' . B' : K $}
\AxiomC{$z \notin FV(M) \cup DV(\Gamma)$}
\RightLabel{abs/f}
\TrinaryInfC{$\Gamma, x : [N/z]B' \judgeCI x : ?\Pi z : T' . B'$}
\end{prooftree}

Finally, we get the following derivation.

\begin{prooftree}
\AxiomC{$\Gamma, x : [N/z]B' \judgeCI x : ?\Pi z : T' . B'$}
\AxiomC{$\Gamma , x : ?\Pi z : T' . B'\judgeCI x : A'$}
\RightLabel{subst}
\BinaryInfC{$\Gamma , x : [N/z]B' \judgeCI x : A'$}
\end{prooftree}

%%%%%%% ABS %%%%%%%%%
\begin{tcase}
In this case we examine the $\m{abs/f}$ rule.
\end{tcase}

\begin{prooftree}
\AxiomC{$\Gamma, x : A, z : T \judgeCI x : [N/z]B \wedge N : T$}
\AxiomC{$\Gamma, x : A \judgeCI ?\Pi z : T' . B' : K $}
\AxiomC{$z \notin FV(M) \cup DV(\Gamma, x)$}
\RightLabel{abs/f}
\TrinaryInfC{$\Gamma, x : A \judgeCI x : ?\Pi z : T . B$}
\end{prooftree}

From this we can infer that $z \notin FV(A)$.  This is usefull since we can derive:
\[
\Gamma, z : T, x : A \judgeCI x : [N/z]B \wedge N : T
\]

We can then apply the induction hypothesis to get the following:

\[
\Gamma, z : T, x : [N'/z]B' \judgeCI x : A'
\]

From this we can infer $\Gamma, x : [N'/z]B' \judgeCI x : A'$
by strengthening since $z \notin FV(x)\cup FV(A')$.

Furthermore, we can infer that $\Gamma \judgeCI N' : T'$ since $N' \equiv_{\beta} N$ 
so $\Gamma \judgeCI N' : T$ by subject reduction and $T' \equiv_{\beta} T$ so $\Gamma \judgeCI N' : T'$
by conversion.

We can also derive $\Gamma, x : ?\Pi z : T' . B' \judgeCI x : ?\Pi z : T' . B'$ by the start rule. 

We get the following proof:

\begin{prooftree}
\AxiomC{$\Gamma, x : ?\Pi z : T' . B' \judgeCI x : ?\Pi z : T' . B'$}
\AxiomC{$\Gamma \judgeCI N' : T'$}
\AxiomC{$ z \notin DV(\Gamma)$}
\RightLabel{inf/f}
\TrinaryInfC{$\Gamma, x : ?\Pi z : T' . B' \judgeCI x : [N'/x] B'$}
\end{prooftree}

Finally, with the knowledge that $z\notin FV(A')$, we can derive the following:

\begin{prooftree}
\AxiomC{$\Gamma, x : ?\Pi z : T' . B' \judgeCI x : [N'/x] B'$}
\AxiomC{$\Gamma, x : [N'/x] B' \judgeCI x : A'$}
\RightLabel{subst}
\BinaryInfC{$\Gamma, x : ?\Pi z : T' . B' \judgeCI x : A'$}
\end{prooftree}

\end{proof}

    \section{Semantics for $CICCI$}


\subsection{Substitution With Implicits}

The formulation of hereditary substitution in the presence of 
implicit arguments is not that unlike the presentation of
heredetary substitution without implicit arguments, 
but for additional checks required.
The main difficulty is the notion of an acceptable substitution. 
Because implicit bindings are not $\alpha$ convertable, 
certain substitutions are not permited.  
Because as many substitions should be permitted as possible, 
the situation becomes significantly more complex in the 
hereditary case, where substitutions might not carry types.  
The easiest way to define substitution in this case is with an ``illegal'' alpha substitution, 
which maps implicitly bound variables to fresh names, 
and produces a memory to map them back.

In this case, we can say that a substitution $[S/x] M$ is legal if 
$FV(S) \subseteq FV(\alpha_I^-1( [\alpha_I(S)/x] M) )$.

\begin{definition}
\textbf{(Implicit Typed Hereditary Substitution)}


\[
[S / x : A]^n_{\Gamma } (?\lambda y : B . N) := ?\lambda y:B . [S / x : A]^n_{\Gamma, y : B} N
\] 

\[
\eta^{-1}_{?\Pi x : A . B}(N) := ?\lambda x : A . N \; \{ x = \eta^{-1}_A(x) \}
\] since $N$ being typable by $?\Pi x $ means that $x$ can not appear free in $N$

\[
\m{H}_{\Gamma}(P \downarrow ?\Pi y : B_1 . B_2 , \{ v := N \} ) := P\; \{ v := N \} \downarrow [N/y : B_1]^n_{\Gamma}B_2
\]

\[
\m{H}_{\Gamma} ((?\lambda v : A_1 . N) \uparrow ?\Pi v : A_1 . A_2 , \{ v := P \}) 
:= [P/v]^n_{\Gamma \vdash v : A_1} N \uparrow A_2
\]

\[ 
\m{H}(?\lambda v : T . P \uparrow \_ , A) := ?\lambda v : T . \m{H}(P,A)
\]

\label{def:hered}
\end{definition}



\subsection{Unification With Implicits}

Now we can use the convenient fact that $\Gamma \vdash A  \leq B$ implies $\Gamma \vdash B \leq A$.

             
\chapter{Implementation}
    In this chapter I discuss the some of the details of the 
specification details of the Caledon language.
    \section{Type Inference}
    \section{Type Families}

 
    \section{Controlled Nondeterminism}

A logic program need not only be a deterministic depth first pattern search. For purely declarative 
axioms, the depth first strategy is usually, in fact, incomplete 
and not representative of what should and should not halt.
Depth first search, however, can be more efficient in many cases and when used intentionally
will constrict nondeterminism. Concurrency in a program with IO has
well known uses and advantages. The breadth first proof search strategy can conveniently
be represented by a program where every pattern-match forks and then executes
concurrently. This is not ideal if used indiscriminately. An ideal implementation
allows one to control the patterns that are searched in parallel and in sequence. 
In the following snippet of the code, the distinction between
breadth first and depth first queries are used to emulate the concept of concurrency and
cause the program to have more complex behavior.

\begin{figure}[H]
\begin{lstlisting}

query main = runBoth false

defn runBoth : bool -> type
  >| run0 = runBoth A
            <- putStr ``ttt ``
            <- A =:= true
   | run1 = runBoth A 
            <- putStr ``vvvv''
            <- A =:= true
   | run2 = runBoth A
            <- putStr ``qqqq''
            <- A =:= true
  >| run3 = runBoth A
            <- putStr `` jjj''
            <- A =:= false

\end{lstlisting}
\caption{Nondeterminism control}
\label{nondet:ex1}
\end{figure}

In example \ref{nondet:ex1}, the query main prints to the screen something similar to “ttt
vqvqvqvq jjj”. This happens because, despite proof search failing on the first three axioms due
to an incorrect match, the fail is deferred until after IO has been performed. The middle
axioms, ``run1'' and ``run2'' are declared to be breadth first axioms, while ``run0'' and ``run3'' are
declared to be depth first axioms.


\begin{figure}[H]
\begin{lstlisting}

defn fail : type
  as true =:= false

defn then : type -> type -> type
  >| then-A = Fst `then` Snd <- Fst <- fail
  >| then-B = Fst `then` Snd <- Snd

query main = print ``Hello `` `then` print ``world!''

\end{lstlisting}
\caption{Sequential predicate}
\label{nondet:seq}
\end{figure}

The predicate “then” in figure \ref{nondet:seq} executes its first routine and subsequently its second sequentially.
To understand how this works, it is helpful to step through the query
“main”. When “main” is called, it will first initiate a goal of the form “then (print ‘Hello
’) (print ‘world!’)”. Because the axioms “then-A” and “then-B” are both preceded by ``>|'',
the “then-A” is attempted, the search of which is constrained to be entire. Since “(print
’Hello ’)” succeeds and does not nondeterministically branch, “fail” will be initiated as
the next goal. ``fail'' will certainly fail, and the current branch will end. “then-B” is the
next attempt. It will succeed since “(print ’world!’)” succeeds.

\begin{figure}[H]
\begin{lstlisting}

defn while : type -> type -> type
   | while-A = Fst `while` Snd <- Fst
   | while-B = Fst `while` Snd <- Snd

query main = print ``aaaa'' `while` print ``bbbb''

\end{lstlisting}
\caption{Concurrent predicate}
\label{nondet:con}
\end{figure}


The predicate “while” in figure \ref{nondet:con} executes its first routine at the same time as its
second. The result of “main” might then be “abababab”. It is important to note that in this notion of concurrency, neither thread is the ``original'' thread.

    \section{IO and Builtin Values and Predicates}

While not this material is neither novel nor particularly difficult, 
I'm including this section as both a specification and a guide 
to future implementors of logic programming languages with proof search.

The first thing to note when programming in Caledon is that searching for items of type $\m{prop}$ 
can might IO, and thus IO can be performed during typechecking.  
This can be understood as the set of axioms differing based on the environment available.

IO is performed when the evaluation function encounters a query for a builtin IO performing function.

\begin{lstlisting}
eval : Proposition -> Environment Formula
eval (a %*$\in$ *) ``print'' Str) = ( if gvar a then print str else ()
                                    ; return (a %* $\doteq $ *) printImp Str)
                                    )
\end{lstlisting}

It is important to include the check that $a$ has not already been resolved such that repeated io actions are not performed
when nondeterministically proof searching.  

It is tempting to define predicates that take input as taking as an argument 
a function that uses the input.  This is in fact a valid way to define such functions, 
and permits for hints of directionality in the types.  
However, it is still possible to escape.  

\begin{lstlisting}
builtin readLine : (string -> prop) -> prop

defn readLineEscape : string -> prop
  as \ s : string . readLine (\ t . t =:= s )

\end{lstlisting}

Ensuring variables do not escape their intended scope is necessary to ensuring
that the intended IO action is only executed once, and not multiple times during proof search.

While it is in fact possible to reason about nondeterministic IO, it is desirable to also have actions
that can not be executed twice, for which nondeterminism is not possible.

For this, the notion of an monad is in fact usefull.  
In this setting io is presented as a series of builtin axioms.

\begin{lstlisting}
defn io : prop -> prop
   | bind     = io A -> (A -> io B) -> io B
   | return   = A -> io A
   | ioReadLine = io string
   | ioPrint    = string -> io unit
\end{lstlisting}

We can now no longer write $A \in \m{readLine}$ though since $\m{readLine} : \m{prop}$ 
is a value and thus has no inhabitants. 

Instead, an interpretation predicate can be created which maps these dummy io actions to real actions.

\begin{lstlisting}
defn run : io A -> A -> prop
  >| runBind = run (bind IOA F) V
            <- run IOA A 
            <- run (F A) V 
  >| runReturn   = run (return V) V
  >| runReadLine = run ioReadLine A <- readLineEscape A
  >| runPrint    = run (ioPrint S) one <- print S
\end{lstlisting}

It is important to note that since the type system is the calculus of constructions, io actions constructed
from $io$ will be total, severely limiting their utility.
More complex io actions and interpreters can be generated, most importantly ones involving recursion 
or infinite loops.

 
\chapter{Programming with Caledon}
    \section{Typeclasses}


I've mentioned previously that implicit arguments along side polymorphism and proof search can subsume Haskell style type classes.

The easiest way to see this is through an implementation of what is known as the ``Show'' type class in Haskell.  
In a logic programming language, a predicate that can be used to print a datatype can also be used to read a datatype, so here we shall discuss a ``serialize'' type class.


\begin{figure}[H]
\begin{lstlisting}

defn serializeBool : bool -> string -> type
  >| serializeBool-true = serializeBool true ``true''
  >| serializeBool-false = serializeBool false ``false''

\end{lstlisting}
\caption{Serializing booleans}
\label{prog:serializing}
\end{figure}


\begin{figure}[H]
\begin{lstlisting}

query readQuery = exists B : bool. serializeBool B ``true''
query printQuery = exists S : string . serializeBool false S

\end{lstlisting}
\caption{Bidirectional serializing}
\label{prog:bidi}
\end{figure}

Notice that the predicate \ref{prog:serialize} in the sense that both of the queries in 
\ref{prog:bidi} will resolve. 

The serialize predicate is a useful one and we would like it to be polymorphic in all types for which we've implemented a serialize function.  This is possible using implicit arguments.

We'd first create an open type for the type class serializable.

\begin{figure}[H]
\begin{lstlisting}

open serializable : [T]{ serializer : T -> string -> type } type

\end{lstlisting}
\caption{The type of the type class serializable }
\label{prog:sty}
\end{figure}

We'd then define a function ``serialize'' which unpacks the the implicit dependency of the type serializable.


\begin{figure}[H]
\begin{lstlisting}

defn serialize : {T}{ serializable : T -> string -> type } T -> string -> type
   | serializeImp = 
       [ Serializer : T -> string -> type ]
       [ Serializable : serializable T { serializer = Serializer }]
       serialize { serializable = Serializable } V S
     <- Serializer V S

\end{lstlisting}
\caption{The implementation of the function serialize }
\label{prog:imp}
\end{figure}


\begin{figure}[H]
\begin{lstlisting}

instance serialize-bool = serializable bool { serializer = serializeBool }
instance serialize-nat = serializable nat { serializer = serializeNat }

\end{lstlisting}
\caption{ Instances of serializable }
\label{prog:inst}
\end{figure}


To implement an instance of the serializable type class, one would add an instance axiom to the environment as in \ref{prog:inst}

Use of the function would then omit the implementation of the ``serializable'' argument and type argument such that they might be resolved automatically as in \ref{prog:uses}


\begin{figure}[H]
\begin{lstlisting}

query readQueryBool = exists B . serialize B ``true''
query printQueryBool = exists S . serialize false S

query printQueryNat = exists S . serialize (succ (succ zero)) S
query readQueryNat = exists S : nat . serialize S ``(succ (succ zero))''

\end{lstlisting}
\caption{ Instances of serializable }
\label{prog:inst}
\end{figure}


This process can be extended to not only open type classes, but closed typeclasses where 
resolution involves arbitrary computation.  
While it is difficult to point to uses of this capability that can be discussed in these confines,
type class computation has been known to the Haskell community for quite some time and has been used in application ranging from embedding an imperative computation monad with local variable use and 
assignment rules similar to those of C, to an RPC framework which creates end points based on functions with arbitrarily complex type signatures.

    \section{Linear Predicates}

One major drawback of the type class paradigm outlined in the previous section is the
inability for a typeclass to uniquely determine membership of type in a type class based
on floating predicates in the environment with matching signatures. While ideal behavior
is possible for theorems in the ``Calculus of Constructions'' which exhibit ideal parametricity,
the type “type” has the trivial inhabitant “type.”  Thus implementations
will nearly always resolve to this version.

\begin{figure}[H]
\begin{lstlisting}

defn serializeable : [T] { serializer : T −> string −> type } type
   | serializable-auto-find = [T] [ Serializer : T -> string -> type ]
      serializable T { serializer = Serializer }

\end{lstlisting}
\caption{This would be nice}
\label{lin:nice}
\end{figure}

For example, it would be very nice if the predicates in figure \ref{lin:nice} automatically
resolved to a reasonable instance of serializable.

\begin{figure}[H]
\begin{lstlisting}

defn serialize : {T} { serializer : T −> string −> type } T -> string -> type
  as ?\T : type . ?\ serializer : T -> string -> type . \v : T . \s : string 
   . serializer v s

\end{lstlisting}
\caption{Nice implementation}
\label{lin:imp}
\end{figure}

We could implement serialize as above in the definition in figure \ref{lin:imp}.
However, this kind of implementation will fail to determine the correct instance of ``serializer.''


\begin{figure}[H]
\begin{lstlisting}

query writeBool = serialize true ``true''
===>
query writeBool = 
   serialize 
       {T = bool}
       { serializer = \ x : bool . \ y : string . type } 
       true ``true''

\end{lstlisting}
\caption{Trivial Failure}
\label{lin:fail}
\end{figure}

Rather, it will infer a trivial predicate, as seen in the query in figure \ref{lin:fail}

\begin{figure}[H]
\begin{lstlisting}

defn show : {T}{shower : T -> string } T -> string
  as ?\T : type . ?\ shower : T -> string . \ v : T . shower v 

query writeBool = print (show true)
===>
query writeBool = print (show {T = bool}{shower = \ x : bool . nil })

\end{lstlisting}
\caption{Functions also fail}
\label{lin:ffail}
\end{figure}

One might think that functions, as an alternative to predicates, are immune to inhabitation by trivial and
incorrect values as in the above scenario. However, unless specified with their properties (tedious), 
functions have similar drawbacks, as is seen in the program in figure \ref{lin:ffail}.


\begin{figure}[H]
\begin{lstlisting}

defn show : {T}
            {shower : T -> string}
            {reader : string -> maybe T}
            {comp1  : [v] reader (shower v) =:= just v}
            {comp2  : [v] fromJust (reader s) (\x . shower x =:= s) type}
            T -> string
   as ?\ T : type
    . ?\ shower : T -> string
    . ?\ reader : _
    . ?\ comp1  : _
    . ?\ comp2  : _
    . \ v : T
    . shower v

\end{lstlisting}
\caption{Kind of a success using proofs}
\label{lin:success}
\end{figure}

One can sometimes work around this by including metatheorems about the implicit
functions, as in the figure \ref{lin:success}. However, proving the metatheorems is often tedious,
impossible, or sometimes just plain slow for the compiler.

One method under investigation to solve this ambiguity problem uses substructural dependent
quantification, in which types indicate that the function argument can  be
used only once in the term, but unlimited times in types. While
this is a subject of ongoing work on my part, I have included a description of my ideas.

\begin{figure}[H]
\begin{lstlisting}

defn serializeBool : bool -o string -o type
   | serializeBool-true = serializeBool true ``true''
   | serializeBool-false = serializeBool false ``false''

defn serialize : {T}{ serializer : T -o string -o type } T -> string -> type
  as ?\ T 
   . ?\ serializer : T -o string -o type ]
   .  \ v : T 
   .  \ s : string 
   . serializer v s

\end{lstlisting}
\caption{Linear types would be useful here}
\label{lin:linideal}
\end{figure}

$F : A \lolli B$ shall mean that the function $F$ only consumes a single resource of type $A$.  
$F : \forall_o x : A . B$ shall mean the same in a dependent setting.


The problem is solved in the figure \ref{lin:linideal} since the only function which linearly 
consumes a single boolean and a single string in the program and outputs a type is ``serializeBool''.  
Other functions that do this might be added later, but functions of the form 
$(\lambda x : \m{bool}. \lambda s : \m{string} . \m{type})$ are not possible.  
Unfortunately, something alon gthe lines of 
$(\lambda x : \m{bool}. \lambda s : \m{string} . \m{isString} s \wedge \m{isBool} x)$ might be possible, 
but these are significantly more manageable provided one is careful.  

Fortunately, the fact that we are working with higher order abstract syntax in the ``Calculus of Constructions''
means that linear dependent products are actually implementable withing Caledon.


\begin{figure}[H]
\begin{lstlisting}

defn restriction : type
   | linear = restriction
   | unused = restriction

defn restrictor : restriction -> restriction -> restriction -> type
   | restrictor-linear1 = restrictor linear unused linear
   | restrictor-linear2 = restrictor linear linear unused 
   | restrictor-unused = restrictor unused unused unused


defn restrict : restriction -> [ T : type ] [ P : T -> type ] ( [ x : T ] P x ) -> type
   | restrict-unused = 
     restrict unused T (\ x : T . P) (\ x : T . G)

   | restrict-linear = 
     restrict linear T (\ x : T . T) (\ x : T . x)

   | restrict-app = 
      restrict Ba T (\ x : T . P x (G x)) (\ x : T . (F x) (G x))
   <- restrict Bb T (\ x : T . [z : Q x] P x z ) (\ x : T . F x)
   <- restrict Bc T (\ x : T . Q x ) (\ x : T . G x)

   | restrict-lam = 
      restrict B T (\ x : T . [y : A] P x y) (\ x : T . \ y : A . F y x)
   <- [ y : A ] restrict B T (\ x : T . P x y) (\ x : T . F y x)

   | restrict-eta = 
      restrict B T (\ x : T . [y : A x] P x y) (\ x : T . \ y : A x. F x y)
   <- restrict B T (\ x : T . [y : A x] P x y) (\ x : T . F x)

\end{lstlisting}
\caption{Linearity in Caledon}
\label{lin:linimp}
\end{figure}



\begin{figure}[H]
\begin{lstlisting}

fixity lambda lolli
fixity lambda llam

defn lolli : [ T : type ] ( T -> type ) -> type
   | llam = [T]{TyF}[F : [x : T] TyF x]
            restrict linear T TyF F => ( lolli x : T . TyF x)

defn lapp : {A : type} { T : A -> type } [ f : lolli x : A . T x ] [ a : A ] T a -> type
   | lapp-imp = lapp (llam _ _ F _) V (F V)

fixity arrow -o
defn -o : type -> type -> type
  as \t : type . \t2 : type . lolli t (\x : t . t2)

\end{lstlisting}
\caption{Linear Dependent Product}
\label{lin:lindep}
\end{figure}


In the figure \ref{lin:linimp} the predicate ``restrict linear'' encodes the test that an arbitrary function, even one with a dependent type, is linear \citep{benton1993term} if its argument is used
exactly once in the term and potentially many times in the type.  
``lolli'' is the linear dependent type constructor and ``llam'' is the linear function constructor.
Provided the code from figure \ref{lin:linimp} was in the environment, figure \ref{lin:linideal} in fact works.


More of these cases could be made less ambiguous through use of an ordered dependent type 
constructor, but this is significantly more complicated to define, although certainly possible.


\begin{figure}[H]
\begin{lstlisting}

defn sum : nat -o nat -o nat -o type
   | sum-zero = [N : nat]
                [Sum : nat -o nat -o type]
                [Sum' : nat -o type]
                [Sum'' : type]
     lapp sum zero Sum -> lapp Sum N Sum' -> lapp Sum' N Sum'' -> Sum''
   | sum-succ = [N M R : nat]
                [Sum1 Sum2 : nat -o nat -o type ]
                [ Sum1' Sum2' : nat -o type ]
                [ Sum1'' Sum2'' : type ]
       lapp sum N Sum1 ->
       lapp Sum1 M Sum1' -> 
       lapp Sum1' R Sum1'' -> Sum1''
    -> lapp sum (succ N) Sum2 -> 
       lapp Sum1 M Sum2' ->
       lapp Sum2' (succ R) Sum2'' -> Sum2''
   
\end{lstlisting}
\caption{Use of a linear type}
\label{lin:use}
\end{figure}


Of course, as seen in figure \ref{lin:use}, actually using this linear dependent product is a bit
absurd.  It requires flattening application into a logic programming form where the target
is a type variable.



\begin{figure}[H]
\begin{lstlisting}

fixity application lapp
defn sum : nat -o nat -o nat -o type
   | sum-zero = [N : nat] *APP=lapp* sum zero N N
   | sum-succ = [N M R : nat] 
        *APP=lapp* sum N M R 
     -> *APP=lapp* sum (succ N) M (succ R)
\end{lstlisting}
\caption{Example of a syntax for flattening application}
\label{lin:flat}
\end{figure}


That the end result is a type variable means that the family checking algorithm
is no longer applicable. In this case, it is helpful to either hard code
linearity or provide a syntax for flattening successive applications using a predicate, as
seen in figure \ref{lin:flat}

Syntax for applications could potentially extend the notion of a family such that
predicates using these applications could also be frozen. In this case, the applicator $x$ used with 
$*APP = x*$ would be declared, so that the compiler could check that its type matches the form $\{ A : \m{type} \}\{ T : A \rightarrow \m{type} \}[f : \m{some}-\m{product} x : A . T x] [ a : A] T a \rightarrow \m{type}$.

  
\chapter{Conclusion}
   \section{Results}

In this thesis I designed a logic programming language with a type system based on the
Calculus of Constructions which integrated the notion of an implicitly quantified type in
a manner useful for automating proof search. I demonstrated a series of reductions from
this language to the Calculus of Constructions where the output of the language could
be interpreted as meaningful theorems. I provided an abstract machine for the language
based on higher order unification with proof search, and demonstrated an elaboration
method to this machine. The semantics of the language based on this compilation and
evaluation joined the notions of type inference and traditional evaluation in a way that
does not appear to have been examined in great detail in the past.

I provided a method to constrain proof search of a predicate to a small subset of the
axioms in the environment using families. I demonstrated a way to explicitly control
whether a predicate was searched in a breadth first or depth first manner, allowing
constructs similar to fork and join to be defined.

I gave usage examples of usage of the Caledon language and demonstrated functionality
equivalent to type classes and ways to extend the applicability of this feature using library defined linearity checking.  

Finally, I provided an implementation of this language in Haskell and provided a
standard library. Since previous dependently typed logic languages have not included
polymorphism, standard libraries were not reasonable or possible to include. However,
with the addition of polymorphism, generic lists, type logic, printing, monad and
functor libraries become possible and useful.

   \section{Future Work}

Although this thesis presented a language, little work was done to provide a framework
for proving theorems about this language. In general, compilation for a language
where the programs are theorems for a consistent logic allows for significant optimization
capability. In Twelf, totality, modes, and worlds allowed predicates to be converted
to programs. In general, running of Caledon programs in the current implementation
is excruciatingly slow, as types need to be recorded and searched during runtime. Algorithms
that take advantage of totality checking \citep{altenkirch2010termination}, 
uniqueness checking \citep{anderson2004verifying}, 
worlds checking\citep{anderson2004verifying}, 
mode checking\citep{anderson2004verifying}, 
and universe checking \citep{harper1991type}, 
could be implemented and applied as they were for Twelf and Agda.  It would be useful to have a type system for a logic programming
language which could ensure closed predicates were theorems. More work needs to be
done to automate typeclass instancing as was demonstrated in the section on Linearity.
While implemented, universe checking during unification has yet to be proven entirely
correct.

The possible applications of the language have only been shallowly addressed and
it is clear that much more interesting programs are possible. While derivatives of one
holed types are possible in the language, automatically providing traversals for these
zipper types is an unexplored topic. While I have demonstrated a concise method of
creating concurrency, libraries for controlling concurrency using the IO primitives have yet to be designed.

    
\appendix
\include{appendix} 

\backmatter

%\renewcommand{\baselinestretch}{1.0}\normalsize

% By default \bibsection is \chapter*, but we really want this to show
% up in the table of contents and pdf bookmarks.
\renewcommand{\bibsection}{\chapter{\bibname}}
%\newcommand{\bibpreamble}{This text goes between the ``Bibliography''
%  header and the actual list of references}
\bibliographystyle{plainnat}
\nocite{*}
\bibliography{MyBib} %your bib file

\end{document}
