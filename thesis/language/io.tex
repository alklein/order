\section{IO and Builtin Values and Predicates}
 
This section is both a specification and a guide 
to future implementers of logic programming languages with proof search.

When programming in Caledon, searching for items of type $\m{prop}$ 
might uncover IO, and thus IO can be performed during typechecking.  
This can be understood as the set of axioms differing based on the environment available.

IO is performed when the evaluation function encounters a query for a built-in IO performing function.

\begin{lstlisting}
eval : Proposition -> Environment Formula
eval (a %*$\in$ *) ``print'' Str) = ( if gvar a then print str else ()
                                    ; return (a %* $\doteq $ *) printImp Str)
                                    )
\end{lstlisting}

It is important to include the check that $a$ has not already been resolved so that repeated IO actions are not performed
when nondeterministically proof searching.  

It is tempting to define predicates that take input as ``taking as an argument 
a function that uses the input.''  This is in fact a valid way to define such functions 
and permits for hints of directionality in the types.  
However, it is still possible to escape.  

\begin{lstlisting}
builtin readLine : (string -> prop) -> prop

defn readLineEscape : string -> prop
  as \ s : string . readLine (\ t . t =:= s )

\end{lstlisting}

Ensuring variables do not escape their intended scope is necessary to ensuring
that the intended IO action is only executed once, and not multiple times during proof search.

While it is possible to reason about nondeterministic IO, it is desirable to also have actions
that cannot be executed twice, for which nondeterminism is not possible.

For this, the notion of a monad is useful.  
In this setting, IO is presented as a series of built-in axioms.

\begin{lstlisting}
defn io : prop -> prop
   | bind     = io A -> (A -> io B) -> io B
   | return   = A -> io A
   | ioReadLine = io string
   | ioPrint    = string -> io unit
\end{lstlisting}

We can now no longer write $A \in \m{readLine}$ though since $\m{readLine} : \m{prop}$ 
is a value and thus has no inhabitants. 

Instead, an interpretation predicate can be created which maps these dummy IO actions to real actions.

\begin{lstlisting}
defn run : io A -> A -> prop
  >| runBind = run (bind IOA F) V
            <- run IOA A 
            <- run (F A) V 
  >| runReturn   = run (return V) V
  >| runReadLine = run ioReadLine A <- readLineEscape A
  >| runPrint    = run (ioPrint S) one <- print S
\end{lstlisting}

Since the type system is the ``Calculus of Constructions,'' IO actions constructed
from $io$ will be total, severely limiting their utility.
More complex IO actions and interpreters can be generated, most importantly, ones involving recursion 
or infinite loops.
