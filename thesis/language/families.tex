\section{Type Families}

Permitting entirely polymorphic axioms at the top level significantly
complicates efficient proof search and can introduce unintended falsehoods.
To remedy this, axioms are grouped together by conclusion, ensuring program
definitions are local and not spread out across the source code. Grouping axioms
in this way optimizes a proof search, because it is now possible to limit the search
for axioms to the same family of the head of the goal. Such grouping thus acts as an automated 
and enforced version of the “freeze” command from Twelf. Mutually recursive 
definitions are automatically inferred.

\begin{figure}[H]

\[
\infer[\Pi-\m{ty}]
{
T_2 \; \m{ty}
}{
\Pi x : T_1 . T_2 \m{ty}
}
\]

\[
\infer[?\Pi-\m{ty}]
{
T_2 \; \m{ty}
}{
?\Pi x : T_1 . T_2 \m{ty}
}
\]

\[
\infer[\m{prop}-\m{ty}]
{
\m{prop}\;\m{ty}
}{
}
\]

\[
\infer[\Pi-\m{fam}]
{
\Pi x : T_1 . T_2 @ n
}{
T_2 @ n 
&
x \neq n
}
\]


\[
\infer[?\Pi-\m{fam}]
{
?\Pi x : T_1 . T_2 @ n
}{
T_2 @ n 
&
x \neq n
}
\]


\[
\infer[\m{var}-\m{fam}]
{
n @ n
}{
}
\]

\[
\infer[\m{app}-\m{fam}]
{
T_1 \; T_2 @ n
}{
T_1 @ n
}
\]

\[
\infer[?\m{app}-\m{fam}]
{
T_1 \; \{ x = T_2  \} @ n
}{
T_1 @ n
}
\]
\label{fam:relation}
\caption{The family relationship}
\end{figure}

In figure \ref{fam:relation}, using the “fam” relation ensures that an axiom belongs to a family,
whereas the “ty” relation ensures that a family type actually results in a type.

Grouping axioms together as families and preventing entirely polymorphic results
ensures that entirely polymorphic results will be consistent. It is possible to 
relax this constraint at the expense of full program speed. Speed results because
the only time a polymorphic axiom is introduced into the proof search context is 
locally within an axiom’s assumptions.
