\section{Logic Programming}

Logic programming languages such as Prolog and Twelf were initialy proposed as tools for AI.  
It was later discovered that when properly constrained, these languages merely represented an alternate
formulation of imperative code with powerful pattern matching, and backtracking tools.  The Caledon language
is a logic programming language in the style of Twelf.  In this section I present some basic intution of logic 
programming, rather than explaining it technically, and demonstrate the descriptive
power of the system implemented in Caledon.

\subsection{Basics}
Let us start out by defining addition on unary numbers in Caledon, ignoring types for the moment.

\begin{lstlisting}
defn add : nat -> nat -> nat -> prop
  >| addZ = add zero A A
  >| addS = add (succ A) B (succ C) 
             <- add A B C
\end{lstlisting}

You might notice that this definition is incredibly similar to its Haskell counterpart.

\begin{lstlisting}
add :: nat -> nat -> nat
add Zero a = a
add (Succ a) b = Succ c
   where c = add a b
\end{lstlisting}

We can read the logic programming definition as we would read the functional definition with pattern
matching, knowing that an intelligent compiler would be able to convert the first into the second.  

Of course there are exceptions to the rule, as generalized pattern matching and 
goal search allows one to define essentially nondeterministic programs.  A common use for logic programming
has been to search for solutions to combinitorial games such as tic-tac-toe, without the programmer worrying about the order
of the search.  As this tends to produce inneficient code, this use style is discouraged.  Rather, a more imperative 
view of logic programming is encouraged where pattern match and search is performed in the order it appears.  

In this view, a program of the form

\begin{lstlisting}
query prg = p t1 ... tn

defn p : T_1 -> ... -> T_r -> prop
  >| n1 = p T_1 ... T_r <- p_1,1 ... <- p_1,k_1
...
  >| nN = p T_1 ... T_r <- p_n,1 ... <- p_n,k_n
\end{lstlisting}

should be considered a program which first attempts to prove using axiom n1 by matching prg with ``p T1 ... Tr'' and then
attempting to prove $p_{1,1}$ and so on.    With enough understanding of this reading of a logic program, generalizations might be made
to allow for more control of the search.  In the following snippet of the code, the distinction between breadth first and depth first
queries are used to emulate the concept of concurrency and cause the program to have more complex behavior.

\begin{lstlisting}
query main = runBoth false

defn runBoth : bool -> prop
  >| run0 = runBoth A 
             <- putStr ``ttt ``
             <- A =:= true

   | run1 = runBoth A
             <- putStr ``vvvv''
             <- A =:= true

   | run2 = runBoth A
             <- putStr ``qqqq''
             <- A =:= true

  >| run3 = runBoth A
             <- putStr `` jjj''
             <- A =:= false
\end{lstlisting}

In this example, the query main should print to the screen something similar to ``ttt vqvqvqvq jjj''.  This is because
despite proof search failing on the first three axioms due to an incorrect match, 
the fail is defered until after io has been performed.  The middle axioms, run1 and run2 are declared to be
breadth first axioms while run0 and run3 are declared to be depth first axioms.  


\subsection{Higher Order Abstract Syntax}

Just as imperative programs benefit from the addition of higher order functions, logic programs benefit from the addition of both
higher order predicates and patterns.  

A common request by functional programming language users is that they would like to be able to abstract patterns even more than just over
arguments.  

Examine the following example:

\begin{lstlisting}
func (Var a) = code1
func (Forall var val) = Exists var code
func (Forall var val) = Forall var code
func (And a b) = code2
\end{lstlisting}

A serious amount of code is repeated in lines 2 and 3, but in common languages it is impossible to simplify this.
What a programmer would actually like to express is the following:

\begin{lstlisting}
func (Var a) = code1
func (f var val) = f var code
func (And a b) = code2
\end{lstlisting}

where $f$ is a constructor in strong head normal form.  

In higher order logic programming languages like $\lambda$ Prolog, this is possible to express:

\begin{lstlisting}
defn func : term -> term -> prop 
  | f1 = func (var A) R <- [code1]
  | f2 = func (F Var Val) (f Var R) <- [code]
  | f3 = func (and A B) R <- [code2]
\end{lstlisting}


\subsection{Higher Order Programming}

Fortunately, higher order functions need not be restricted to patterns.  Lisp style macros provide even further ways
to generalize code.  

A great example is the function application operator from Haskell.  
We can define this in Caledon as follows.

\begin{lstlisting}
fixity right 0 $
defn $ : {at bt:prop} (at -> bt) -> at -> bt
  as ?\ at bt . \ f . \ a . f a

defn func : term -> term -> prop 
  | f1 = func (var A) R <- [code1]
  | f2 = func (F Var Val) $ f Var R <- [code]
  | f3 = func (and A B) R <- [code2]
\end{lstlisting}

In many cases, allowing these macros allows for significant simplification of syntax.
