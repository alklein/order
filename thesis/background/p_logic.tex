\section[Logic Primer]{Logic Programming Primer}

%------------------------------------------------
\subsection[Basics]{Learning by Addition}
%------------------------------------------------

\begin{frame}[fragile]
\frametitle{Logic Programming Basics}
\begin{lstlisting}
defn add : nat -> nat -> nat -> prop
   | addZ = add zero A A
   | addS = add (succ A) B (succ C) 
             <- add A B C
\end{lstlisting}
\end{frame}

%------------------------------------------------

\begin{frame}[fragile]
\frametitle{Comparing it to Haskell}

\begin{lstlisting}
add :: nat -> nat -> nat
add Zero a = a
add (Succ a) b = Succ c
   where c = add a b
\end{lstlisting}
\end{frame}

%------------------------------------------------
\subsection[Conceptualizing]{Conceptualizing Logic Programs}
%------------------------------------------------

\begin{frame}
\frametitle{Conceptualizing}
\begin{itemize}
\item Search allows for nondeterministic programs.
\item Finding solutions for search games - tic-tac-toe
\item A procedural view can be used
\end{itemize}
\end{frame}

%------------------------------------------------

\begin{frame}[fragile]
\frametitle{Procedural View}

\begin{lstlisting}
defn actNormal : bool -> prop
  >| truth = actNormal true 
            <- print ``first truth''
            <- print ``then the world''
  >| lies = actNormal false
            <- print ``You've failed me''

query a = actNormal true
query b = actNormal false
\end{lstlisting}
\end{frame}

%------------------------------------------------

\begin{frame}[fragile]
\frametitle{Logical View}

Can decide later on whether to match.

\begin{lstlisting}
defn failLater : prop 
  >| c1 = failLater 
           <- print ``prints anyway'' 
           <- false == true
  >| c2 = failLater
           <- print ``then this''

query prg = failLater
\end{lstlisting}
\end{frame}

%------------------------------------------------

\begin{frame}[fragile]
\frametitle{Generalized Patterns}

Enables trivial equality case

\begin{lstlisting}
defn same_nat : nat -> nat -> prop
  >| is_same_nat = same_nat A A
\end{lstlisting}

Haskell would be: 

\begin{lstlisting}
same_nat Zero Zero = True
same_nat (Succ a) (Succ b) = same_nat a b
same_nat _ _ = False
\end{lstlisting}
\end{frame}

%------------------------------------------------

\begin{frame}[fragile]
\frametitle{Predicates From Functions}

We can turn addition into an even test trivially.

\begin{lstlisting}
defn even : nat -> prop
  >| is_even = even B <- add A A B

defn odd : nat -> prop
  >| is_odd = odd B <- even (succ B)
\end{lstlisting}
\end{frame}

%------------------------------------------------
\subsection[HOLP]{Higher order logic programming}
%------------------------------------------------

\begin{frame}[fragile]
\frametitle{Function Match}

Here's some unnecessarily verbose code.

\begin{lstlisting}
func (Var a) = code1
func (Forall var val) = Exists var code
func (Exists var val) = Forall var code
func (And a b) = code2
\end{lstlisting}
\end{frame}

%------------------------------------------------

\begin{frame}[fragile]
\frametitle{Logic Version}

Here's the logic solution

\begin{lstlisting}
defn func : term -> term -> prop 
  | f1 = func (var A) R <- [code1]
  | f2 = func (F Var Val) Result <- [code]
  | f3 = func (and A B) R <- [code2]
\end{lstlisting}
\end{frame}

%------------------------------------------------

\begin{frame}[fragile]
\frametitle{Computation with Free Variables}

We can also use higher order logic programming to make computations even when 
there are free variables.

\begin{lstlisting}
defn conv : term -> term -> prop
   | c_lam = conv (lam F) (lam F') 
            <- [x : term] conv (F x) (F' x)
\end{lstlisting}
\end{frame}

%------------------------------------------------
\subsection{Type Classes}
%------------------------------------------------

\begin{frame}[fragile]
\frametitle{Type classes in Haskell}

Type classes in haskell allow you to append constraints
to paramaterized values.

\begin{lstlisting}
show :: Show a => a -> String
\end{lstlisting}
\end{frame}

%------------------------------------------------

\begin{frame}[fragile]
\frametitle{Type classes in Caledon}

In Caledon we do this with an implicit argument.
\begin{lstlisting}
defn show : showC A => A -> String
defn show : {unused : showC A } A -> string
\end{lstlisting}
\end{frame}

%------------------------------------------------

\begin{frame}[fragile]
\frametitle{Modern haskell}

Modern haskell allows for logic style type class computation.
\verb|res| will result in ``SSSZ'' despite both
the arguments and cases for add being undefined.

\begin{lstlisting}
data Z
data S a

instance Show Z where
   show _ = ``Zero''
instance Show a => Show (S a) where
   show _ = show (undefined :: a)

class Add a b c | a b -> c where
   add :: a -> b -> c
instance Add Z a a
instance Add a b c => Add (S a) b (S c)

res = show (add (undefined :: S (S Z)) 
                (undefined :: S Z)) 
\end{lstlisting}
\end{frame}

%------------------------------------------------

\begin{frame}[fragile]
\frametitle{Caledon Equivalent}

We can see that \verb|add| actually performs no computation.

\begin{lstlisting}
defn nat2 : nat -> prop
   | any_nat2 = {r : nat} nat2 r

defn add : {a b c : nat}{ r : add a b c } 
           nat2 a -> nat2 b -> nat2 c
  as ?\ a b c . ?\ adder . \ na . \ nb . 
      any_nat2 c

defn show_nat2 : nat2 A -> string -> prop
   | show-A = show_nat2 (any_nat2 : nat2 A) St 
           <- show_nat A St

query res = findOne $ show_nat2 $ 
      add (any_nat2 : nat2 (succ (succ zero)))
          (any_nat2 : nat2 (succ zero))
                                 
\end{lstlisting}
\end{frame}
